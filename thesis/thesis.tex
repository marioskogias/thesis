% DOCUMENT FORMAT ======================= -*- mode: LaTeX; coding: utf-8 -*- ===

\documentclass[diploma]{softlab-thesis}


% PACKAGE SETTINGS =============================================================

\usepackage{fontspec}
\usepackage{amsmath}
\usepackage{amsfonts}
\usepackage{multirow}
\usepackage{array}
\usepackage{mdwlist}
\usepackage{subfig}
\usepackage{floatrow}
%\usepackage{float}
\usepackage{verbatim}
\usepackage{color}
\usepackage{graphicx}
\usepackage{xunicode}
\usepackage{xltxtra}
\usepackage{url}
%\usepackage{dsfont}
%\usepackage{microtype}
\usepackage{hyphenat}
\usepackage{multicol}
\usepackage{wrapfig}
\usepackage{lipsum}
\usepackage{listings}
\usepackage{paralist}
\usepackage{ulem}

% FONT SETTINGS ===============================================================

%\setromanfont[Mapping=tex-text]{CMU Serif}
%%\setromanfont[Mapping=tex-text]{CMU Sans Serif} % temporary change until printing
%%\setsansfont[Mapping=tex-text]{CMU Sans Serif}
%%\setmonofont[Mapping=tex-text]{CMU Typewriter Text}
%\setmainfont[Mapping=tex-text]{CMU Serif}
%%\setmainfont[Mapping=tex-text]{CMU Sans Serif}  % temporary change until printing

%\setromanfont[Mapping=tex-text,ExternalLocation=fonts/]{cmunrm.otf}
%\setsansfont[Mapping=tex-text,ExternalLocation=fonts/]{cmunss.otf}
%\setmonofont[Mapping=tex-text,ExternalLocation=fonts/]{cmuntt.otf}
%\setmainfont[Mapping=tex-text, ExternalLocation=fonts/]{cmunss.otf}

\defaultfontfeatures{Mapping=tex-text}
%\setromanfont{Linux Libertine O}
\setromanfont{Times New Roman}

% CUSTOM COLORS ===============================================================

\definecolor{gray}{rgb}{0.5,0.5,0.5}
\definecolor{darkgreen}{rgb}{0.0,0.5,0.0}
\definecolor{mygreen}{rgb}{0,0.6,0}
\definecolor{mygray}{rgb}{0.5,0.5,0.5}
\definecolor{mymauve}{rgb}{0.58,0,0.82}
\definecolor{myorange}{RGB}{246,177,50}

% CUSTOM COMMANDS =============================================================

\newcommand\fixme{\textrm{\textbf{\textcolor{red}{FIXME: }}}}
\newcommand\todo{\textrm{\textbf{\textcolor{myorange}{TODO: }}}}
\newcommand\mytilde{\raise.17ex\hbox{$\scriptstyle\sim$}}
\newcommand\okeanos{\textbf{\raise.17ex\hbox{$\scriptstyle\sim$}okeanos }}


% Layout macros
\newcommand\spa[1]{\; #1 \;}

% Font macros
\newcommand\resfont[1]{\ensuremath{\mathtt{#1}}}

% Mathematical macros
\newcommand\setmap[3]{#1\{#2 \mapsto #3\}}
\newcommand\getmap[3]{(#2 \mapsto #3) \in #1}
\newcommand\tuple[2]{\ensuremath{\langle#1, #2\rangle}}
\newcommand\mfrac[2]{\ensuremath{\dfrac{#1}{#2}}}
\newcommand\nequiv[2]{\ensuremath{#1 \not\equiv #2}}

% Core Ruby Operational Semantics letter bindings
\newcommand\mem{\mu}

% Core Ruby Operational Semantics low level macros
\newcommand\state[2]{(#1, #2)}
\newcommand\transition[1]{\ensuremath{\overset{#1, c*}{\rightarrow}}}
\newcommand\range[2]{#1, ..., #2}
\newcommand\midrange[5]{\range{#1}{#2}, #3, \range{#4}{#5}}

% Core Ruby Operational Semantics high level macros
\newcommand\operation[5]{\ensuremath{\state{#1}{#2} \transition{#3} \state{#4}{#5}}}
\newcommand\propagation[2]{\operation{#1}{\mem}{#2}{#1'}{\mem'}}

% Core Ruby specific Operational Semantics macros
\newcommand\semicolon[2]{#1; \; #2}
\newcommand\assign[2]{#1 = #2}
\newcommand\mcall[3]{#1.\texttt{#2}(#3)}
\newcommand\ifte[3]{\resfont{if} \; #1 \; \resfont{then} \; #2 \; \resfont{else} \; #3}
\newcommand\newclass[2]{#1.\resfont{new}(#2)}
\newcommand\methoddef[3]{\resfont{def} \; #1(#2) = #3}
\newcommand\classdef[2]{\resfont{class} \; #1 = #2}
\newcommand\with[3]{with \; \tuple{#1}{#2} \; do \; #3}

% Success Typing macros
\newcommand\ssub{\sqsubseteq\_S}

% Core Ruby Success Typing letter bindings
\newcommand\classlist{\Delta}
\newcommand\envir{\Gamma}
\newcommand\fields{\Phi}
\newcommand\currclass{l}

% Core Ruby Success Typing inferencing macros
\newcommand\stinfer[5]{\classlist; \; #1; \; \fields \; \underset{\currclass}{\vdash} \; #2: #3 \; \& \; #4; \; #5}


%%%%%%%%%%%%%%%%%%%%%%%%%%% CACHED STUFF %%%%%%%%%%%%%%%%%%%%%%%%%%%

\newcommand\xcache{\texttt{xcache} }

%%%%%%%%%%%%%%%%%%%%%%%%%%% HASKELL STUFF %%%%%%%%%%%%%%%%%%%%%%%%%%%

%\newcommand\typerep[1]{\ensuremath{#1}}
\newcommand\typerep[1]{\lstinline[basicstyle=\normalsize\ttfamily,keywords={}]|#1|}
\newcommand\typefootrep[1]{\textbf{\lstinline[basicstyle=\footnotesize\ttfamily,keywords={}]|#1|}}
% \newcommand\ttyperep[1]{\typerep{#1}}
% \newcommand\mtyperep[1]{\mbox{\typerep{#1}}}

% Arrow types
\newcommand\typeto[2]{\typerep{#1} \typerep{->} \typerep{#2}}
\newcommand\typetotwo[3]{\ensuremath{\typerep{#1} \typerep{->}
                                     \typerep{#2} \typerep{->}
                                     \typerep{#3}}}

\newcommand\tyconapone[2]{\ensuremath{\mbox{\typerep{#1}} \:\: \mbox{#2}}}
\newcommand\tyconaponeC[2]{\ensuremath{\mbox{\typerep{#1}} \:\: \mbox{\typerep{#2}}}}
% \newcommand\tyconapone[2]{\typerep{#1} $\:$ \typerep{#2}}

\newcommand\tyconaptwo[3]{\ensuremath{\mbox{\typerep{#1}} \:\: \mbox{#2} \:\: \mbox{#3}}}


% FIGURE SETUP ===============================================================

\newcommand\diagram[2]{
	\begin{figure}[h!]
		\centering
		\includegraphics[width=\textwidth,height=\textheight,keepaspectratio]
		{diagrams/#2}
		\caption{#1}
		\label{fig:#2}
	\end{figure}
}

\newcommand\diagramstrict[2]{
	\begin{figure}[H]
		\centering
		\includegraphics[keepaspectratio]
		{diagrams/#2}
		\caption{#1}
		\label{fig:#2}
	\end{figure}
}


% SPELLING =====================================================================

% CODE HIGHLIGHTING ============================================================

% Define common settings for code listings

\lstset{
	backgroundcolor=\color{white},
	basicstyle=\small\ttfamily,		% style for code
	breakatwhitespace=false,        % sets if automatic breaks should only
									% happen at whitespace
	breaklines=true,                % sets automatic line breaking
	captionpos=b,                   % sets the caption-position to bottom
	commentstyle=\color{mygreen},   % style for comments
	escapeinside={\%*}{*)},         % if you want to add LaTeX within your code
	frame=single,                   % adds a frame around the code
	keepspaces=true,                % keeps spaces in text, useful for 
	%keywordstyle=\color{blue}\bfseries,
					% keyword style
	numbers=left,
	numbersep=5pt,                  % how far the line-numbers are from the 
					% code
	numberstyle=\tiny\color{mygray},% style for line-numbers
	rulecolor=\color{black},
	stepnumber=1,                   % the step between two line-numbers. If 
					% it's 1, each line will be numbered
	stringstyle=\color{mymauve},    % style for strings
	tabsize=2,                      % sets default tabsize to 2 spaces
}

% Define specific rules for each language

\lstdefinestyle{c}
{
	language=C,
	tabsize=4
}

\lstdefinestyle{haskell}
{
	language=Haskell
}

\lstdefinestyle{ruby}
{
	language=Ruby
}

\lstdefinestyle{erlang}
{
	language=Erlang.
	captionpos=
}

\lstdefinestyle{plain}
{
	stepnumber=0
}

% Create new commands for simpler usage

\newcommand\ccode[2]{
	\lstinputlisting[float=h!, style=c, caption={#1}, label=lst:#2]{src/#2}
}

\newcommand\haskellcode[3]{
	\lstinputlisting[style=haskell, caption={#1}, label=lst:#2]{src/#3}
}

\newcommand\rubycode[2]{
	\lstinputlisting[style=ruby, caption={#1}, label=lst:#1]{src/#2}
}

\newcommand\erlangcode[2]{
	\lstinputlisting[style=erlang, caption={#1}, label=lst:#1]{src/#1}
}

\newcommand\plaintext[2]{
	\lstinputlisting[float=h!, style=plain, caption={#1}, 
	label=lst:#2]{src/#2}
}

% CHANGE MATH FONT ============================================================

% HYPERREF MUST BE LAST =======================================================

\usepackage[xetex,colorlinks=true,linkcolor=blue,citecolor=darkgreen]{hyperref}

% DOCUMENT INFORMATION =========================================================

\title{Τίτλος} 

% ===============> FIXME
\author{marios}
\authoren{marios}
\datedefense{1}{1}{1111}
\supervisor{test}
\supervisorpos{test}
\committeeone{test}
\committeeonepos{test}
\committeetwo{test}
\committeetwopos{test}
\committeethree{test}
\committeethreepos{test}

\hypersetup{
	pdftitle={},
	pdfauthor={test},
	pdfsubject={},
	pdfkeywords={}
}


% MAIN DOCUMENT ================================================================

\begin{document}

\frontmatter
\maketitle

\def\templen{\parindent}
\setlength{\parindent}{0pt}
\setlength{\parskip}{1.5ex plus 0.5ex minus 0.2ex}
\begin{abstractgr}
Η εξέλιξη της επιστήμης των υπολογιστών καθώς και οι αυξημένες απαιτήσεις σε
υπολογιστικούς πόρους αλλά και χώρους αποθήκευσης δεδομένων, στη ραγδαία
ανάπτυξη των κατανεμηνένων συστημάτων. Πλεόν μια υπηρεσία παρέχεται σαν το
αποτέλεσμα διαφορετικών λειτουργιών οι οποίες αλληλεπιδρούν και εκτελούνται σε
διαφορετικούς υπολογιστικούς κόμβους. Ωστόσο, τόσο η αποσφαλμάτωση όσο και οι
παρακολούθηση της σωστής λειτουργίας καθίσταται εξαιρετικά δύσκολη λόγω της
πολυπλοκότητας των συστημάτων. Πλέον η συμπεριφορά ενός κατανεμημένου συστήματος
εξαρτάται από τις εκάστοτες συνθήκες λειτουργίας, οι οποίες πρέπει να ληφθούν
σοβαρά υπόψιν στην λήψη σχεδιαστικών αλλά και βελτιωτικών αποφάσεων.

Η παρούσα διπλωματική μελετά το σχεδιασμό και την ανάπτυξη μιας υποδομής
παρακολούθησης κατανεμημένων εφαρμογών γραμμένων σε C/C++. Ο μηχανισμός
προσφέρει τη δυνατότητα παρακολούθησης της εφαρμογής σε πραγματικό χρόνο καθώς
αυτή εκτελείται σε πλήρεις συνθήκες λειτουργίας ώστε να διευρευνηθεί η
συμπεριφορά της κάτω από διαφορετικά φορτία. Επιπλέον, παρέχεται γραφικό
περιββάλον μέσω του οποίου ο τελικός χρήστης μπορεί να ερευνήσει περαιτέρω το
σύνολο των αιτήσεων καθώς και τις σχέσεις εξάρτησης μεταξύ των υποσυστημάτων που
συνθέτουν το τελικό προς παρακολούθηση σύστημα. Ο μηχανισμός αυτός ονομάζεται
BlkKin και βασίζεται πάνω σε τεχνολογίες ανοιχτού κώδικα, προσπαθώντας να
εκμεταλλευτεί τα δυνατά στοιχεία κάθε υποσυστήματος και συνδιάζοντάς τα να
επιτύχει το ζητούμενο στόχο.

Η συνεισφορά αυτής της διπλωματικής έγγυται στην υλοποίση του μοντέλου
καταγραφής συγκεκριμένα για εφαρμογές χαμηλής επιβάρυνσης, καθώς και των
συνδετικών τμημάτων μεταξύ των διάφορων υποσυστημάτων που συνολικά υλοποιούν το
BlkKin.

Ο μηχανισμός αυτός χρησιμοποιήθηκε για την παρακολούθηση αιτήσεων Ε/Ε από
εικονικές μηχανές πάνω από Qemu προς το Archipelago, ενός κατανεμημένου
συστήματος αποθήκευσης σε περιβάλλον υπολογιστικού νέφους (cloud computing
environment). Η πορεία μιας τέτοια αίτησης καταγράφηκε μέχρι να ικανοποιηθεί από
το κατανεμηνένο σύστημα αποθήκευσης block, το RADOS. Επομένως δίνεται η
δυνατότητα γραφικής απεικόνησης των διαφορετικών στρωμάτων λογισμικού που
απαιτούνται να συνεργαστούν για την ικανοποίηση της συγκεκριμένης αίτησης.	
    \begin{keywordsgr}
    καταγραφή αιτήσεων, παρακολούθηση, κατανεμημένο σύστημα αποθήκευσης, RADOS,
    Archipelago, LTTng, Dapper
	\end{keywordsgr}
\end{abstractgr}

\begin{abstracten}
Distributed storage systems require special treatment concerning their
monitoring and tracing. In this thesis, we present the design and implementation
of BlkKin, a mechanism that provides the necessary infrastructure for tracing
software-defined storage systems. It enables live tracing and inserts minimal
overhead to the instrumented system, so that it can continue working effectively
in production scale. Its tracing semantics lead to a cross-layered, end-to-end
representation of the system and how the IO requests interact with it.
End-to-end request tracing enables elaborate information extraction concerning
specific system parts, specific workloads or specific system resources, allowing
the, otherwise impossible, localization of latencies and bottlenecks. BlkKin is
based on open-source technologies and provides a full stack implementation with
a data collector, data aggregator and a Web UI for visualizing the tracing
information, while it can be easily incorporated by any system, not only
distributed storage, in need of such a tracing framework.
	\begin{keywordsen}
    distributed storage, tracing, real-time, low-latency, low-overhead, LTTng, 
    Zipkin, Dapper, metrics, RADOS, Archipelago
	\end{keywordsen}
\end{abstracten}

\begin{acknowledgementsgr}
Με την παρούσα διπλωματική ολοκληρώνεται μια πορεία 5 χρόνων στη Σχολή
Ηλεκτρολόγων Μηχανικών και Μηχανικών Υπολογιστών. Διάστημα μέσα στο οποίο ήρθα
αντιμέτωπος με πολλές διαφορετικές προκλήσεις αλλά και δυσκολίες, η αντιμετώπιση
των οποίων όμως με έκανε να καταλάβω τι πραγματικά σημαίνει να είσαι μηχανικός.

Η διπλωματική αυτή, που αποτελεί το επιστέγασμα αυτής της προσπάθειας,
εκπονήθηκε υπό την καθοδήγηση του καθηγητή Νεκτάριου Κοζύρη, τον οποίο θα ήθελα
να ευχαριστήσω θερμά γιατί ήταν αυτός που από το μάθημα της Αρχιτεκτονικής
Υπολογιστών ακόμα με εισήγαγε στην έννοια των υπολογιστικών συστημάτων και γιατί
μου έδωσε τη δυνατότητα να ασχoληθώ με το συγκεκριμένο τομέα των κατανεμημένων
συστήματων και του cloud computing. 

Επιπλέον, οφείλω ένα μεγάλο ευχαριστώ στον Δρα. Βαγγέλη Κούκη και του οποίου το
πάθος να λύσουμε πραγματικά προβλήματα οδήγησε στην υλοποίηση της παρούσας
διπλωματικής και σε ένα εργαλείο το οποίο ξεφεύγει από τα πλαίσια της θεωρίας
και μπορεί να αξιοποιηθεί στην πράξη. Οι συζητήσεις μαζί του πάντα έδιναν λύσεις
στα προβλήματα με τα οποία ήρθα αντιμέτωπος και κατάφεραν να μου μεταδώσουν την
αγάπη για τα υπολογιστικά συστήματα.

Δεδομένου ότι μεγάλο τμήμα της συγκεκριμένης δουλειάς πραγματοποιήθηκε στα πλαία
του προγράμματος Google Summer of Code 2014, θεωρώ υποχρέωσή μου να ευχαριστήσω
τον μέντορά μου στη συγκεκριμένη προσπάθεια Jeremie Galarnau (LTTng project) ο
οποίος πίστεψε στην ιδέα της συγκεκριμένης εργασίας και με στήριξε σε ό,τι αφορά
το σύστημα LTTng.

Ακόμα, θα ήταν παράλειψη να μην ευχαριστήσω τον άνθρωπο με τον οποίο
συνεργάστηκα περισσότερο απ'όλους τον περασμένο χρόνο, τον Υ.Δ Φίλιππο Γιαννάκο,
ο οποίος με βοήθησε έμπρακτα τόσο στην κατανόηση των ποικίλων συστημάτων που
χρησιμοποιήθηκαν όσο και στην σχεδίαση του τελικού μηχανισμού.

Tέλος, θέλω να ευχαριστήσω την οικογένειά μου και τους φίλους μου οι οποίοι
στάθηκαν πάντα δίπλα μου, ανεχόμενοι τις παραξενιές μου, σε όλη της διάρκεια
αυτής της πορείας, όντας πάντα εκεί για να με στηρίξουν, να μου δώσουν δύναμη
να συνεχίσω αλλά και για να μου χαρίσουν ωραίες αναμνήσεις οι οποίες θα με
συντροφεύουν πάντα στη μετέπειτα πορεία μου.
\end{acknowledgementsgr}


\setlength{\parindent}{\templen}
\setlength{\parskip}{0pt}
\tableofcontents
\listoffigures
\listoftables
\renewcommand{\lstlistlistingname}{List of Listings}
\lstlistoflistings % changed the title above

\mainmatter
% moved these two commands here so that they don't influence the toc
\setlength{\parindent}{0pt}
\setlength{\parskip}{1.5ex plus 0.5ex minus 0.2ex}

\renewcommand\floatpagefraction{.7}

\chapter{Introduction}\label{ch:intro}

When back in April 1965 Gordon E. Moore stated the following 
\begin{quotation}
    ``The complexity for minimum component costs has increased at a rate of 
    roughly a factor of two per year. Certainly over the short term this rate 
    can be expected to continue, if not to increase. Over the longer term, the 
    rate of increase is a bit more uncertain, although there is no reason to 
    believe it will not remain nearly constant for at least 10 years. That 
    means by 1975, the number of components per integrated circuit for minimum 
    cost will be 65,000. I believe that such a large circuit can be built on a 
    single wafer.''\cite{Moore}
\end{quotation}

had no idea that he had actually started a race among the academia and the
industry to overcome or at least abide the this law.

At first, since the technology was premature, the evolution in VLSI technology
went hand in hand with the evolution in computer architecture. The more and
faster transistors resulted in achievements in instruction level parallelism
(ILP). From 1975 to 2005 the endeavour put in computer architecture resulted in 
technological advances varying from deeper pipelines and faster clock speeds to
superscalar architectures. But in around 2005 the ILP wall was hit. Transistors
could not be utilized to increase serial performance, logic became too complex
and performance attained was very low compared to power consumption. This lead
to the creation of multicore systems and entered the programmers to the jungle
of parallel software. So far the evolution was almost in accordance with the
famous law. However, in around 2009 to 2011, it was the power wall's time to be
hit. The famous power equation $P=cV^2f$ along with the CPU to memory gap
(eikona) led to the technological burst of distributed and cloud computing.

In 2009 Amazon.com introduced the Elastic Compute Cloud and since then the term
`cloud' is one of the hottest buzzwords not only among the industry and academia
but also among everyday people that take advantage of the `power of cloud'.
Although the term may be vague, the definition of cloud computing, according to
NIST (National Institute of Standards and Technology), is the following:

\begin{quotation}
    ``Cloud computing is a model for enabling ubiquitous, convenient, on-demand
    network access to a shared pool of configurable computing resources (e.g.,
    networks, servers, storage, applications, and services) that can be rapidly
    provisioned and released with minimal management effort or service provider
    interaction.  This cloud model is composed of five essential characteristics
    ,three service models, and four deployment models.''\cite{clouddef}
\end{quotation}

In the previous brief computer chronology, I kept describing bottlenecks and
walls to be overcome. However, it not clear how these bottlenecks become obvious
and how scientists can be sure that they have reached one's technology's limits
before moving on to the next one. The answer to the previous questions has
always been given through tracing. Tracing is a process recording information
about a program's execution, while it is being executed. These information may
be low level metrics like performance counters or time specific metrics in order
to evaluate system's latencies and throughput. Tracing data are mostly useful
for developers and can be used for debugging, performance tuning and performance
evaluation. From the single-cpu, integrated computer to the hundreds-node cloud
infrastructure, trace and performance engineers face challenging problems that
vary from platform to platform, but in any case play a vital role the system's
design and implementation.

Cloud and distributed computing provided trace engineers with more challenging
problems. The system scale is now much greater and program execution is far from
deterministic and can take place in any cluster node. So each program execution
is not bounded to a specific context. Other problems that needed solving
was data and time correlation between the different computing nodes. Also,
unlike single chip platforms that can be individually traced and evaluated,
cloud infrastructures need to be traced with full-load under production
conditions. This set more restrictions concerning the overhead that tracing adds
to the application. Finally, tracing is notorious about the amount of data that
produces. So distributed and cloud tracing demands the use of distributed data
storage systems and processing methods like distributed NOSQL databases and
Map-Reduce frameworks.

So to sum up, as described by any design model, the system verification consists
a major part of a system's implementation and working  process. Verification is
achieved through monitoring and tracing. Depending on the system's nature
tracing and monitoring process and the tools used may vary. Picking the right
tracing tools that will reveal the system's vulnerabilities and faults can be
very demanding and the performance engineer for bringing them to light,
respecting all the prerequisites set by the system.

\section{Thesis motivation}
The motivation behind this thesis emerged from concerns about the storage 
performance of the Synnefo \footnote{www.synnefo.org/} cloud software, which 
powers the \okeanos \footnote{https://okeanos.grnet.gr/} public cloud service 
\cite{okeanos}. I will briefly explain what \okeanos and Synnefo are in the 
following paragraphs.

\okeanos is an IaaS (Infrastructure as a Service) that provides Virtual 
Machines, Virtual Networks and Storage services to the Greek Academic and 
Research community. It is an open-source service that has been running in 
production servers since 2011 by GRNET S.A.
\footnote{Greek Research and Technology Network, https://www.grnet.gr/}

Synnefo \cite{synnefo} is a cloud software stack, also created by GRNET S.A., 
that implements the following services which are used by \okeanos:

\begin{itemize}
    \item \textit{Compute Service}, which is the service that enables the 
        creation and management of Virtual Machines.
    \item \textit{Network Service}, which is the service that provides network 
        management, creation and transparent support of various network 
        configurations.
    \item \textit{Storage Service}, which is the service responsible for 
        provisioning the VM volumes and storing user data.
    \item \textit{Image Service}, which is the service that handles the 
        customization and the deployment of OS images.
    \item \textit{Identity Service}, which is the service that is responsible 
        for user authentication and management, as well as for managing the 
        various quota and projects of the users.
\end{itemize}

Synnefo provides each virtual machine with at least one virtual volume
provisioned by the Volume Service called Archipelago\cite{archip-paper} and will
be furthered detailed in Chapter \label{}. This thesis' purpose is to provide
the developer or the system administrations with a cross-layer representation
accompanied with the equivalent metrics and time information of an I/O request's
route within the infrastructure from the time it is created inside the virtual
machine till it is finally served by the storage backend. The design and
implementation has to be done respecting the following two prerequisites:

\begin{itemize}
    \item The tracing information should be gathered and processed in real-time
        from every node participating in the request serving.
    \item The tracing infrastructure should add the least possible overhead to
        the instrumented system, which should continued working properly 
        production-wise
\end{itemize}

After the end of the tracing infrastructure implementation, the developer should
be able to identify the distinct phases and the duration of each that an IO
request passes through, measure communication latencies between the different
layers and collect all the necessary information (chosen by him) that would help
him understand the full context under which this specific request was served.
All these information can be used for software faults detection and performance
tuning as well as hardware malfunctions and faults like disk or network failures
that would be difficult to detect otherwise.

The novelty of this thesis consists in combining live cross-layer, multi-node
data aggregation, which is typical for monitoring but not for tracing, with the
precision and accuracy of tracing, respecting a hard prerequisite of low
overhead. Previous tracing infrastructures offered only partial solutions. Some
of them would separate the tracing from the working phase because of the great
added overhead, others provided no mechanism for data correlation, while the
traditional monitoring systems did not meet our low-level tracing needs.

The proposed system is called \textit BlkKin. It is designed respected the
aforementioned prerequisites and make use of the latest tracing semantics and
infrastructures employed by great tech companies like Google and Twitter.  

\section{Thesis structure} 
This thesis is structured as follows:


\backmatter
\cleardoublepage % start at the next odd page
\phantomsection % correct hyperlinking
\bibliography{references}
\bibliographystyle{plain}
% \include{glossary}
% \chapter{Appendix}
% \printindex

\end{document}
