% DOCUMENT FORMAT ======================= -*- mode: LaTeX; coding: utf-8 -*- ===

\documentclass[diploma]{softlab-thesis}


% PACKAGE SETTINGS =============================================================

\usepackage{fontspec}
\usepackage{amsmath}
\usepackage{amsfonts}
\usepackage{multirow}
\usepackage{array}
\usepackage{mdwlist}
\usepackage{subfig}
\usepackage{floatrow}
%\usepackage{float}
\usepackage{verbatim}
\usepackage{color}
\usepackage{graphicx}
\usepackage{xunicode}
\usepackage{xltxtra}
\usepackage{url}
%\usepackage{dsfont}
%\usepackage{microtype}
\usepackage{hyphenat}
\usepackage{multicol}
\usepackage{wrapfig}
\usepackage{lipsum}
\usepackage{listings}
\usepackage{paralist}
\usepackage{ulem}
\usepackage{tocvsec2}
\usepackage[toc,page]{appendix}
% FONT SETTINGS ===============================================================

%\setromanfont[Mapping=tex-text]{CMU Serif}
%%\setromanfont[Mapping=tex-text]{CMU Sans Serif} % temporary change until printing
%%\setsansfont[Mapping=tex-text]{CMU Sans Serif}
%%\setmonofont[Mapping=tex-text]{CMU Typewriter Text}
%\setmainfont[Mapping=tex-text]{CMU Serif}
%%\setmainfont[Mapping=tex-text]{CMU Sans Serif}  % temporary change until printing

%\setromanfont[Mapping=tex-text,ExternalLocation=fonts/]{cmunrm.otf}
%\setsansfont[Mapping=tex-text,ExternalLocation=fonts/]{cmunss.otf}
%\setmonofont[Mapping=tex-text,ExternalLocation=fonts/]{cmuntt.otf}
%\setmainfont[Mapping=tex-text, ExternalLocation=fonts/]{cmunss.otf}

\defaultfontfeatures{Mapping=tex-text}
%\setromanfont{Linux Libertine O}
\setromanfont{Times New Roman}

% CUSTOM COLORS ===============================================================

\definecolor{gray}{rgb}{0.5,0.5,0.5}
\definecolor{darkgreen}{rgb}{0.0,0.5,0.0}
\definecolor{mygreen}{rgb}{0,0.6,0}
\definecolor{mygray}{rgb}{0.5,0.5,0.5}
\definecolor{mymauve}{rgb}{0.58,0,0.82}
\definecolor{myorange}{RGB}{246,177,50}

% CUSTOM COMMANDS =============================================================

\newcommand\fixme{\textrm{\textbf{\textcolor{red}{FIXME: }}}}
\newcommand\todo{\textrm{\textbf{\textcolor{myorange}{TODO: }}}}
\newcommand\mytilde{\raise.17ex\hbox{$\scriptstyle\sim$}}
\newcommand\okeanos{\textbf{\raise.17ex\hbox{$\scriptstyle\sim$}okeanos }}


% Layout macros
\newcommand\spa[1]{\; #1 \;}

% Font macros
\newcommand\resfont[1]{\ensuremath{\mathtt{#1}}}

% Mathematical macros
\newcommand\setmap[3]{#1\{#2 \mapsto #3\}}
\newcommand\getmap[3]{(#2 \mapsto #3) \in #1}
\newcommand\tuple[2]{\ensuremath{\langle#1, #2\rangle}}
\newcommand\mfrac[2]{\ensuremath{\dfrac{#1}{#2}}}
\newcommand\nequiv[2]{\ensuremath{#1 \not\equiv #2}}

% Core Ruby Operational Semantics letter bindings
\newcommand\mem{\mu}

% Core Ruby Operational Semantics low level macros
\newcommand\state[2]{(#1, #2)}
\newcommand\transition[1]{\ensuremath{\overset{#1, c*}{\rightarrow}}}
\newcommand\range[2]{#1, ..., #2}
\newcommand\midrange[5]{\range{#1}{#2}, #3, \range{#4}{#5}}

% Core Ruby Operational Semantics high level macros
\newcommand\operation[5]{\ensuremath{\state{#1}{#2} \transition{#3} \state{#4}{#5}}}
\newcommand\propagation[2]{\operation{#1}{\mem}{#2}{#1'}{\mem'}}

% Core Ruby specific Operational Semantics macros
\newcommand\semicolon[2]{#1; \; #2}
\newcommand\assign[2]{#1 = #2}
\newcommand\mcall[3]{#1.\texttt{#2}(#3)}
\newcommand\ifte[3]{\resfont{if} \; #1 \; \resfont{then} \; #2 \; \resfont{else} \; #3}
\newcommand\newclass[2]{#1.\resfont{new}(#2)}
\newcommand\methoddef[3]{\resfont{def} \; #1(#2) = #3}
\newcommand\classdef[2]{\resfont{class} \; #1 = #2}
\newcommand\with[3]{with \; \tuple{#1}{#2} \; do \; #3}

% Success Typing macros
\newcommand\ssub{\sqsubseteq\_S}

% Core Ruby Success Typing letter bindings
\newcommand\classlist{\Delta}
\newcommand\envir{\Gamma}
\newcommand\fields{\Phi}
\newcommand\currclass{l}

% Core Ruby Success Typing inferencing macros
\newcommand\stinfer[5]{\classlist; \; #1; \; \fields \; \underset{\currclass}{\vdash} \; #2: #3 \; \& \; #4; \; #5}


%%%%%%%%%%%%%%%%%%%%%%%%%%% CACHED STUFF %%%%%%%%%%%%%%%%%%%%%%%%%%%

\newcommand\xcache{\texttt{xcache} }

%%%%%%%%%%%%%%%%%%%%%%%%%%% HASKELL STUFF %%%%%%%%%%%%%%%%%%%%%%%%%%%

%\newcommand\typerep[1]{\ensuremath{#1}}
\newcommand\typerep[1]{\lstinline[basicstyle=\normalsize\ttfamily,keywords={}]|#1|}
\newcommand\typefootrep[1]{\textbf{\lstinline[basicstyle=\footnotesize\ttfamily,keywords={}]|#1|}}
% \newcommand\ttyperep[1]{\typerep{#1}}
% \newcommand\mtyperep[1]{\mbox{\typerep{#1}}}

% Arrow types
\newcommand\typeto[2]{\typerep{#1} \typerep{->} \typerep{#2}}
\newcommand\typetotwo[3]{\ensuremath{\typerep{#1} \typerep{->}
                                     \typerep{#2} \typerep{->}
                                     \typerep{#3}}}

\newcommand\tyconapone[2]{\ensuremath{\mbox{\typerep{#1}} \:\: \mbox{#2}}}
\newcommand\tyconaponeC[2]{\ensuremath{\mbox{\typerep{#1}} \:\: \mbox{\typerep{#2}}}}
% \newcommand\tyconapone[2]{\typerep{#1} $\:$ \typerep{#2}}

\newcommand\tyconaptwo[3]{\ensuremath{\mbox{\typerep{#1}} \:\: \mbox{#2} \:\: \mbox{#3}}}


% FIGURE SETUP ===============================================================

\newcommand\diagram[2]{
	\begin{figure}[h!]
		\centering
		\includegraphics[width=\textwidth,height=\textheight,keepaspectratio]
		{diagrams/#2}
		\caption{#1}
		\label{fig:#2}
	\end{figure}
}

\newcommand\diagramscale[3]{
	\begin{figure}[h!]
		\centering
		\includegraphics[scale={#3}]
		{diagrams/#2}
		\caption{#1}
		\label{fig:#2}
	\end{figure}
}
\newcommand\diagramstrict[2]{
	\begin{figure}[H]
		\centering
		\includegraphics[keepaspectratio]
		{diagrams/#2}
		\caption{#1}
		\label{fig:#2}
	\end{figure}
}


% SPELLING =====================================================================

% CODE HIGHLIGHTING ============================================================

% Define common settings for code listings

\lstset{
	backgroundcolor=\color{white},
	basicstyle=\small\ttfamily,		% style for code
	breakatwhitespace=false,        % sets if automatic breaks should only
									% happen at whitespace
	breaklines=true,                % sets automatic line breaking
	captionpos=b,                   % sets the caption-position to bottom
	commentstyle=\color{mygreen},   % style for comments
	escapeinside={\%*}{*)},         % if you want to add LaTeX within your code
	frame=single,                   % adds a frame around the code
	keepspaces=true,                % keeps spaces in text, useful for 
	%keywordstyle=\color{blue}\bfseries,
					% keyword style
	numbers=left,
	numbersep=5pt,                  % how far the line-numbers are from the 
					% code
	numberstyle=\tiny\color{mygray},% style for line-numbers
	rulecolor=\color{black},
	stepnumber=1,                   % the step between two line-numbers. If 
					% it's 1, each line will be numbered
	stringstyle=\color{mymauve},    % style for strings
	tabsize=2,                      % sets default tabsize to 2 spaces
}

% Define specific rules for each language

\lstdefinestyle{c}
{
	language=C,
	tabsize=4
}

\lstdefinestyle{haskell}
{
	language=Haskell
}

\lstdefinestyle{ruby}
{
	language=Ruby
}

\lstdefinestyle{erlang}
{
	language=Erlang.
	captionpos=
}

\lstdefinestyle{plain}
{
	stepnumber=0
}

% Create new commands for simpler usage

\newcommand\ccode[2]{
	\lstinputlisting[float=h!, style=c, caption={#1}, label=lst:#2]{src/#2}
}

\newcommand\pcode[2]{
	\lstinputlisting[float=h!, language=Python, caption={#1},label=lst:#2]{src/#2}
}

\newcommand\bcode[2]{
	\lstinputlisting[float=h!, language=bash, caption={#1},label=lst:#2]{src/#2}
}
\newcommand\cccode[2]{
	\lstinputlisting[style=c, caption={#1}, label=lst:#2]{src/#2}
}

\newcommand\haskellcode[3]{
	\lstinputlisting[style=haskell, caption={#1}, label=lst:#2]{src/#3}
}

\newcommand\rubycode[2]{
	\lstinputlisting[style=ruby, caption={#1}, label=lst:#1]{src/#2}
}

\newcommand\erlangcode[2]{
	\lstinputlisting[style=erlang, caption={#1}, label=lst:#1]{src/#1}
}

\newcommand\plaintext[2]{
	\lstinputlisting[float=h!, style=plain, caption={#1}, 
	label=lst:#2]{src/#2}
}

% CHANGE MATH FONT ============================================================

% HYPERREF MUST BE LAST =======================================================

\usepackage[xetex,colorlinks=true,linkcolor=blue,citecolor=darkgreen]{hyperref}

% DOCUMENT INFORMATION =========================================================

\title{Μηχανισμός παρακολούθησης αιτήσεων με χαμηλή επιβάρυνση σε κατανεμημένο
σύστημα Ε/Ε} 

% ===============> FIXME
\author{Ευάγγελος-Μάριος Κόγιας}
\authoren{Marios Kogias}
\datedefense{16}{10}{2014}
\supervisor{Νεκτάριος Κοζύρης}
\supervisorpos{Καθηγητής ΕΜΠ}
\committeeone{Νεκτάριος Κοζύρης}
\committeeonepos{Καθηγητής Ε.Μ.Π.}
\committeetwo{Νικόλαος Παπασπύρου}
\committeetwopos{Αναπ. Καθηγητής Ε.Μ.Π.}
\committeethree{Δημήτριος Σούντρης}
\committeethreepos{Επίκ. Καθηγητής Ε.Μ.Π.}
\hypersetup{
	pdftitle={},
	pdfauthor={test},
	pdfsubject={},
	pdfkeywords={}
}


% MAIN DOCUMENT ================================================================

\begin{document}

\frontmatter
\maketitle

\def\templen{\parindent}
\setlength{\parindent}{0pt}
\setlength{\parskip}{1.5ex plus 0.5ex minus 0.2ex}
\begin{abstractgr}
Η εξέλιξη της επιστήμης των υπολογιστών καθώς και οι αυξημένες απαιτήσεις σε
υπολογιστικούς πόρους αλλά και χώρους αποθήκευσης δεδομένων, στη ραγδαία
ανάπτυξη των κατανεμηνένων συστημάτων. Πλεόν μια υπηρεσία παρέχεται σαν το
αποτέλεσμα διαφορετικών λειτουργιών οι οποίες αλληλεπιδρούν και εκτελούνται σε
διαφορετικούς υπολογιστικούς κόμβους. Ωστόσο, τόσο η αποσφαλμάτωση όσο και οι
παρακολούθηση της σωστής λειτουργίας καθίσταται εξαιρετικά δύσκολη λόγω της
πολυπλοκότητας των συστημάτων. Πλέον η συμπεριφορά ενός κατανεμημένου συστήματος
εξαρτάται από τις εκάστοτες συνθήκες λειτουργίας, οι οποίες πρέπει να ληφθούν
σοβαρά υπόψιν στην λήψη σχεδιαστικών αλλά και βελτιωτικών αποφάσεων.

Η παρούσα διπλωματική μελετά το σχεδιασμό και την ανάπτυξη μιας υποδομής
παρακολούθησης κατανεμημένων εφαρμογών γραμμένων σε C/C++. Ο μηχανισμός
προσφέρει τη δυνατότητα παρακολούθησης της εφαρμογής σε πραγματικό χρόνο καθώς
αυτή εκτελείται σε πλήρεις συνθήκες λειτουργίας ώστε να διευρευνηθεί η
συμπεριφορά της κάτω από διαφορετικά φορτία. Επιπλέον, παρέχεται γραφικό
περιββάλον μέσω του οποίου ο τελικός χρήστης μπορεί να ερευνήσει περαιτέρω το
σύνολο των αιτήσεων καθώς και τις σχέσεις εξάρτησης μεταξύ των υποσυστημάτων που
συνθέτουν το τελικό προς παρακολούθηση σύστημα. Ο μηχανισμός αυτός ονομάζεται
BlkKin και βασίζεται πάνω σε τεχνολογίες ανοιχτού κώδικα, προσπαθώντας να
εκμεταλλευτεί τα δυνατά στοιχεία κάθε υποσυστήματος και συνδιάζοντάς τα να
επιτύχει το ζητούμενο στόχο.

Η συνεισφορά αυτής της διπλωματικής έγγυται στην υλοποίση του μοντέλου
καταγραφής συγκεκριμένα για εφαρμογές χαμηλής επιβάρυνσης, καθώς και των
συνδετικών τμημάτων μεταξύ των διάφορων υποσυστημάτων που συνολικά υλοποιούν το
BlkKin.

Ο μηχανισμός αυτός χρησιμοποιήθηκε για την παρακολούθηση αιτήσεων Ε/Ε από
εικονικές μηχανές πάνω από Qemu προς το Archipelago, ενός κατανεμημένου
συστήματος αποθήκευσης σε περιβάλλον υπολογιστικού νέφους (cloud computing
environment). Η πορεία μιας τέτοια αίτησης καταγράφηκε μέχρι να ικανοποιηθεί από
το κατανεμηνένο σύστημα αποθήκευσης block, το RADOS. Επομένως δίνεται η
δυνατότητα γραφικής απεικόνησης των διαφορετικών στρωμάτων λογισμικού που
απαιτούνται να συνεργαστούν για την ικανοποίηση της συγκεκριμένης αίτησης.	
    \begin{keywordsgr}
    καταγραφή αιτήσεων, παρακολούθηση, κατανεμημένο σύστημα αποθήκευσης, RADOS,
    Archipelago, LTTng, Dapper
	\end{keywordsgr}
\end{abstractgr}

\begin{abstracten}
Distributed storage systems require special treatment concerning their
monitoring and tracing. In this thesis, we present the design and implementation
of BlkKin, a mechanism that provides the necessary infrastructure for tracing
software-defined storage systems. It enables live tracing and inserts minimal
overhead to the instrumented system, so that it can continue working effectively
in production scale. Its tracing semantics lead to a cross-layered, end-to-end
representation of the system and how the IO requests interact with it.
End-to-end request tracing enables elaborate information extraction concerning
specific system parts, specific workloads or specific system resources, allowing
the, otherwise impossible, localization of latencies and bottlenecks. BlkKin is
based on open-source technologies and provides a full stack implementation with
a data collector, data aggregator and a Web UI for visualizing the tracing
information, while it can be easily incorporated by any system, not only
distributed storage, in need of such a tracing framework.
	\begin{keywordsen}
    distributed storage, tracing, real-time, low-latency, low-overhead, LTTng, 
    Zipkin, Dapper, metrics, RADOS, Archipelago
	\end{keywordsen}
\end{abstracten}

\begin{acknowledgementsgr}
Με την παρούσα διπλωματική ολοκληρώνεται μια πορεία 5 χρόνων στη Σχολή
Ηλεκτρολόγων Μηχανικών και Μηχανικών Υπολογιστών. Διάστημα μέσα στο οποίο ήρθα
αντιμέτωπος με πολλές διαφορετικές προκλήσεις αλλά και δυσκολίες, η αντιμετώπιση
των οποίων όμως με έκανε να καταλάβω τι πραγματικά σημαίνει να είσαι μηχανικός.

Η διπλωματική αυτή, που αποτελεί το επιστέγασμα αυτής της προσπάθειας,
εκπονήθηκε υπό την καθοδήγηση του καθηγητή Νεκτάριου Κοζύρη, τον οποίο θα ήθελα
να ευχαριστήσω θερμά γιατί ήταν αυτός που από το μάθημα της Αρχιτεκτονικής
Υπολογιστών ακόμα με εισήγαγε στην έννοια των υπολογιστικών συστημάτων και γιατί
μου έδωσε τη δυνατότητα να ασχoληθώ με το συγκεκριμένο τομέα των κατανεμημένων
συστήματων και του cloud computing. 

Επιπλέον, οφείλω ένα μεγάλο ευχαριστώ στον Δρα. Βαγγέλη Κούκη και του οποίου το
πάθος να λύσουμε πραγματικά προβλήματα οδήγησε στην υλοποίηση της παρούσας
διπλωματικής και σε ένα εργαλείο το οποίο ξεφεύγει από τα πλαίσια της θεωρίας
και μπορεί να αξιοποιηθεί στην πράξη. Οι συζητήσεις μαζί του πάντα έδιναν λύσεις
στα προβλήματα με τα οποία ήρθα αντιμέτωπος και κατάφεραν να μου μεταδώσουν την
αγάπη για τα υπολογιστικά συστήματα.

Δεδομένου ότι μεγάλο τμήμα της συγκεκριμένης δουλειάς πραγματοποιήθηκε στα πλαία
του προγράμματος Google Summer of Code 2014, θεωρώ υποχρέωσή μου να ευχαριστήσω
τον μέντορά μου στη συγκεκριμένη προσπάθεια Jeremie Galarnau (LTTng project) ο
οποίος πίστεψε στην ιδέα της συγκεκριμένης εργασίας και με στήριξε σε ό,τι αφορά
το σύστημα LTTng.

Ακόμα, θα ήταν παράλειψη να μην ευχαριστήσω τον άνθρωπο με τον οποίο
συνεργάστηκα περισσότερο απ'όλους τον περασμένο χρόνο, τον Υ.Δ Φίλιππο Γιαννάκο,
ο οποίος με βοήθησε έμπρακτα τόσο στην κατανόηση των ποικίλων συστημάτων που
χρησιμοποιήθηκαν όσο και στην σχεδίαση του τελικού μηχανισμού.

Tέλος, θέλω να ευχαριστήσω την οικογένειά μου και τους φίλους μου οι οποίοι
στάθηκαν πάντα δίπλα μου, ανεχόμενοι τις παραξενιές μου, σε όλη της διάρκεια
αυτής της πορείας, όντας πάντα εκεί για να με στηρίξουν, να μου δώσουν δύναμη
να συνεχίσω αλλά και για να μου χαρίσουν ωραίες αναμνήσεις οι οποίες θα με
συντροφεύουν πάντα στη μετέπειτα πορεία μου.
\end{acknowledgementsgr}


\setlength{\parindent}{\templen}
\setlength{\parskip}{0pt}
\tableofcontents
\listoffigures
\listoftables
\renewcommand{\lstlistlistingname}{List of Listings}
\lstlistoflistings % changed the title above

\mainmatter
% moved these two commands here so that they don't influence the toc
\setlength{\parindent}{0pt}
\setlength{\parskip}{1.5ex plus 0.5ex minus 0.2ex}

\renewcommand\floatpagefraction{.7}

\chapter{Introduction}\label{ch:intro}

When back in April 1965 Gordon E. Moore stated the following 
\begin{quotation}
    ``The complexity for minimum component costs has increased at a rate of 
    roughly a factor of two per year. Certainly over the short term this rate 
    can be expected to continue, if not to increase. Over the longer term, the 
    rate of increase is a bit more uncertain, although there is no reason to 
    believe it will not remain nearly constant for at least 10 years. That 
    means by 1975, the number of components per integrated circuit for minimum 
    cost will be 65,000. I believe that such a large circuit can be built on a 
    single wafer.''\cite{Moore}
\end{quotation}

had no idea that he had actually started a race among the academia and the
industry to overcome or at least abide the this law.

At first, since the technology was premature, the evolution in VLSI technology
went hand in hand with the evolution in computer architecture. The more and
faster transistors resulted in achievements in instruction level parallelism
(ILP). From 1975 to 2005 the endeavour put in computer architecture resulted in 
technological advances varying from deeper pipelines and faster clock speeds to
superscalar architectures. But in around 2005 the ILP wall was hit. Transistors
could not be utilized to increase serial performance, logic became too complex
and performance attained was very low compared to power consumption. This lead
to the creation of multicore systems and entered the programmers to the jungle
of parallel software. So far the evolution was almost in accordance with the
famous law. However, in around 2009 to 2011, it was the power wall's time to be
hit. The famous power equation $P=cV^2f$ along with the CPU to memory gap
(eikona) led to the technological burst of distributed and cloud computing.

In 2009 Amazon.com introduced the Elastic Compute Cloud and since then the term
`cloud' is one of the hottest buzzwords not only among the industry and academia
but also among everyday people that take advantage of the `power of cloud'.
Although the term may be vague, the definition of cloud computing, according to
NIST (National Institute of Standards and Technology), is the following:

\begin{quotation}
    ``Cloud computing is a model for enabling ubiquitous, convenient, on-demand
    network access to a shared pool of configurable computing resources (e.g.,
    networks, servers, storage, applications, and services) that can be rapidly
    provisioned and released with minimal management effort or service provider
    interaction.  This cloud model is composed of five essential characteristics
    ,three service models, and four deployment models.''\cite{clouddef}
\end{quotation}

In the previous brief computer chronology, I kept describing bottlenecks and
walls to be overcome. However, it not clear how these bottlenecks become obvious
and how scientists can be sure that they have reached one's technology's limits
before moving on to the next one. The answer to the previous questions has
always been given through tracing. Tracing is a process recording information
about a program's execution, while it is being executed. These information may
be low level metrics like performance counters or time specific metrics in order
to evaluate system's latencies and throughput. Tracing data are mostly useful
for developers and can be used for debugging, performance tuning and performance
evaluation. From the single-cpu, integrated computer to the hundreds-node cloud
infrastructure, trace and performance engineers face challenging problems that
vary from platform to platform, but in any case play a vital role the system's
design and implementation.

Cloud and distributed computing provided trace engineers with more challenging
problems. The system scale is now much greater and program execution is far from
deterministic and can take place in any cluster node. So each program execution
is not bounded to a specific context. Other problems that needed solving
was data and time correlation between the different computing nodes. Also,
unlike single chip platforms that can be individually traced and evaluated,
cloud infrastructures need to be traced with full-load under production
conditions. This set more restrictions concerning the overhead that tracing adds
to the application. Finally, tracing is notorious about the amount of data that
produces. So distributed and cloud tracing demands the use of distributed data
storage systems and processing methods like distributed NOSQL databases and
Map-Reduce frameworks.

So to sum up, as described by any design model, the system verification consists
a major part of a system's implementation and working  process. Verification is
achieved through monitoring and tracing. Depending on the system's nature
tracing and monitoring process and the tools used may vary. Picking the right
tracing tools that will reveal the system's vulnerabilities and faults can be
very demanding and the performance engineer for bringing them to light,
respecting all the prerequisites set by the system.

\section{Thesis motivation}
The motivation behind this thesis emerged from concerns about the storage 
performance of the Synnefo \footnote{www.synnefo.org/} cloud software, which 
powers the \okeanos \footnote{https://okeanos.grnet.gr/} public cloud service 
\cite{okeanos}. I will briefly explain what \okeanos and Synnefo are in the 
following paragraphs.

\okeanos is an IaaS (Infrastructure as a Service) that provides Virtual 
Machines, Virtual Networks and Storage services to the Greek Academic and 
Research community. It is an open-source service that has been running in 
production servers since 2011 by GRNET S.A.
\footnote{Greek Research and Technology Network, https://www.grnet.gr/}

Synnefo \cite{synnefo} is a cloud software stack, also created by GRNET S.A., 
that implements the following services which are used by \okeanos:

\begin{itemize}
    \item \textit{Compute Service}, which is the service that enables the 
        creation and management of Virtual Machines.
    \item \textit{Network Service}, which is the service that provides network 
        management, creation and transparent support of various network 
        configurations.
    \item \textit{Storage Service}, which is the service responsible for 
        provisioning the VM volumes and storing user data.
    \item \textit{Image Service}, which is the service that handles the 
        customization and the deployment of OS images.
    \item \textit{Identity Service}, which is the service that is responsible 
        for user authentication and management, as well as for managing the 
        various quota and projects of the users.
\end{itemize}

Synnefo provides each virtual machine with at least one virtual volume
provisioned by the Volume Service called Archipelago\cite{archip-paper} and will
be furthered detailed in Chapter \label{}. This thesis' purpose is to provide
the developer or the system administrations with a cross-layer representation
accompanied with the equivalent metrics and time information of an I/O request's
route within the infrastructure from the time it is created inside the virtual
machine till it is finally served by the storage backend. The design and
implementation has to be done respecting the following two prerequisites:

\begin{itemize}
    \item The tracing information should be gathered and processed in real-time
        from every node participating in the request serving.
    \item The tracing infrastructure should add the least possible overhead to
        the instrumented system, which should continued working properly 
        production-wise
\end{itemize}

After the end of the tracing infrastructure implementation, the developer should
be able to identify the distinct phases and the duration of each that an IO
request passes through, measure communication latencies between the different
layers and collect all the necessary information (chosen by him) that would help
him understand the full context under which this specific request was served.
All these information can be used for software faults detection and performance
tuning as well as hardware malfunctions and faults like disk or network failures
that would be difficult to detect otherwise.

The novelty of this thesis consists in combining live cross-layer, multi-node
data aggregation, which is typical for monitoring but not for tracing, with the
precision and accuracy of tracing, respecting a hard prerequisite of low
overhead. Previous tracing infrastructures offered only partial solutions. Some
of them would separate the tracing from the working phase because of the great
added overhead, others provided no mechanism for data correlation, while the
traditional monitoring systems did not meet our low-level tracing needs.

The proposed system is called \textit BlkKin. It is designed respected the
aforementioned prerequisites and make use of the latest tracing semantics and
infrastructures employed by great tech companies like Google and Twitter.  

\section{Thesis structure} 
This thesis is structured as follows:

\chapter{Theoretical Background}\label{ch:bkg}

In this chapter we provide the necessary background to familiarize the reader
with the main concepts and mechanism used later in the document. For every
subsystem employed in BlkKin we briefly describe some counterparts justifying
our choice. The approach made is rudimentary, intended to introduce a reader
with elementary knowledge on distributed systems.

Specifically, Section \ref{sec:storage} covers the concepts around distributed
storage systems and the difficulties concerning their monitoring.  In Section
\ref{sec:archip-bkg} we describe Archipelago, Synnefo's Volume Service, and how
IO requests initiated within the virtual machine end up being served by a
distributed storage system. In Section \ref{sec:tracing-bkg} we explain the need
for tracing and cite various open-source tracing systems  with their advantages
and disadvantages. Finally, in Section \ref{sec:logging-bkg} we describe the
different needs covered by logging and cite some popular logging systems.


\section{Distributed storage systems}\label{sec:storage}

Providing reliable, high-performance storage that scales has been an ongoing
challenge for system designers. High-throughput and low-latency storage for file
systems, databases, and related abstractions are critical to the performance of
a broad range of applications. Historically, data centers first created
`islands' of SCSI disk arrays as direct-attached storage (DAS), each dedicated
to an application, and visible as a number of `virtual hard drives' (i.e.
LUNs). Initally, a SAN (Storage-Area-Network) consolidates such storage islands
together using a high-speed network. However, a SAN does not provide file
abstraction, only block-level operations. Also, the cost of scaling a SAN
infrastructure scales exponentially. These boosted the development of more
\emph{service-oriented-architectures}. Emerging clustered storage architectures
constructed from storage bricks or object storage devices (OSDs) seek to
distribute low-level block allocation decisions and security enforcement to
intelligent storage devices, simplifying data layout and eliminating I/O
bottlenecks by facilitating direct client access to data. OSDs constructed from
commodity components combine a CPU, network interface, and local cache with an
underlying disk or RAID, and replace the convention block-based storage
interface with one based on named, variable-length objects. As storage clusters
grow to thousands of devices or more, consistent management of data placement,
failure detection, and failure recovery places an increasingly large burden on
client, controller, or metadata directory nodes, limiting scalability.

One of the design principles of object storage is to abstract some of the lower
layers of storage away from the administrators and applications. Thus, data is
exposed and managed as objects instead of files or blocks. Objects contain
additional descriptive properties which can be used for better indexing or
management. Administrators do not have to perform lower level storage functions
like constructing and managing logical volumes to utilize disk capacity or
setting RAID levels to deal with disk failure. File metadata are explicitly
separate from data and data manipulation is allowed through programmatic
interfaces. These interfaces include CRUD functions for basic read, write and
delete operations, while some object storage implementations go further,
supporting additional functionality like object versioning, object replication,
and movement of objects between different tiers and types of storage. Most API
implementations are ReST-based, allowing the use of many standard HTTP calls.
This results in the abstraction shown in Figure
\ref{fig:object_storage_arch.png}.
 
\diagram{Storage Abstraction}{object_storage_arch.png}

Alhtough they differ substantially concerning their implementation, some of the
most popular examples of such systems are: Amazon S3, OpenStack Swift and RADOS.

However, one common characteristic of all these systems, that led to the
development of this thesis, is that they provide an architecture that easily
scales out, based on APIs, but which is difficult to monitor and find out what
really went wrong in case of a problem. This leads to a di-centralized data
collection and a centralized data processing architecture for tracing
information which is further explained in Chapter \ref{}.

\subsection{RADOS}\label{sec:rados}
RADOS stands for  Reliable, Autonomic Distributed Object Store. It is the object
store component of Ceph\footnote{http://ceph.com/}.Ceph is a free distributed
object store and file system that has been created by Sage Weil for his doctoral
dissertationi\cite{weil-thesis} and has been supported by his company, Inktank,
ever since.  RADOS seeks to leverage device intelligence to distribute the
complexity surrounding consistent data access, redundant storage, failure
detection, and failure recovery in clusters consisting of many thousands of
storage devices.

RADOS basic characteristics are:
\begin{itemize}
\item \textit{Replication}, which means that there can be many copies
of the same object so that the object is always accessible,
even when a node experiences a failure.
\item \textit{Fault tolerance}, which is achieved by not having a
single point of failure. Instead, RADOS uses elected servers
called \textbf{monitors}, each of which have mappings of the
storage nodes where the objects and their replicas are stored.
\item \textit{Self-management}, which is possible since monitors know
at any time the status of the storage nodes and, for example,
can command to create new object replicas if a node experiences
a failure.
\item \textit{Scalability}, which is aided by the fact that there is no
point of failure, which means that adding new nodes
theoretically does not add any communication overhead.
\end{itemize}

Ceph's building blocks can be seen in Figure \ref{fig:ceph.png}

\diagram{Ceph abstraction}{ceph.png}

RADOS operations are based on the following components:
\begin{itemize}
    \item \textit{object store daemons}, which are userspace processes that run 
        in the storage backend and are responsible for storing the data.
    \item \textit{monitor daemons}, which are monitoring userspace processes 
        that run in an odd number of servers that form a Paxos part-time 
        parliament\cite{Paxos}. Their main responsibility is holding and 
        reliably updating the mapping of objects to object store daemons, as 
        well as self-healing when an object store daemon or monitor daemon has 
        crashed.
\end{itemize}

Ceph's logic is based on \textit CRUSH algorithm. According to this algorithm a
map is created, called CRUSH map, which maps objects to store daemons. A
fundamental idea in RADOS is the \textit{placement group} (pg). Placement groups
are used for load balancing. The number of placement groups is predefined. Then,
when we want to create a new object, its name is hashed and assigned to a
specific group. Each placement group makes IO requests to the same OSDs. So,
objects belonging to the same pg, will be replicated across the same OSDs. The
relationship between placement groups and object store daemons is stored in
CRUSH maps that each monitor daemon holds.   
  
Since we would like to instrument RADOS code and measure its performance, apart
from the theoretical background, we should also explain some of its operating
internals, so that further analysis is consolidated. So, in brief, we will try
to explain an IO request's route within a RADOS infrastructure.

Although, as seen in Figure \ref{fig:ceph.png}, RADOS has multiple entry points
(RBD, CephFS, RADOSGW), we are interested in the interaction with librados.
Librados provides a well defined API for data manipulation and control, namely
an API that enables to modify (CRUD) objects and interact with the Ceph
monitors. There are binding for various languages like C, Python and Java. 

Hypothetically, we have an application using librados, which can also run
remotely from the RADOS cluster. The application want to write an objects. So, a
nIO request is initiated from librados. RADOS employs an asynchronous, ordered
point to point message passing library for communication. So, this request is
serialized and a TCP message is created and sent to the RADOS cluster. After
receive, this packet is handled by the equivalent RADOS Messenger classes,
decoded and based on its kind, is placed in a \textit{dispatch queue} to be
served. This specific object belongs to a specific placement group. So, when the
request reaches the top of the queue, based on this pg, the equivalent OSD
undertakes its serving. Based on the replication factor, the equivalent number
of replication requests is sent to other OSDs responsible for the same pg.
During request handling per OSD, based on the request type, there are phases
like \textit{Journal Access} and finally the \textit{Filestore Access}.

From the above analysis, we understood that request processing in RADOS is a
perplexed procedure including multiple remote nodes collaborating. The only way
to understand the internals and debug possible latencies and bottlenecks is
through tracing and this is what we are going to examine further in this thesis.

\section{Archipelago}\label{sec:archip-bkg}

\section{Tracing Systems}\label{sec:tracing-bkg}
Understanding where time has been spent in performing a computation or servicing
a request is at the forefront of the performance analyst’s mind. Measurements
are available from every layer of a computing system, from the lowest level of
the hardware up to the top of the distributed application stack. In recent years
we have seen the emergence of tools which can be used to directly trace events
relevant to performance. This is augmenting the traditional event count and
system state instrumentation, and together they can provide a very detailed view
of activity in the complex computing systems prevalent today.

Event tracing has the advantage of keeping the performance data tied to the
individual requests, allowing deep inspection of a request which is useful when
performance problems arise. The technique is also exceptionally well suited to
exposing transient latency problems. The downsides are increased overheads
(sometimes significantly) in terms of instrumentation costs as well as volumes
of information produced. To address this, every effort is taken to reduce the
cost of tracing - it is common for tracing to be enabled only conditionally, or
even dynamically inserted into the instrumented software and removed when no
longer being used.

In early 1994, a technique called dynamic instrumentation or Dyninst API
\cite{dynist} was
proposed to provide efficient, scalable and detailed data collection for
large-scale parallel applications (Hollingsworth et al., 1994). Being one of the
first tracing systems, the infrastructure built for data extraction was limited.
The operating systems at hand were not able to provide efficient services for
data extraction. They had to build a data transport component to read the
tracing data, using the ptrace function, that was based on a time slice to read
data. A time slice handler was called at the end of each time slice, i.e when
the program was scheduled out, and the data would be read by the data transport
program built on top.

This framework made possible new tools like DynaProf \cite{dynaprof} and
graphical user interface for data analysis. DynaProf is a dynamic profiling tool
that provides a command line interface, similar to gdb, used to interact with
the DPCL API and to basically control tracing all over your system.

Kernel tracing brought a new dimension to infrastructure design, having the
problem of extracting data out of the kernel memory space to make it available
in user-space for analysis. The K42 project \cite{k42}  used shared
buffers between kernel and user space memory, which had obvious security issues.
A provided daemon waked up periodically and emptied out the buffers where all
client trace control had to go through. This project was a research prototype
aimed at improving tracing performance. Usability and security was simply
sacrificed for the proof of concept. For example, a traced application could
write to these shared buffers and read or corrupt the tracing data for another
application, belonging to another user.

In the next sections, recent tracers and how they built their tracing
infrastructure will be examined.

\subsection{Magpie}

One of the earliest and most comprehensive event tracing frameworks is Magpie
\cite{magpie}.  This project builds on the Event Tracing for Windows
infrastructure which underlies all event tracing on the Microsoft Windows
platform. Magpie is aimed primarily at workload modelling and focuses on
tracking the paths taken by application level requests right through a system.
This is implemented through an instrumentation framework with accurate and
coordinated timestamp generation between user and kernel space, and with the
ability to associate resource utilisation information with individual events.

The Magpie literature demonstrates not only the ability to construct high-level
models of a distributed system resource utilisation driven via Magpie event
tracking, but also provides case studies of low-level performance analysis,
such as diagnosing anomalies in individual device driver performance. Magpie
utilises a novel concept in behavioural clustering, where requests with similar
behaviour (in terms of temporal alignment and resource consumption) are
grouped. This clustering underlies the workload modelling capability, with each
cluster containing a group of requests, a measure of “cluster diameter”, and
one selected “representative request” or “centroid”. The calculation of cluster
diameter indicates deep event knowledge and inspection capabilities, and
although not expanded on it implies detailed knowledge of individual types of
events and their parameters. This indicates a need for significant user
intervention to extend the system beyond standard operating system level
events.

As an aside, it is worth noting here that, for the first time, we see in Magpie
the use of a binary tree graph to represent the flow of control between events
and sub-events across distinct client/server processes and/or hosts.

\subsection{DTrace}
Then, Sun Microsystemsa released, in 2005, DTrace \cite{dtrace} which offers the
ability to dynamically instrument both user-level and kernel-level software. As
part of a mass effort by Sun, a lot of tracepoints were added to the Solaris 10
kernel and user space applications. Projects like FreeBSD and NetBSD also ported
dtrace to their platform, as later did Mac OS X. The goal was to help developers
find serious performance problems. The intent was to deploy it across all
Solaris servers and to use it in production.  If we look at the DTrace
architecture, it uses multiple data providers, which are basically probes used
to gather tracing data and write it to memory buffers.  The framework provides a
user space library (libdtrace) which interacts with the tracer through ioctl
system calls. Through those calls, the DTrace kernel framework returns specific
crafted data for immediate analysis by the dtrace command line tool. Thus, every
interaction with the DTrace tracer is made through the kernel, even user space
tracing.  On a security aspect, groups were made available for different level
of user privileges. You have to be in the dtrace proc group to trace your own
applications and in the dtrace kernel group to trace the kernel.  A third group,
dtrace user, permits only system call tracing and profiling of the user own
processes.  This work was an important step forward in managing tracing in
current operating systems in production environment. The choice of going through
the kernel, even for user space tracing, is a performance trade-off between
security and usability.

\subsection{SystemTap}
In early 2005, Red Hat released SystemTap \cite{systemtap} which also
offers dynamic instrumentation of the Linux kernel and user applications. In
order to trace, the user needs to write scripts which are loaded in a tapset
library. SystemTap then translates these in C code to create a kernel module.
Once loaded, the module provides tracing data to user space for analysis.
Two system groups namely stapdev and stapusr are available to separate possible
trac- ing actions. The stapdev group can do any action over Systemtap
facilities, which makes it the administrative group for all tracing control (Don
Domingo, 2010) and module creation.
The second group, stapusr, can only load already compiled modules located in
specific protected directories which only contain certified modules.
The project also provides a compile-server which listens for secure TCP/IP
connections using SSL and handles module compilation requests from any certified
client. This acts as a SystemTap central module registry to authenticate and
validate kernel modules before loading them.
This has a very limited security scheme for two reasons. First, privileged
rights are still needed for specific task like running the compilation server
and loading the modules, since the tool provided by Systemtap is set with the
setuid bit. Secondly, for user space tracing, only users in SystemTap’s group
are able to trace their own application, which implies that a privileged user
has to add individual users to at least the stapusr group at some point in time,
creating important user management overhead.
It is worth noting that the compilation server acts mostly as a security barrier
for kernel module control. However, like DTrace, the problem remains that it
still relies on the kernel for all tracing actions. Therefore, there is still a
bottleneck on performance if we consider that a production system could have
hundreds of instrumented applications tracing simultaneously. This back and
forth in the kernel, for tracing control and data retrieval, cannot possibly
scale well.

\section{Logging Systems}\label{sec:logging-bkg}

Logs are a critical part of any system, they give you insight into what a system
is doing as well what happened. Unlike tracing, log data are not low-level and
do not refer to the system's performance and there is no special care about the
overhead that logging add to the system. Virtually every process running on a
system generates logs in some form or another. Usually, these logs are written
to files on local disks. When your system grows to multiple hosts, managing the
logs and accessing them can get complicated. Searching for a particular error
across hundreds of log files on hundreds of servers is difficult without good
tools. A common approach to this problem is to setup a centralized logging
solution so that multiple logs can be aggregated in a central location.

There are various options for log data aggregation as well as for visualizing
the aggregated data. Some of them are cited here:

\subsection{Syslog}

Syslog is a standard for computer message logging. It permits separation of the
software that generates messages from the system that stores them and the
software that reports and analyzes them.

Syslog can be used for computer system management and security auditing as well
as generalized informational, analysis, and debugging messages. It is supported
by a wide variety of devices (like printers and routers) and receivers across
multiple platforms. Because of this, syslog can be used to integrate log data
from many different types of systems into a central repository.

Messages are labeled with a facility code (one of: auth, authpriv, daemon, cron,
ftp, lpr, kern, mail, news, syslog, user, uucp, local0 ... local7) indicating
the type of software that generated the messages, and are assigned a severity
(one of: Emergency, Alert, Critical, Error, Warning, Notice, Info, Debug).

Implementations are available for many operating systems. Specific configuration
may permit directing messages to various devices (console), files (/var/log/) or
remote syslog servers. Most implementations also provide a command line utility,
often called logger, that can send messages to the syslog. Some implementations
permit the filtering and display of syslog messages.

Syslog is standardized by the IETF in RFC 5424. This standardization specifies a
very important characteristic of Syslog that we would like to have available in
our tracing infrastructure and this is severity levels. Every event to be traced
is associated with a severity level varying from Emergency when the system is
unusable to informational or debug level messages. From the syslog side the
administrator can define which events he is interested about. So, for testing
environments more events should be traced, while for production environments the
events to be traced should be restricted to the absolutely needed. 

\subsection{Scribe}
A new class of solutions that have come about have been designed for high-volume
and high-throughput log and event collection. Most of these solutions are more
general purpose event streaming and processing systems and logging is just one
use case that can be solved using them. They generally consist of logging
clients and/or agents on each specific host. The agents forward logs to a
cluster of collectors which in turn forward the messages to a scalable storage
tier. The idea is that the collection tier is horizontally scalable to grow with
the increase number of logging hosts and messages. Similarly, the storage tier
is also intended to scale horizontally to grow with increased volume. This is
gross simplification of all of these tools but they are a step beyond
traditional syslog options.

One popular solution is
Scribe\footnote{https://github.com/facebookarchive/scribe}. Scribe is a server
for aggregating log data that's streamed in real time from clients. It is
designed to be scalable and reliable.  It was used and released by Facebook as
open source. Scribe is written in C++ and it worths mentioning its transport
layer and how Scribe logging data are processed and finally stored.

Concerning its \textbf{transport} layer, Scribe uses
Thrift\footnote{https://thrift.apache.org/}. The Apache Thrift software
framework, for scalable cross-language services development, combines a software
stack with a code generation engine to build services that work efficiently and
seamlessly between different programming languages. After describing the service
in a specific file (thrift file), the framework is  responsible for generating
the code to be used to easily build RPC clients and servers that communicate
seamlessly across programming languages. For Scribe especially the thrift file
is the following:  

\ccode{Scribe thrift definition file}{scribe.thrift}

In the above file a \texttt Log method is defined, which takes a list of
\texttt LogEntry items as parameter. Every \texttt LogEntry consists of two
strings, a category and a message. This specific Log method can return two
different results codes, either `OK' or `TRY\_LATER'. Based on this file, using
Thrift we can create Scribe clients for every programming language.

Concerning \textbf{data manipulation} Scribe provides the following options.
Based on the message's category, it can store the log entries in different
files, one per category. Also Scribe has Hadoop support and can store the
tracing information to an HDFS so that they can be processed later using
Map-Reduce jobs.

Scribe servers are arranged in a directed graph, with each server
knowing only about the next server in the graph. This network topology allows
for adding extra layers of fan-in, as a system grows and batching messages
before sending them between datacenters as well as providing reliability in case
of intermittent connectivity or node failure. So a Scribe server can operate
either as a terminal server where data are finally stored, or as an intermediate
server that forwards data to the next Scribe server. In case of congestion or of
network problems, data are stored locally and forwarded when the problem is
restored.

\subsection{Graphite} 
Graphite is an enterprise-scale monitoring tool that runs well on cheap
hardware. It is released under the open source Apache 2.0 license and it is
used by many big companies like Google and Canonical. Although Graphite is not
responsible for collecting data, it can store efficiently numeric time-series
data and render graphs of this data on demand. Graphite can cooperate with other
tools like collectd\footnote{https://collectd.org/} for data aggregation.

From an architectural aspect, Graphite consists of 3 software components:

\begin{description}
\item[carbon] - a Twisted daemon that listens for time-series data
\item[whisper] - a simple database library for storing time-series data (similar
in design to RRD)
\item[graphite webapp] - A Django webapp that renders graphs on-demand using
Cairo\footnote{http://www.cairographics.org/}
\end{description}

\subsection{Ganglia}
Ganglia\cite{ganglia} is a scalable distributed monitoring system for high performance
computing systems such as clusters and Grids and grew out of the University of
California, Berkeley. It is based on a hierarchical design targeted at
federations of clusters. It relies on a multicast-based listen/announce protocol
to monitor state within clusters and uses a tree of point-to-point connections
amongst representative cluster nodes to federate clusters and aggregate their
state. It leverages widely used technologies such as XMLfor data representation,
XDR for compact, portable data transport, and RRDtool for data storage and
visualization. It uses carefully engineered data structures and algorithms to
achieve very low per-node overheads and high concurrency. The implementation is
robust, has been ported to an extensive set of operating systems and processor
architectures, and is currently in use on over 500 clusters around the world. 

\diagram{Ganglia architecture}{ganglia-architecture.png}

Ganglia architecture is made up of the following components.

\begin{description}

\item[gmond] The Ganglia MONitor Daemon is a data-collecting agent that you must
install on every node in a cluster. Gmond gathers metrics about the local node
and sends information to other nodes via XML to a browser window. Gmond is
portable and collects system metrics, such as CPU, memory, disk, network and
process data. The Gmond configuration file /etc/gmond.conf controls the Gmond
daemon and resides on each node where Gmond is installed.
    
\item[gmetad] The Ganglia METAdata Daemon is a data-consolidating agent that
provides a query mechanism for collecting historical information about groups of
machines. Gmetad is typically installed on a single, task-oriented server (the
monitoring node), though very large clusters could require more than one Gmetad
daemon. Gmetad collects data from other Gmetad and Gmond sources and stores
their state in indexed RRDtool (round-robin) databases, where a Web interface
reads and returns information about the cluster. The Gmetad configuration file
/etc/gmetad.conf controls the Gmetad daemon and resides on the monitoring node.
    
\item[RRDtool] RRDTool is an open-source data logging and graphing system that
Ganglia uses to store the collected data and to render the graphs for Web-based
reports. Cron jobs that run in the background to collect information from the HP
Vertica monitoring system tables are stored in the RRD database.
    
\item[PHP-based Web interface] — The PHP-based Web interface contains a
collection of scripts that both the Ganglia Web reporting front end and the HP
Vertica extensions use. The Web server starts these scripts, which then collect
HP Vertica‑specific metrics from the RRD database and generate the XML graphs.
These scripts provide access to HP Vertica health across the cluster, as well as
on each host.

\item[Web server] The Web server uses lighttpd, a lightweight http server that
can be any Web server that supports PHP, SSL, and XML. The Ganglia web front end
displays the data stored by Gmetad in a graphical web interface using PHP.
\item[Advanced tools] Gmetric, an executable, is added during Ganglia
installation. Gmetric provides additional statistics and is used to store
user-defined metrics, such as numbers or strings with units.

\end{description}

\section{Conclusion}
To sum up, it is obvious from the previous analysis that the tracing systems
mentioned do not fit in our demands concerning the added overhead to the
instrumented application since their solutions pass through the kernel space.
This extra overhead makes them unsuitable for live tracing. The solution for
for the BlkKin tracing backend was given from the Linux Trace Toolkit - next
generation (LTTng) because it provides separate mechanisms for kernel and user
space tracing. LTTng is furthered examined in Chapter \ref{}.

Concerning the logging systems, we need to imitate their architecture for
BlkKin's architecture, since we need a central trace aggragation point and a UI
that visualizes the information. We can conclude that we need:
\begin{itemize}
\item[tracing daemon] that runs on every cluster node and collects data  with a low-overhead
\item[central data collector] where all tracing information are stored
\item[Web UI] where tracing information are rendered in a way that extracts the necessary
information revealing problems and performance issues in the first place.
For more elaborate information extraction, trace information can be furthered
processed apart from the UI.  
\end{itemize}
For data collection, we are going to use LTTng, while for the data aggregation and the visualization we
are going to use Zipkin, a distributed tracing system created by Twitter. Zipkin
as well as the reasons for our choice are furthered examined in Chapter \ref{}.

\chapter{Linux Trace Toolkit - next generation (LTTng)}\label{ch:lttng}

In this chapter we analyze Linux Trace Toolkin - next generation (LTTng), which
was our choice for BlkKin's tracing backend, and we describe its internal
characteristics that led us to using it. Specifically, we give an overall
outline of its architecture and basic notions in Section
\ref{sec:lttng-overview}. Then, we describe the buffering scheme used both for
kernel and user space (Section \ref{sec:buffers}) and we continue by citing the
kernel and use space implementation mechanism in Sections
\ref{sec:kernel-tracing}, \ref{sec:user-tracing}. Finally we cite the tracing
format used by LTTng (Section \ref{sec:ctf}) and the mechanism for live tracing
in Section \label{sec:relayd}.

\section{Overview}\label{sec:lttng-overview}

Linux Trace Toolkin - next generation is the successor of Linux Trace Toolkit.
It started as the Mathew Desnoyer's PhD dissertation \cite{desnoyer} in École
Polytechnique de Montréal. Since then, it is maintained by EfficiOS
Inc\footnote{http://www.efficios.com/} and the DORSAL lab in  École
Polytechnique de Montréal.

The LTTng project aims at providing highly efficient tracing tools for Linux.
Its tracers help tracking down performance issues and debugging problems
involving multiple concurrent processes and threads. Tracing across multiple
systems is also possible. This toolchain allows integrated kernel and user-space
tracing from a single user interface. It was initially designed and implemented
to reproduce, under tracing, problems occurring in normal conditions. It uses a
linearly scalable and wait-free RCU (Read-Copy Update) synchronization mechanism
and provides zero-copy data extraction. These mechanisms were implemented in
kernel and then ported to user-space as well.
 
Apart from LTTng's kernel tracer and userspace tracer, viewing and analysis
tools are part of the project. In this thesis, we worked with and extended 
\textit{Babeltrace} \footnote{http://lttng.org/babeltrace}.

Except for the fact LTTng is a complete toolchan that can be easily installed in
almost any Linux distribution and the integrated kernel and user space tracing
offered, we chose LTTng because of its minimal performance overhead. Since it
was initially designed to `reproduce, under tracing, problems occurring in
normal conditions', LTTng was the ideal tool to use for real-time low-overhead,
block-storage tracing with BlkKin.

In order to understand how LTTng manages to have such a good performance, we
have to go through its internals. But first, we give an overview of its
architecture and basic components. According to D. Goulet's Master thesis
(\cite{goulet}), LTTng's architecture can be summarized as shown in Figure
\ref{fig:lttng-arch.png}.  

\diagram{LTTng Architecture}{lttng-arch.png}

The \texttt{lttng} command line interface is a small program used to interact
with the session daemon. Possible interaction are creating sessions, enabling
events, starting tracing and so on. The use of this command line tool to create
and configure tracing seesions is further explained in Section
\ref{sec:blk-env} about how to use BlkKin.

Tracing sessions are used to isolate users from each other and create coherent
tracing data between all tracing sources (Ex: MariaDB vs Kernel). This
\textit{session daemon} routes user commands to the tracers and keeps an
internal state of the requested actions. The daemon makes sure that this
internal state is in complete synchronization with the tracers, and therefore no
direct communication with the tracers is allowed other than via the session
daemon.  This daemon is self-contained between users. Each user can run its own
session daemon but only one is allowed per user. No communication happens
between daemons. 

\textit{Consumer daemons} extract data from buffers containing recorded data and
write it to disk for later analysis. There are two separate consumer daemons,
one handling user space and the second one the kernel. A single consumer daemon
handles all the user space (and similarly for kernel space) tracing sessions for
a given session daemon. It is the session daemon that initiates the execution of
the user space and kernel consumer daemons and feeds them with tracing commands.

LTTng internals define and make use of the following concepts in order to create
an abstraction layer between the user and the tracers.
 
\begin{description}

\item[Domains] 
are essentially a type of tracer or tracer/feature tuple.  Currently, there are
two domains in lttng-tools. The first one is \texttt{UST} which is the global
user space domain. Channels and events registered in that domain are enabled on
all current and future registered user space applications. The other domain is
\texttt{KERNEL}.  Three more domains are not yet implemented but are good
examples of the tracer/feature concept. They are UST PID for specific PID
tracing, UST EXEC NAME based on application name and UST PID FOLLOW CHILDREN
which is the same as tracing a PID but follows spawned children.

\item[Session]
is an isolated container used to separate tracing sources and users from each
other. It takes advantage of the session feature offered by the tracer.  Each
tracing session has a human readable name (Ex.: myapps) and a directory path
where all trace data is written. It also contains the user UID/GID, in order to
handle permissions on the trace data and also determine who can
interact with it. Credentials are passed through UNIX socket for that purpose.

\item[Event] 
relates to a TRACE EVENT statement in your application code or in the Linux
kernel instrumentation.  Using the command line tool \texttt{lttng}, you can
enable and disable events for a specific tracing session on a per domain basis.
An event is always bound to a channel and associated tracing context.

\item[Channel]
is a pipe between an information producer and consumer. They existed in the
earlier LTTng tracers but were hardcoded and specified by the tracer. In the
new LTTng 2.0 version, channels are now definable by the user and completely
customizable (size of buffers, number of subbuffer, read timer, etc.).  A
channel contains a list of user specified events (e.g. system calls and
scheduling switches) and context information (e.g. process id and priority).
Channels are created on a per domain basis, thus each domain contains a list of
channels that the user creates.  Each event type in a session can belong to a
single channel. For example, if event A is enabled in channel 1, it cannot be
enabled in channel 2. However, event A can be enabled in channel 2 (or channel
1 but not both) of another session.

\end{description}

\section{Buffering scheme}\label{sec:buffers}

In this part we analyze the buffering scheme employed by LTTng for efficient
tracing.

As mentioned, a channel is a pipe between an information producer and consumer.
It serves as a buffer to move data efficiently. It consists of one buffer per
CPU to ensure cache locality and eliminate false-sharing. Each buffer is made
of many sub-buffers where slots are reserved sequentially.  A slot is a
sub-buffer region reserved for exclusive write access by a probe.  This space
is reserved to write either a sub-buffer header or an event header and payload.
Figure \ref{fig:buffers.png} shows how space is being reserved. On CPU 0, space
is reserved in sub-buffer 0 following event 0. 

\diagram{Channel layout}{buffers.png}

In this buffer, the header and event 0 elements have been complelety written to
the buffer. The grey area represents slots for which associated commit count
increment has been done. Committing a reserved slot makes it available for
reading. On CPU n, a slot is reserved in sub-buffer 0 but is still uncommitted.
It is however followed by a committed event. This is possible due to the non
serial nature of event write and commit operations. This situation happens when
execution is interrupted between space reservation and commit count update and
another event must be written by the interrupt handler.  Sub-buffer 1, belonging
to CPU 0, shows a fully committed sub-buffer ready for reading.


Events written in a reserved slot are made of a header and a variable-sized
payload. The header contains information such as the timestamp associated with
the event and the event type (an integer identifier). The event type
information allows parsing the payload and determining its size. The maximum
slot size is bounded by the sub-buffer size. Both the number of the sub-buffers
and their size can be configured by the \texttt{lttng} command line tool.

In order to synchronize the producer and consumer scheme, LTTng makes use of
atomic operations. The two atomic instructions required are the \texttt{CAS}
(Compare-And-Swap) and a simple atomic increment. Each per-CPU buffer has a
control structure which contains the \textit{write count}, the \textit{read
count}, and an array of \textit{commit counts} and \textit{commit seq counters}.
The counters \textit{commit count} keep track of the amount of data committed in
a sub-buffer using a lightweight increment instruction. The \textit{commit seq}
counters are updated with a concurrency-aware synchronization primitive each
time a sub-buffer is filled. The read count is updated using a standard
SMP-aware \texttt{CAS} operation. This is required because the reader thread can
read sub-buffers from buffers belonging to a remote CPU.

In the next two sections we will present how tracing is achieved in the
different domains, kernel and user space.

\section{Kernel-space tracing}\label{sec:kernel-tracing}

In the previous section we described the buffering scheme used by LTTng. In this
chapter we will analyze the kernel mechanism that enables LTTng to add a minimum
overhead to the instrumented application during tracing or when the tracing is
stopped.

In order to trace the Linux kernel with minimum overhead and without hurting
the performance when the tracing is disabled, the equivalent mechanism should
be provided by the kernel. The initial approach was given through
\textit{Kprobes}\cite{kprobes}. Kprobes are a hardware breakpoint-based
instrumentation approach. The Kprobe infrastructure dynamically replaces each
kernel instruction to instrument with a breakpoint, which generates a trap each
time the instruction is executed. A tracing probe can then be executed by the
trap handler. However, due to the heavy performance impact of breakpoints, the
inability to extract local variables anywhere in a function due to compiler
optimizations, and the maintenance burden of keeping instrumentation separate
from the kernel code, a more elaborate solution was needed.

This solution was given by Mathew Desnoyers with the Linux Kernel Markers
\cite{kmarkers} and Tracepoints infrastructure. The markers and tracepoints
allow us to declare instrumentation statically at the source-code level without
affecting performance significantly and without adding the cost of a function
call when instrumentation is disabled. A marker placed in code provides a hook
to call a function (probe) that can be provided at runtime. A marker can be
`on' (a probe is connected to it) and the function is called so information is
logged, or `off' (no probe is attached). When a marker is `off' it has no
effect, except for adding a tiny time penalty (checking a condition for a
branch). This instrumentation mechanism enables the instrumentation of an
application at the source-code level. Markers consists essentially of a C
preprocessing macro which adds, in the instrumented function, a branch over a
function call. By doing so, neither the stack setup nor the function call are
executed when the instrumentation is not enabled. At runtime, each marker can
be individually enabled, which makes the branch execute both the stack setup
and the function call 

Having extremely low-overhead when instrumentation is dynamically disabled is
crucially important to provide Linux distributions the incentive to ship
instrumented programs to their customers. Markers and tracepoints consist in a
branch skipping a stack setup and function call when instrumentation is
dynamically disabled (dormant). These individual instrumentation sites can be
enabled dynamically at runtime by dynamic code modification, and only add low
overhead when tracing. The typical overhead of a dormant marker or tracepoint
has been measured to be below 1 cycle \cite{marker-perf} when cache-hot. Static
declaration of tracepoints helps manage this instrumentation within the Linux
kernel source-code. Given that the Linux kernel is a distributed collaborative
project, enabling each kernel subsystem instrumentation to be maintained by
separate maintainers helps distributing the burden of managing kernel-wide
instrumentation.  

However, statically declaring an instrumentation site for dynamic activation
typically incurs a non-zero performance overhead due to the test and branch
required to skip the instrumentation call. To overcome this limitation,
Desnoyers created the concept of activating statically compiled code
efficiently by dynamically modifying an immediate operand within an
instruction, which is called Immediate Values \cite{marker-perf}. This
mechanism replaces the standard memory read, loading the condition variable, by
a constant folded in the immediate value encoding of an instruction operand.
This removes any data memory access to test for disabled instrumentation by
keeping all the information encoded in the instruction stream. However, this
involves dynamically modifying code safely against concurrent multiprocessor
accesses. This requires either stopping all processors for the duration of the
modification, or using a more complex, yet more lightweight, core
synchronization mechanism. The choice made was the temporary breakpoint bypass
\cite{bp-bypass}.

In order to overcome a Kernel Markers' drawback, which was the limited type
verification to scalar types because its API is based on format string,
\textit{Tracepoints}
\footnote{https://www.kernel.org/doc/Documentation/trace/tracepoints.txt} were
created.

Two elements are required for tracepoints :
\begin{itemize}
\item A tracepoint definition, placed in a header file.
\item The tracepoint statement, in C code.
\end{itemize}

In order to use tracepoints, you should include \texttt{linux/tracepoint.h}.

Define an event in \texttt{include/trace/events/subsys.h} as shown in Listing
\ref{lst:kernel-event.h}. You can use the Tracepoint within kernel code as shown
in Listing \ref{lst:kernel-use.c}.

\ccode{Kernel event definition}{kernel-event.h}
\ccode{Kernel Tracepoint activation}{kernel-use.c}

As far as LTTng is concerned, the traced data is entirely controlled by the
kernel. However, a mechanism should be provided to interact with the userspace
and the \texttt{lttng} tool and the session daemon. According to
\cite{desnoyer}, the kernel exposes a transport pipeline (Ex: character device
or anonymous file descriptor) and a user space daemon (session daemon) simply
extracts data through this mechanism. This mechanism is based on
\texttt{DebugFS}
\footnote{https://www.kernel.org/doc/Documentation/filesystems/debugfs.txt}

\section{User-space tracing}\label{sec:user-tracing}

User-space tracing needs a different approach from kernel-tracing. Approaches
like SystemTap\footnote{https://sourceware.org/systemtap/} or
DTrace\footnote{http://dtrace.org/blogs/} based user-space tracing on
breakpoints or system-calls whenever a tracing point is reached. However, this
has a severe performance impact on the instrumented application and makes them
inappropriate for live tracing and system monitoring. 

During BlkKin' implementation, we tried to implement a custom tracing mechanism
based on a memory-mapped ring buffer. However, this mechanism should handle
with all the consumer-producer concurrency issues. Inspecting the LTTng
user-space tracer, we found out that the aforementioned buffering scheme
(\ref{sec:buffers}) with the generalized ring buffer is implemented for
user-space tracing as well. This mechanism is not based on breakpoints or
system-calls and does not affect the system's performance. So we decided to
base out backend on LTTng ust-trace. 

As far as LTTng is concerned, while the kernel tracer is the most complex entity
in terms of code and algorithms, it is the simplest to handle. For the session
daemon, this tracer is a single tracing source. However, tracing in user-space
sets challenges concerning multiple users and concurrency. D. Goulet in his
master thesis \cite{goulet} created the \texttt{lttng-tools} project, which
provides the needed unified user and kernel tracing. This project handles with
all the issues concerning multiple concurrent tracing sources and the mechanism
for their synchronization.

Since all these problems are handled by LTTng, in this section we will describe
the mechanism behind a single tracing session.

\diagram{User-space tracer architecture}{ust-architecture.png}

As seen in Figure \ref{fig:ust-architecture.png}, each instrumented application
creates a dedicated thread for tracing. This thread communicated with the
sessiond over a UNIX-domain socket. The creation of this dedicated thread is
created when the instrumented application is launched. Its creation is coded
within functions labeled with \texttt{\_\_attribute\_\_((constructor))}. The
instrumented applications are dynamically linked with the ust libraries. So,
when the object files are loaded, the specific code is executed and the threads
are created. The session daemon communicates with the consumer over a
UNIX-domain socket. Over this path all the control messages pass. For example,
over these UNIX sockets pass the file descriptors of the shared memory segment,
so that the consumer and the instrumented application refer to the same segment.
The elaborate buffering scheme is deployed on a shared memory segment. The
synchronization issues for the access to the segment are handled by the
\texttt{liburcu}\footnote{https://lttng.org/urcu}. Whenever there are data
available, the instrumented application notifies the consumer over a UNIX pipe.
After that the consumer (which is different from the kernel consumer), writes
the tracing data to a local folder. The tracing data will be available for
viewing using viewers like Babeltratrace\footnote{https://lttng.org/babeltrace}
or LTTTV\footnote{https://lttng.org/lttv} only after the end of the session.
This will be furthered discussed in Section \ref{sec:ctf}.

The mechanism that supports the Tracepoints was ported in user-space as well, as
mentioned in \cite{userspace-markers}. The user-space Tracepoints are defined in
a header file. This file is compiled into an object file, which is finally
linked along with the \texttt{liblttng-ust} with the instrumented application.
So, the tracing threads will be created as mentioned and the tracepoint function
calls will trace information only when tracing is enabled.

\section{Common Trace Format (CTF)}\label{sec:ctf}

LTTng makes use of \textit{Common Trace Format}
(CTF)\footnote{http://www.efficios.com/ctf} for the traces created. CTF is a
trace format based on the requirements of the industry. It is the result of the
collaboration between the Multicore
assosication\footnote{http://www.multicore-association.org/} and the Linux
community. This format was created to cover the tracing needs from versatile
communities like the embedded, telecom, high-performance and kernel communities.
It is a  high-level model meant to be an industry-wide, common model, fulfilling
the tracing requirements. It is meant to be application-, architecture-, and
language-agnostic. One major element of CTF is the Trace Stream Description
Language (TSDL) which flexibility enables description of various binary trace
stream layouts. The CTF format is formally described in RFC.

According to this abstract model:

A \textit{trace} is divided into multiple event streams. Each event stream
contains a subset of the trace event types. The final output of the trace, after
its generation and optional transport over the network, is expected to be either
on permanent or temporary storage in a virtual file system. Because each event
stream is appended to while a trace is being recorded, each is associated with a
distinct set of files for output. Therefore, a stored trace can be represented
as a directory containing zero, one or more files per stream.

An \textit{event stream} can be divided into contiguous event packets of
variable size. An event packet can contain a certain amount of padding at the
end. The stream header is repeated at the beginning of each event packet

CTF offers a variety of \textit{data types} for tracing, like integers, arrays
or strings, which are defined in the RFC. Τhese types allow inheritance so
that other types can be derived. 

The overall structure of an event is:

\begin{description}
\item[Event Header] \hfill \\
(as specified by the stream meta-data). These information are the same for all
streams in the trace. Example information: trace UUID
\item[Stream Event Context] \hfill \\ 
(as specified by the stream meta-data) The stream context is applied to all
events within the stream. Example information: pid, payload size
\item[Event Context] \hfill \\
(as specified by the event meta-data) The event context contains information
relative to the current event. Example information: missing fields
\item[Event Payload] \hfill \\
(as specified by the event meta-data) An event payload contains fields specific
to a given event type
\end{description}

As it became obvious, each trace is associated with some metadata. For example,
the trace stream layout description is located in the trace meta-data or the
fields belonging to an event type are described in the event-specific meta-data.
The meta-data is itself located in a separate stream identified by its name:
`metadata'.

The fact that the trace metadata are located in a different stream, prevents an
LTTng `local' trace from being read (reliably) without stopping the tracing
session. LTTng offers no guarantee that the metadata on disk contains all the
layout information needed to read any packet previously flushed to disk. For
example, a new application, instrumented with previously unknown events, could
be launched and fill a buffer with events. That buffer would then be flushed to
disk. At that point, there would be no guarantee that the lttng-sessiond would
have had the chance to flush the updated metadata to disk. Thus, reading that
trace would fail.

In order to read an LTTng CTF-formatted event before the end of the tracing
session, LTTng created \texttt{relayd} and enabled live tracing, which is
furthered analyzed in Section \ref{sec:relayd}.

\section{Live tracing}\label{sec:relayd}

Version 2.4.0 LTTng introduced live tracing support. Instead of waiting the
end of the session in order to read the traces, lttng-live enabled developers to
read the traces live while they are being created using Babeltrace. 

In order to live read trace data, traces have to be streamed, even if the tracer
and the viewer operate on the same machine. Live tracing is achieved through the
use of special daemon called \texttt{relayd}. When creating the tracing session,
you can define whether you prefer live tracing. If so, you have to provide the
IP address of the computer node where realyd runs. When the session starts,
relayd stores data to the remote machine, so that they will be saved after the
end of the session. In order to view the traces, you have to use Babeltrace
which connects to the relayd and prints text data to the stdout, when they
arrive over the network. Again Babeltrace can run on different machine from the
one being traced or the one where relayd runs.

LTTng relayd handles with the previous-mentioned metadata inconsistencies.
Whenever new events appear, for example when a new instrumented application is
launched, the relayd updates the metadata accordingly. As a result, the viewer
(Babeltrace) receives from relayd a data packet with the actual tracing
information and an index packet to properly locate the information. The updated
metadata are also streamed to the client in a separate stream, as already
mentioned. At any point, the live client must have all the metadata associated
with the data packets it receives. The resulting interconnection is seen in
Figure \ref{fig:relayd.png}.

\diagram{LTTng live tracing}{relayd.png} 

\chapter{Tracing semantics}\label{ch:dapper-zipkin}

Apart from the mechanism that enable tracing, it is very important to choose
what kind of information concerning the application is finally logged. A wise
choice will ease the process of data correlation and assumption extraction. In
this chapter, we cite the different schools behind tracing semantics, we analyze
our tracing choice and finally we discuss how this choice is implemented.

\section{Data correlation}\label{sec:data-cor}

Data resulting after tracing can be very bulky. Consequently, the process of
extracting the information needed that triggered tracing is challenging. Data
should be correlated and only the needed parts of the logs should be isolated
and processed in order to extract meaningful information. This requires that
tracing data are capable of being correlated. By data correlation we refer to
data that refer to a specific subsystem or a specific request grouping.

Although there have been proposed many different tracing schemes, according to
specific applications' needs, all these schemes can be summarized into two
categories. According to Google's Dapper paper\cite{dapper} these categories
are:

\begin{description}

\item[black-box] schemes \cite{blackbox1, blackbox2} assume there is no
additional information other than the message record described above, and use
statistical regression techniques to infer that association.

\item[annotation-based] schemes \cite{magpie, xtrace} rely on applications or
middleware to explicitly tag every record with a global identifier that links
these message records back to the originating request.

\end{description}

While black-box schemes are more portable than annotation-based methods, they
need more data in order to gain sufficient accuracy due to their reliance on
statistical inference. The key disadvantage of annotation-based methods is,
obviously, the need to manually instrument programs by adding instrumentation
points in their source code.

As mentioned, BlkKin wants to achieve an end-to-end tracing so that latencies
and faults between the different subsystem layers to become obvious. So, we
decided to use an annotation-based tracing schema.

\section{Dapper tracing concepts}\label{sec:dapper}

Dapper\cite{dapper} is a large scale distributed systems tracing infrastructure
created by Google. It uses an annotation-based tracing scheme, which enables
Google developers to monitor Google infrastructure only by instrumenting a small
set of common libraries (RPC system, control flow). Although it is closed
source, the tracing semantics used in Dapper are publicly available and have
been used in BlkKin's development. Indeed, Google proposed a complete
annotation-based scheme, which describes the following concepts for tracing:

\begin{description}
\item[annotation]
The actual information being logged. There are two kinds of annotations. Either
\emph{timestamp}, where the specific timestamp of an event is being logged or
\emph{key-value}, where a specific key-value pair is being logged.
\item[span]
The basic unit of the process tree. Each specific processing phase can be
depicted as a different span. Each span should have a specific name and a
distinct span id. It is important to note that each span can contain information
from multiple hosts.
\item[trace]
Every span is associated with a specific trace. A different trace id is used to
group data so that all spans associated with the same initial request share the
common trace id. For our case, information concerning a specific IO request
share the same trace id and each distinct IO request initiates a new trace id.
\item[parent span]
In order to depict the causal relationships between different spans in a single
trace, parent span id is used. Spans without a parent span ids are  known as
\emph{root spans}.
\end{description}

The previous concepts fit out demands for end-to-end tracing. So, we implemented
them in a tracing library for C/C++ applications, which is described thoroughly
in Chapter \ref{}.
 
\section{Zipkin: a Dapper open-source implementation}\label{sec:zipkin}

Dapper does not only describe the tracing semantics mentioned before, but is a
full stack tracing infrastructure which includes subsystems to aggregate data
per host, a central collector, a storage service and a user interface to query
across the collected information. BlkKin, also has the same needs. So, to cover
some of them, instead of rewriting the needed subsystems, we decided to use
\textit{Zipkin}.

Zipkin\footnote{http://twitter.github.io/zipkin/} is an open-source
implementation of the Dapper paper by Twitter. It is used to gather timing data
for all the disparate services at Twitter. It manages both the collection and
lookup of this data through a Collector and a Query service as well as the data
presentation through a Web UI. Zipkin is written in Scala, while the UI is
written in Ruby and Javascript using the D3.js\footnote{http://d3js.org/}
framework. So, Zipkin is a full-stack systems that encapsulates the Dapper
tracing semantics out of the box. This is why we chose to use Zipkin.

Concerning transportation, Zipkin uses Scribe, which enables Zipkin to Scale.
So, in order to feed Zipkin with data a Scribe client is needed. As metioned in
Chapter \ref{sec:logging-bkg} about Scribe, a category and a message is needed
to log to Scribe. Zipkin messages are also Thrift encoded so that the collector
and treat them and add them in the database. Zipkin thrift messages are encoded
according to the following thrift file: 

\ccode{Zipkin message thrift definition file}{zipkin.thrift}

This thift file defines:

\begin{description}
\item[Endpoint] is the location when an annotation took place. An endpoint is
identified by its name, ip and port.
\item[Annotation] is the tracing information itself, exactly like the Dapper
annotation
\item[Span] is also the Dapper span identified by its name, id, trace and parent
ids and can contain multiple annotations.
\end{description} 

\chapter{BlkKin Design}\label{ch:design}

In the previous chapters we described the various challenges faced when dealing
with distributed tracing and tools that can be used in order to achieve our
tracing goals. In this chapter we cite the design of the tracing infrastructure
called BlkKin. The name comes from the amalgamation of \textit{Block storage}
and \textit{Zipkin}, which is one of the used building blocks and was described
thoroughly in Section \ref{sec:zipkin}. BlkKin uses different open-source
technologies and is designed to scale. 

By building BlkKin we wanted to create a tracing infrastructure intended to
cover the tracing needs created in software defined and distributed storage
systems (but of course not restricted to them). After investigating the various
needs that this kind of systems and the people developing and monitoring them
have, we tried to summarize the points that are needed for our tracing
infrastructure. We defined the following prerequisites that should be present in
BlkKin:

\begin{description}

\item[low-overhead tracing] \hfill \\
The traced system should be able to continue working in production scale serving
real workloads in order to locate deficiencies and faults that are not obvious
in debugging or tracing mode.

\item[live-tracing]
BlkKin should be able to send traces at the time the are being produced so that
the developer or the administrator can have an overview of the system at that
specific time.

\item[Dapper tracing semantics]
Tracing logic should be implemented in accordance with the concepts used by
Dapper so that causal relationships and cross-layer architecture are depicted.

\item[User interface]
BlkKin should provide various endpoints for the end user to collect and analyze
data. One of those should be a graphical user interface that gives a graphical
overview of the system's performance per specific layer.

\end{description}

In the following sections we will step by step examine BlkKin. Specifically, in
Section \ref{sec:rationale} we will describe the rationale behind Blkin, while
Section \ref{sec:components} goes through BlkKin's building blocks and Section
\ref{sec:contribution} analyzises the BlkKin contributions. Finally, in Section
\ref{sec:flow} we illustrate the resulting BlkKin architecture and the flow of a
traced request from the time created in the instrumented infrastructure until it
ends up handled by BlkKin and stored.

\section{Design rational}\label{sec:rationale}

Since distributed systems follow a service-oriented architecture and consist of
different software layers communicating with each other while running on
different cluster nodes, we have to implement a tracing architecture that
provides distributed tracing on each node, but collects all tracing data to a
central repository. Thus, we can discover the relationships between the
different layers. Otherwise, it would be impossible to find out what actions did
a specific event on a specific node triggered throughout the cluster.

So, imitating the monitoring systems architecture as described in Section
\ref{sec:logging-bkg}, BlkKin consists of the following parts

\begin{description}
\item[tracing agent] \hfill \\
This agent runs on every cluster node. It is responsible for capturing traces
both from user and kernel-space. Is supposed to add minimum overhead to the
instrumented application.

\item[central data collector]
The tracing agents from all the cluster nodes that we are interested about,
should send the aggregated tracing information to a central collecting place so
that they can be correlated. This system includes both the collecting part, a
server that receives the data, and a storage system (database, file system),
where information is finally kept.

\item[User interface]
The aggregated information should be available through a Web user-interface that
depicts the correlation between the systems' distinct software layers. Through
that interface the user should be able to locate the information he wants and
make basic queries. Also, this interface should be able to be used as a
monitoring and alerting tool in case something misoperates.

\end{description}

\section{BlkKin building blocks}\label{sec:components}

After having identified our basic needs for BlkKin and respecting the proverb
\textit{`not to reinvent the wheel'}, we turned to the open-source community to
find projects that met our requirements and we could combine to build BlkKin.

First we came across \textit{Zipkin} (see Section \ref{sec:zipkin} for more).
Zipkin implements the Dapper semantics and provides a mechanism for data
aggregation, data storage and a Web UI. So, we decided to employee Zipkin. Also,
another crucial factor in favor of using Zipkin was the fact that it makes use
of Scribe as a collector server. This is important because instead of storing
tracing information in a database, we can store them in HDFS and run Map-Reduce
jobs on them. An mentioned, tracing is notorious for the amount of data it
produces. In order to manipulate that amount of data, Twitter engineers used
distrubuted NoSQL databases and especially Cassandra. However, data from Zipkin
that are stored in Cassandra follow a specific indexing pattern that is created
in accordance with Zipkin's quering needs. This pattern makes hard (or even
prevents) to run ad-hoc custom queries to extract any kind of correlation or
information such as average values. Even Twitter uses Hadoop for these purposes.
So we could use Zipkin for some visualizing purposes and HDFS for custom data
manipulation using the same collecting mechanism and without the need to change
data for the one or the other purpose. The same data stored in the Zipkin
database and used to depict the causal relationships in the Web UI, can be
stored in HDFS and investigated through Map-Reduce jobs.  

Having covered the data collection and storage and the user-interface part, we
had to create or use a system for the tracing agent. After some some custom
endeavours to create such a system, we concluded that in order to be fast, this
system has to use memory-mapped IO and specifically a ring buffer where a
producer and a consumer exchange data. We created such a mechanism using a
shared memory segment where the instrumented application wrote binary data and
another daemon consumed them. However, we found out that to build such a
mechanism, we had to solve multiple synchronization and concurrency problems and
the situation would become more difficult with multi-threaded applications.
Since BlkKin was designed to cope with production scale environments, we
searched for a tested, open-source technology that covers that need. So, we
decided to use LTTng. An mentioned in Section \ref{sec:user-tracing}, LTTng
enables us to trace both kernel and user-space applications with the same
infrastructure. Since LTTng uses tracepoints inserted in the source code, by
using it we can deploy any tracing logic we want. In our case, we had decided
beforehand to use an annotation-based logic through the Dapper semantics, custom
instrumentation was exactly what we needed. Also, LTTng has live tracing
support, which could make use of. So, instantly another prerequisite was
covered. Finally, like Syslog, LTTng implements different severity levels,
namely a tracepoint can be considered emergency or waring for example. So, this
enabled us to create different tracepoint with different severity levels and in
case of live monitoring enable only the most basic ones, while in case of
extensive tracing all tracepoints should be logged.

Finally, since we design BlkKin to be a distributed tracing infrastructure, we
should take special care about the time skews among the cluster node clocks.
When tracing a single node, a single clock is used. However, distributed
applications create challenges concerning clock synchronizations. If we do not
care about clock synchronization, there is serious possibility to find a request
reply virtually taking place before the request itself, because there is time
skew between the sever and the client. Older approaches (\cite{hp}) used
post-processing techniques in order to adjust the time skews. According to these
techniques, each cluster node collected tracing information based on its local
clock, while a specific cluster node is considered anchor. During the tracing
session, periodically each host sends an echo message to the anchor, and the
anchor replies with the sender's timestamp and the anchor's current timestamp.
The sending host receives the reply and records a time measurement consisting of
three timestamps: send, remote (i.e., anchor), and reply. Relying on the fact
that the network communication time is the same for sending and replying, we can
compute the time skew. After the end of the tracing session these timeskews are
interpolate to create the each host's time-skew line throughout the tracing
session. Theses skews are applied to the logged timestamps before the events end
up to the database. These approaches performed quite well, but their major
disadvantages were the post-processing overhead to calculate the skew, which
could be significant in case of long tracing sessions, and the fact that
live-tracing was impossible due to this needed post processing.

However, when the previous methods were developed, NTPs performance was not
acceptable for distributed tracing. According to \cite{hp} using NTP caused
skews over 1000 $\mu$sec. In 2010 NTP version 4 was developed. According its
RFC\footnote{http://tools.ietf.org/html/rfc5905} NTPv4 NTPv4 includes
fundamental improvements in the mitigation and discipline algorithms that extend
the potential accuracy to the tens of microseconds with modern workstations and
fast LANs. After experimenting with NTPv4 performance and accuracy, we decided
that it was adequate for our tracing needs, namely there was no response
happening before its request, only after a few hours of NTP syncing. One
deployment that we also tried without remarkably better result, was to use a
cluster node as NTP server, since we care only about relative timestamps and not
absolute. Thus, exploiting the fast LAN we waited to have better synchronization
results, but we concluded that even a global NTP server operated well.

\section{BlkKin contribution}\label{sec:contribution}

At that point, after having decided about the different building blocks that
BlkKin would use, we had to find a way to make them communicate. In this section
we will explain how we made the above systems communicating with each other in
order to created a unified tracing infrastructure and what we needed to add in
order to complete BlkKin as an end-to-end, unified tracing infrastructure. This
contribution is returned back to the open-source community and the major part
was created during my participation in the Google Summer of Code 2014 in the
LTTng project, under the supervision of Jeremie Galarneau.

\subsection{BlkKin library}
Since we had decided on the tracing infrastructure, we had to find a way to
trace low-overhead application in accordance with the Dapper semantics. Although
Zipkin provides a variety of instrumentation libraries for languages such as
Java, Scala or Python, there was no such library for C/C++. So we created a
C/C++ library in accordance with the Dapper semantics which provides a useful
API with all the functions that a programmer would need to instrument such an
application. This API is further explained in \ref{}. Although this library in
backend-independent, which means that anybody could just keep the API and
implement a custom tracing infrastructure, we implemented a backed based on
LTTng. So, our library makes use of LTTng \texttt{tracepoint} in order to log
the information we want. In addition, since we wanted to use BlkKin for
monitoring purposes as well, BlkKin library should implement a kind of sampling.
Otherwise, the amount of data created would be huge and the network traffic
would be really high. Although a more elaborate mechanism for sampling could
have been created, BlkKin currently implements only a rate sampling which means
only 1/N root spans are actually initiated. Since we are talking about IO
requests, the route of only one out of N requests will be actually traced.

\subsection{Babeltrace plugins}
As mentioned, Scribe uses Thrift as a communication protocol. So, we needed to
implement a mechanism that would tranform LTTng data and send them to Scribe.
LTTng data are encoded using the CTF format and Babeltrace is responsible for
transforming these data to a human readable format. Consequently, we had to
extend Babeltrace to communicate with Scribe. At first, we tried to implement
this functionality within Babeltrace. So firstly we created a Scribe client
written in C\footnote{https://github.com/marioskogias/scribe-c-client} which is
used in Babeltrace. This version of
Babeltrace\footnote{https://github.com/marioskogias/babeltrace/tree/scribe-client}
was abandoned because Babeltrace internal code architecture was hard to extend.
Instead, we decided to use the Babeltrace Python bindings and deploy this
functionality as a babeltrace plugin. 

So, using these Python bindings and the \texttt{facebook-scribe} module from
\texttt{pip} we created two different plugins. The first plugin in generic and
send to Scribe any kind of LTTng data after transforming them in a \textit{JSON
format}.  Using this plugin, we can avoid the tedious job of searching for
information within log-files. Instead, we can send our data to Scribe and store
them in HDFS. After that creating simple Map-Reduce jobs that read JSON encoded
data can subtract any information we want. The motivation for this functionality
came from the Facebook's equivalent tracing infrastructure called
Scuba\footnote{https://www.facebook.com/notes/facebook-engineering/under-the-hood-data-diving-with-scuba/10150599692628920}
\cite{scuba}. 

Finally, LTTng live tracing supports only CTF to text transformation and the
above plugins could not be used for live tracing. So we had to extend Babeltrace
live support. Because at the spefic moment of BlkKin's development, Babeltrace
underwent a generic code refactoring. After that refactoring the above plugins
would work with live support as well. Consequently, we offered just an
evanescent solution for LTTng live tracing for Zipkin data. According to this
solution, LTTng is obliged to log only specific tracepoints. After that,
Babeltrace sends these data to a Python consumer communicating over a named
Pipe. Implementation details are further explained in Section \ref{}.

\subsection{BlkKin Monitoring UI}
Although Zipkin's UI was adequate for investigating the correlations between the
different storage layers, it didn't cover our needs as an alerting tool. So, we
developed a simple Python-django\footnote{https://www.djangoproject.com/}
application, what communicates with the Zipkin database. This application is
responsible to gather particular metrics, mostly average values, such as the
average network communication over the last 10 minutes, and present them
accordingly. This application includes thresholds for these metrics. If the
values are under the thresholds the UI shows them green, but when the metrics
are over the threshold values, they are illustrated red. These threshold values
are the result from elaborate data analysis over a wide series of data in the
HDFS.

\section{BlkKin architecture and data flow}\label{sec:flow}

In this section we present the overall BlkKin architecture and data flow. As we
can see in Figure \ref{fig:blkin-internal.png} we have an application written in
C/C++ we are interested to trace or monitor. After we have identified the
different application layers and decided how to implement the Dapper semantics
for it, we instrument its source code using the BlkKik library. The host where
the application runs, run LTTng daemons as well. So, whenever an instrumentation
point is reached, a \texttt{tracepoint} logs the tracing information to LTTng.
After that, depending on whether we are having a live tracing session or not,
the CTF data will be send either to the \texttt{relayd} or to local storage. In
case of live tracing, our version of Babeltrace will communicate with the
\texttt{relayd}, get the CTF data, turn them into binary C-types and send them
to the Python consumer over a named Pipe. Then this consumer will transform them
into Scribe messages and send them to the Scribe server. In case of a non-live
tracing session, the CTF data will end up to the local disk. After the end of
the session, using our Babeltrace plugins, will transform them again into Scribe
messages and send them to Scribe.

\diagram{BlkKin Internal Communication}{blkin-internal.png}

One important thing to mention is that, every communication arrow in Figure
\ref{fig:blkin-internal.png} is a TCP connection. This, in conjunction with the
Scribe characteristics, gives us the ability for a variety of BlkKin deployments
within the instrumented cluster. For example we can have multiple relayds and a
single Scribe server. However, as it is going to be clarified in the Evaluation
Section \ref{}, the most appropriate and even suggested by Twitter deployment is
the one illustrated in Figure \ref{fig:blkin-deploy.png}.  In this deployment,
there is a whole BlkKin stack running on each cluster node, where the
instrumented application runs. This includes a local Scribe daemon as well. This
deployment enables us to take advantage of the Scribe batch messaging
capability. All the CTF messages are send to relayd and Babeltrace over
\texttt{localhost} which is faster and then end up to the local Scribe server.
Scribe in optimized to batch messages or temporarily store them locally if the
next Scribe server is unavailable. This lowers the network traffic and prevents
us from data loss. Also, it enables us from changing the final data destination
(Zipkin or HDFS) simply by changing the XML file from which each local Scribe
daemon gets configured. 

\diagram{BlkKin Deployment}{blkin-deploy.png}

\chapter{BlkKin Implementation}\label{ch:implementation}

In the previous chapter, we discussed how BlkKin was design to fulfil all the
needed prerequisites. In this chapter, we will present how we implemented the
BlkKin's interconnecting parts and the parts needed to use BlkKin and subtract
useful information. There also included code snippets that clarify the
implementation.

\section{Instrumentation Library}\label{sec:library}
As mentioned, we needed to implement a library in C/C++ that encapsulates the
tracing semantics mentioned in Dapper. This API should give programmers the
ability to perform any kind of tracing operation or correlation they want. Since
Zipkin was designed for distributed Web services, the existing, equivalent
Zipkin libraries for other languages, make use of HTTP headers in order to
transport the correlating information. In fact, there are three distinct HTTP
headers that travel along with the HTTP requests and used for tracing. These
headers are \texttt{X-B3-TraceId}, \texttt{X-B3-SpanId} and
\texttt{X-B3-ParentSpanId}. In our case, we have C/C++ applications
communicating, so instead of these HTTP headers, we created the equivalent
C-struct which includes the same information. This struct is the one in
Listing \ref{lst:blkin-info.h}.

\ccode{BlkKin basic struct}{blkin-info.h}
\ccode{BlkKin trace struct}{blkin-trace.h}

The above struct is exchanged between the different software layers and used for
tracing their correlations.

According to Zipkin, in order to create a trace, you also need an
\texttt{Endpoint} and a name for the trace. So, when an application receives or
creates a new \texttt{struct blkin\_trace\_info}, it also creates a
\texttt{blkin\_trace} as seen in Listing \ref{lst:blkin-trace.h} and then moves
on to the other library operations. 

After the above structs creation, the instrumented application could either
create annotations (Listing \ref{lst:blkin-record.h}) either timestamp or
key-value. Also, based on the instrumentation, the library enables the
programmer to create a \textit{child} span (Listing \ref{lst:blkin-child.h}),
based on the information of the current span, namely a span with the same trace
id, a different span id and a parent span id which is the same with the previous
span id. We have to mentioned that there is also a C++ wrapper for these
function calls, but is omitted since it shares exactly the same logic but
implemented in an object-oriented way.

\cccode{BlkKin child actions}{blkin-child.h}
\cccode{BlkKin annotations}{blkin-record.h}

All these mentioned structs include ids which are supposed to be unique not only
per computer node, but per cluster as well, since we plan to implement
distributed tracing. For our implementation, these ids are \texttt{uint\_64}
numbers that are randomly generated. In order to have a simple implementation,
we use \texttt{rand()}. However, the normal procedure to feed it with the
current timestamp failed for us, since we have multiple processes starting
almost simultaneously on the same host and this resulted in all these services
taking the same feed and producing the same ids. Instead, we feed
\texttt{srand()} by reading from \texttt{\textbackslash dev\textbackslash
urandom}, so each process gets at least different initial feed for the random
generator.

\section{LTTng tracing backend}

After having defined the above API, we had to provide an implementation which is
going to be based on LTTng. LTTng is activated only whenever the fucntion
\texttt{record} is called. So, someone could keep the rest of the library and
implement a custom tracing backend only by changing this function. Concerning
the LTTng case, the \texttt{record()} function makes LTTng \texttt{tracepoint()}
calls. These calls are predefined in a header file. This header file is used by
LTTng to create the methods bodies and the object file which is finally linked
to the final BlkKin object file, which in turn is linked with the instrumented
application. In Listing \ref{lst:zipkin_trace.h} you can see how we defined the
tracepoint for the key-value annotations. To avoid repetition, in the Listing
only the key-value tracepoint is illustrated. The timestamp tracepoint is the
same, but instead of key-value information, we have the event name. These
tracepoint function calls are defined in such a way that they include all the
necessary information for Dapper tracing, such as trace ids. In this
proof-of-concept version of BlkKin, all the tracepoints are considered WARNINGS,
and the severity level configured is such, so that all of them are eventually
logged.

\ccode{BlkKin tracepoints}{zipkin_trace.h}

LTTng assigns a timestamp to every event it records, so in case of the
timestamp events we do not need to care about timing information by calling
\texttt{gettimeofday()} for example. Instead, LTTng makes use of
\texttt{CLOCK\_MONOTONIC}, which is transformed to real timestamp during the
reading process in Babeltrace. As part of the CTF metadata, LTTng also sends the
timestamp and the monotonic value of the time when the session started, so that
the real timestamp can be formed during reading.

As far as sampling is concerned, in order not to trace all the requests, one has
to export an environmental variable called \texttt{RATE}. This variable is an
integer N which indicates that 1/N calls to \texttt{blkin\_init\_new\_trace}
should actually create a new trace. This way we can regulate the amount of
traces we produce.

So, the BlkKin library provides a header file to be included in the source code
and a dynamically linked object file to be linked with the application. This dll
includes all the necessary LTTng functions as well. However, as we described in
Section \ref{sec:user-tracing}, normally the LTTng threads are created whenever
the dll is loaded. This would cause problems for applications that fork(),
because the child would not have its own LTTng threads to trace. So, instead, we
used \texttt{dlopen} \texttt{dlsym} and manually load the BlkKin functions which
in turn load the LTTng object file and create the needed threads whenever the
function \texttt{blkin\_init()} is called.

To sum up, we cite an execution example in Listing \ref{lst:blkin-example.c} and
its Makefile in Listing \ref{lst:Makefile}.

\ccode{BlkKin example Makefile}{Makefile}
\ccode{BlkKin execution example}{blkin-example.c}

\section{Babeltrace plugins implementation}

In this section we will describe how we implemented the Babeltrace plugins that
convert CTF data to Scribe messages and send them to the Scribe server. As
mentioned, we implemented two different plugins one generic that sends JSON data
to Scribe and one Zipkin-specific that creates Scribe messages that end up to
Zipkin. Since Scribe messages are simple strings after all, both plugins share a
common core that handles with the Scribe connection and message transportation.
Each plugin implements a different message formation part which results to a
string message to be sent to Scribe.

First we will explain how we extract the tracing information. Babeltrace exposes
a Python library which is created using swig2.0\footnote{http://www.swig.org/}.
According to this library, in order to read the CTF data you have to create a
\texttt{Tracecollection} object and call the method \texttt{add\_trace} on it
passing the trace path. After that, a Python generator is created in
\texttt{Tracecollection.events}. If we iterate over that generator, we can take
the CTF event trace information formated as a Python object which has properties
such as the event name and the timestamp as well as another generator that
returns the event information in the form of a tuple (item-name, item-value),
namely the values passed to the \texttt{tracepoint} function call. After that we
can manipulate the data the way we want.

The case of the JSON format is easy since after that we can easily create a
Python dictionary and transform it into a JSON object which is ready to be
forwarded to Scribe as it is already a string. On the other hand, in order to
feed data to Zipkin the procedure is different. As mentioned in Section
\ref{sec:zipkin}, Zipkin makes use of Thrift as well in order to create a binary
representation of the events and the Thrift file used can be found in Listing
\ref{lst:zipkin.thrift}. So, in order to create these messages, we used Thrift
to create the Python code from the Zipkin Thrift file. Thrift created the Python
classes needed. So, when we obtained the tracing information with the method
mentioned above, we passed them to the class constructors and created the
equivalent \texttt{Span}, \texttt{Annotation} and \texttt{Endpoint} objects.
After that each span is encoded using the Thrift \texttt{TBinaryProtocol} and
this value, in order to become a string, is base64 encoded. After this
procedure, we have the final string to forward to Scribe. However, we should
mention that the use of the BlkKin plugin is possible only when the BlkKin
library was been used for application instrumentation and is not a
general-purpose plugin.

After having formatted the messages, we have to send them to Scribe. To do that
we used the Python module \texttt{facebook-scribe} from \texttt{pip}. Sending a
message to Scribe is as simple as it seen in Listing \ref{lst:scribe.py}.

\pcode{Scribe messaging}{scribe.py}

In case we want to annotate on a subsystem that does not have distinct beginning
and end, Zipkin provides some special annotations that are used to start and end
a span. They are used only for instrumentation and visualization purposes.
However, by using these annotations we can be sure that a span has
ended and then forward the packed message to Scribe, namely a message that
includes multiple annotations. This reduces the number of messages and
consequently the network traffic and the server load. Depending on the
instrumentation this may not be possible when having a span that collects
annotations from multiple computer nodes in cases of distributed traces. So, the
Zipkin plugin has the option to collect annotations, temporarily store them in
a dictionary using as a key the trace and span id pair and when a predefined
annotation is asked to be logged then create a message including all the
annotations for this specific span. In case this is not possible, the plugin
just creates single-annotation messages and forwards them to Scribe.

\section{BlkKin Live tracing}\label{sec:bbl-live}
Although, as already mentioned, the solution we gave for live tracing is only
termporary, in this section we explain how we implemented live tracing in BlkKin
for reasons of completeness. This solution probably will be abandoned when
Babeltrace refactoring is over. Then, the above plugins will operate with the
live tracing as well.

As mentioned, Babeltrace offers only CTF-to-text transformations for live
tracing. Although we could redirect Babeltrace text output to a process that
would create the Scribe messages, we chose to avoid this CTF-to-text and
text-to-Scribe conversion. Instead, we created a version of Babeltrace
especially for BlkKin. This version instead of reading generic CTF-events is
aware of the events' content as described in \ref{lst:zipkin_trace.h}.
Consequently, instead of iterating over the CTF trace for generic events, this
version of Babeltrace extracts the included information and creates a C-struct,
as seen in Listing \ref{lst:babeltrace-live.h}, including all the necessary
information to create a Zipkin message.

\ccode{Babeltrace live struct}{babeltrace-live.h}

After that this struct is forwarded to a Unix pipe. On the other side of the
pipe there is a Python process. This process makes use of Python
ctypes\footnote{https://docs.python.org/2/library/ctypes.html}. So whenever a
struct is read, a Python object is created and from that point, the Python
consumer reuses the Zipkin Babeltrace plugin code to create the equivalent Span,
Annotation and Endpoint objects and finally send the base64-encoded message to
Scribe.

\section{BlkKin Endpoints}\label{sec:blkin-hadoop}

\subsection{BlkKin monitoring tool}
As we mentioned in the design part, Zipkin covered a basic need for the user
interface with its Web UI. However, this specific UI can be used mostly for
trace visualization and not as a tool that can be used to detect abnormalities,
fault or as an alerting mechanism.  Thus, we created a simple Python-django
application. In this proof-of-concept implementation, this application
communicates with the Zipkin databse which is a MySQL database. We chose MySQL
so that we can make any kind of ad-hoc queries, avoiding schema-less options
like Cassandra. The application queries the database for aggregate and average
information that depict the cluster state over the last N minutes. After that
either using a threshold for these values or judging on other criteria, it
illustrates this information as green when everything is working properly, or
red when there is a possible anomaly. There screenshots of this tool in the
evaluation part (Section \ref{sec:failures}).

\subsection{Hadoop}
Scribe can be configured to store the data it receives in HDFS. So, we used
Hadoop 0.21.0 to store tracing data from BlkKin for further analysis. The
usecase scenario includes extensive tracing (not real time) without sampling and
post-processing the tracing data in order to locate the thresholds and metrics
used in the BlkKin monitoring tool. After that BlkKin can use sampling and send
data to Zipkin.

Based on the plugin used, different kind of data ends up in HDFS. If we use the
JSON Babeltrace plugin we end up having JSON data in the HDFS which can be
manipulated easily with Map-Reduce jobs. However, if we use the Zipkin
Babeltrace plugin, then in the HDFS we have multiple files containing base64
encoded strings. In order to extract the information wanted through Map-Reduce,
we had to decode this information following the opposite direction. Again using
Thrift and the Zipkin Thrift file we created the equivalent Java code which
created the Annotation, Span and Endpoint classes in Java. Then after reading
from HDFS, we base64 decoded the strings and then TBinaryProtocol decoded them.
After that we could create the Span object. Since most of the data we wanted to
extract where the duration between two specific annotations, we created a simple
Java interface to help us. The interface can be seen in Listing \ref{lst:interface.java}
and its up to the user how to implement it.
This interface can be implemented either by the
Mapper or by a class used by the Mapper used to pick the right annotations.
Whenever a Span with including annotations is read we have to define whether to
keep the span or not. The same decision has to be made for the included
annotations in case we keep the span. The timestamp of the kept annotations will
be then forwarded to reducers. This will happen in the form of a key-value pair
(id, timestamp). The reducer's job is just to subtract the two different values
belonging to the same id. This id is a string and its creation is also defined
by the interface. For our case we needed to create Map-Reduce jobs that compute
average durations and it was really easy to do so using the interface and the
Thrift-produced Java code.

\ccode{Hadoop interface}{interface.java}

\chapter{Evaluation}\label{ch:evaluation}

In this section we will describe our experience from using BlkKin in a real
usecase senario. The instrumented infrastructure is described in Section
\ref{sec:infra}. After that, in Section \ref{sec:metrics} we analyse performance
metrics concerning the network and system overhead that justify our design and
deployment choices. Finally, in Section \ref{sec:failures} we explain how we
used BlkKin to identify system faults which are virtually inserted by us, but
reflect possible real failures or bottlenecks. 


\section{Instrumented infrastructure}\label{sec:infra} As a use-case, we used
BlkKin to instrument Archipelago and RADOS. These systems were examined in
Sections \ref{sec:archip-bkg} and \ref{sec:rados} respectively. Archipelago is
written is C and RADOS in C++. So we used the BlkKin library C++ wrapper for
the RADOS instrumentation. Instead of using the \texttt{archip-bench} tool,
which is part of Archipelago, to initiate IO requests, we instrumented the Qemu
Archipelago driver. So, using Qemu we start a virtual machine which has an
Archipelago drive and create different IO loads to this driver using
\texttt{fio}\footnote{http://linux.die.net/man/1/fio}. Thus we can track the IO
request from the time Qemu receives it until it is finally served by RADOS. 

The Qemu Archipelago driver receives the IO requests from Qemu and creates XSEG
requests for the VLMC. Qemu initiates the tracing information as well and
Qemu spans are the root spans. After that, these tracing information is
carried as part of the XSEG request. To do that, we needed to extend
\texttt{libxseg}\footnote{https://github.com/grnet/libxseg} and add the
tracing information needed as shown in Listing \ref{lst:blkin-info.h} nested in
the XSEG request. So, as far as Archipelago in concerned, the tracing
information is transmitted as part of the XSEG request. Each Archipelago peer
is considered a different service, with a different endpoint that creates a
single span per IO request in the general case. So, in the Zipkin UI we expect
to see each peer represented as a single bar, whose length indicates the time
this peer needed to serve this specific request.

Unlike Archipelago where the instrumentation was obvious, instrumenting RADOS
was more challenging. RADOS exposes a C-API (librados) which is used in the
Archipelago rados-peer. So, the first thing we did was to instrument the read
and write calls of this API. Then, we needed to extend the RADOS classes to
transfer the tracing information. In a nutshell, after librados, Ceph protocol
which is TCP-based transfers the IO request to the cluster. So, the tracing
information is encoded as part of the \texttt{MOSDOp} Ceph object. Then the
request after being decoded, enters a dispatch queue and waits to be served.
Based on the objects affected, a different placement group handles it. After
the dispatch queue, the request is handled by this pg's primary OSD and then
based on the replication factor, equal number of replication operations are
issued that follow the same route. Request handling includes journal access and
filestore access. In an attempt not to expose much of the RADOS internals, so
that the Zipkin UI would be self-explanatory even for someone that is not
familiar with the RADOS code architecture, we tried to instrument the code so
that we can extract information such as the time spent in the dispatch queue,
the network communication time, or the journaling duration and at the same time
we follow the causal relations used by Zipkin. For example, the IO handling by
the primary OSD causes the replication operations. So the replication
operations are children spans.

As far as the test-bed is concerned, we used two physical nodes LAN
interconnected, and set up two OSDs on each node. On one of this nodes we
installed Archipelago and Qemu. So, on the one node we had the running VM,
Archipelago and 2 OSDs and on the other just two OSDs. Each node had a whole
BlkKin stack running and a local Scribe server. Each Scribe server communicated
with the central Zipkin collector or the Scribe server logging to the Hadoop
cluster. For Zipkin we used a 4-core. 8-gb RAM virtual machine, while for the
Hadoop cluster, as it was used only as a proof of concept, we used 2 2-core,
4-gb virtual machines. 

You can find some specs regarding the hardware and software infrastructure in
Tables
\ref{tab:hardware-specs} and \ref{tab:software-specs}. 

\begin{table}[H]
    \centering
    \begin{tabular}{ | l | l | }
        \hline
        Component & Description \\ \hline \hline
        CPU &  2 x Intel(R) Xeon(R) CPU E5645 @ 2.40GHz \cite{e5645} \\
         & Each CPU has six cores with Hyper-Threading enabled, which equals to 
         24 threads. \\ \hline
        RAΜ & 2 banks x 6 DIMMs PC3-10600 \\
        & Peak transfer rate: 10660 MB/s \\ \hline
        Hard disks & 12x 7.2k RPM 2TB SAS HDs, 12x 7.2k RPM 600GB SAS HDs, 6x \\
        & 100GB SSD SATA HDs \\ \hline
    \end{tabular}
    \caption{Test-bed hardware specs}
    \label{tab:hardware-specs}
\end{table}

The Ceph OSDs on the one node used two SSD disks in RAID 0, one each, and the
other two on the other node two SAS disks in RAID 0, one each.

\begin{table}[H]
    \centering
    \begin{tabular}{ | l | l | }
        \hline
        Software & Version \\ \hline \hline
        OS &  Debian Wheezy \\ \hline
        Linux kernel & 3.2.0-4 \\ \hline
        lttng-tools & 2.4 \\ \hline
    \end{tabular}
    \caption{Test-bed software specs}
    \label{tab:software-specs}
\end{table}

\section{BlkKin tracing environment}
In order to set up the tracing environment, apart from the central Scribe
collector which could be either the Zipkin collector or a Scribe server
connected to HDFS, on each cluster node you have to follow the next steps to
live trace:

\begin{enumerate}
\item Start the local Scribe daemon \\ 
    The local Scribe daemon is configured to send the received messages to the
    next Scribe server. In case of connection loss or congestion, the data are
    stored locally and forward when the problem is solved.

\item Start LTTng live tracing \\
    Create a live tracing session and enable all the userspace events for it.
    Then start Babeltrace live and redirect its output to a named pipe.

\item Start the consumer \\
    Start the python consumer that reads from the named pipe and send data to
    the local Scribe server.  
\end{enumerate}

The overall procedure can be seen in Listing \ref{lst:start.sh}
\bcode{Setting up the tracing environment}{start.sh}
\section{Evaluation metrics}\label{sec:metrics}

In this section we analyze some metrics concerning the network and the system
overhead that BlkKin poses to the system. These metrics led us to the previous
deployment architecture.

\subsection{LTTng vs Syslog}
First of all, we should evaluate the use of LTTng versus another logging system
that is based on system calls every time some information needs to be logged.
SystemTap, DTrace or even syslog make a system call or a stop at a break point
whenever they need to trace something. This is claimed to have high overhead for
a system that need to run in full load. As mentioned the BlkKin library was
backend independent, so we we changed the LTTng backend for a syslog backend and
instrumented a simple client server application with the BlkKin library. Then we
measured the time eacg backend took to finish tracing. The application was a
simple two-process application, that communicated over a UNIX pipe. The server
created a root span, annotated and then passed a message to the client including
tracing information. The client created a child span annotated and answered
back. A single iteration triggers four annotations and we repeated this message
passing for 10000 times. The results can be seen in Table
\ref{tab:lttng-syslog}. 

\begin{table}[H]
    \centering
    \begin{tabular}{ | l | l | }
        \hline
        Backend & Time for 10000 iterations \\ \hline \hline
        LTTng &  1.8 sec \\ \hline
        Syslog & 3.38 sec \\ \hline
    \end{tabular}
    \caption{LTTng vs Syslog comparison}
    \label{tab:lttng-syslog}
\end{table}

As it was expected the syslog backend was about 90\% slower than LTTng even for
this single example. So LTTng was the only choice possible that combined the
low-overhead capability with the variety of tracing options such as user/kernel
space tracing and performance counter access.

\subsection{Thrift vs JSON format}
As we mentioned we created two similar Babeltrace plugins, one sending JSON
formatted messages and the other Thrift encoded messages to Scribe. It worths
measuring the message size in these two cases because the smaller the message
the lower the network overhead.
In order to evaluate this parameter we created a simple BlkKin message and send
it to a Scribe server both with the JSON and the Zipkin plugin. The message was
as simple as seen in Listing \ref{lst:message.json}

\ccode{A simple JSON formatted message}{message.json}

The packet size of the Zipkin-Thrift-encoded and the JSON encoded messages sent
to Scribe can be seen in Table \ref{tab:payloads}

\begin{table}[H]
    \centering
    \begin{tabular}{ | l | l | }
        \hline
        Protocol & Packet size in bytes \\ \hline \hline
        Thrift & 246 \\ \hline
        JSON  & 316 \\ \hline
    \end{tabular}
    \caption{Packet sizes per protocol used}
    \label{tab:payloads}
\end{table}

As it was expected, the Thrift message is much smaller that the JSON one even in
this case that the service and event names are small.

\subsection{Scribe vs relayd}
Another comparison we need to make to decide on the deployment architecture is
the network overhead created by Scribe and relayd. For example, instead of
running a local Scribe server, we could run a central realyd server per cluster
and then send the data to the central Scribe server.  Scribe, as mentioned
offers buffering and batch messaging. Also, the LTTng consumerd will be faster
when writing to localhost rather than to a remote server, thus reducing the
possibility to lose tracing information. However, we have to figure out the
amount of network traffic produced in the two cases. To evaluate this, we
created 10 simple messages as the previous ones and sent them to a local relayd
and then they were forward using Babeltrace live to the Scribe server. We
measured the network traffic to localhost and to the Scribe server.

Our first notice when using \texttt{tcpdump} on localhost is that relayd polls
consumerd on a specific interval to find out if there are any new data
available. So, we wouldn't like to have our cluster being flooded by polling
messages. Concerning the tracing data themselves, excluding polling, the sums of
all the TCP packets' payloads sent for the 10 messages mentioned, can be found
in Table \ref{tab:relayd-scribe} for each daemon.

\begin{table}[H]
    \centering
    \begin{tabular}{ | l | l | }
        \hline
        Daemon & Data size in bytes \\ \hline \hline
        Scribe & 1974 \\ \hline
        relayd  & 1079 \\ \hline
    \end{tabular}
    \caption{Data sent for 10 Scribe messages}
    \label{tab:relayd-scribe}
\end{table}

However, even if CTF-format is more compact that Thrift, we prefer to avoid the
polling messages in the cluster LAN and restrict them to localhost and to make
use of the Scribe batch messaging capability in favor of the less payload size
that CTF has to offer.

\subsection{System overhead}
In this subsection we evaluate the overhead that BlkKin poses to the
instrumented application. To do so, we created two different but typical IO
loads using fio. The first one was 2GB of random 4Kb writes and the second was
2GB of 64Kb sequential writes. These loads are diverse but really common and
help up evaluate our systems performance under different working conditions.
So, we ran fio in the virtual machine towards the Archipelago volume and
measured the bandwidth the system had under different conditions. The scenarios
we chose was without instrumentation, live tracing without sampling, live
tracing with 1/500 sampling and tracing without live support. The results can
be seen in Listings \ref{fig:4.png} and \ref{fig:64.png} and Tables
\ref{tab:4k-random} and \ref{tab:64k-sequential} respectively.

\diagramscale{Performance overhead for 4k random writes}{4.png}{0.82}
\diagramscale{Performance overhead for 64k sequential writes}{64.png}{0.82}

\begin{table}[H]
    \centering
    \begin{tabular}{ | l | l | }
        \hline
        Scenario  & Bandwidth in Kbytes/sec  \\ \hline \hline
        no tracing & 16887  \\ \hline
        stopped tracing & 16076 \\ \hline
        normal tracing & 14882  \\ \hline
        live tracing & 14941  \\ \hline
        live tracing sampling 500 & 15480  \\ \hline
    \end{tabular}
    \caption{Bandwidth for 64Kb sequential writes}
    \label{tab:64k-sequential}
\end{table}

\begin{table}[H]
    \centering
    \begin{tabular}{ | l | l | }
        \hline
        Scenario  & Bandwidth in Kbytes/sec  \\ \hline \hline
        no tracing & 1326.7  \\ \hline
        stopped tracing & 1247.1 \\ \hline
        normal tracing & 1100 \\ \hline
        live tracing & 1107.8  \\ \hline
        live tracing sampling 500 & 11927  \\ \hline
    \end{tabular}
    \caption{Bandwidth for 4Kb random writes}
    \label{tab:4k-random}
\end{table}

Although the above figures depict different loads and working conditions, BlkKin
performs in a similar way. We can see that when tracing is disabled, we have
about 5\% bandwidth degradation. This degradation is caused by the check whether
LTTng should trace or not. In case of full tracing the performance degradation
increases at about 12\% in case of sequential writes and at about 17\% in case
of small, random writes. This degradation is caused by the tracepoint function
calls that actually log the information. The case of 4Kb writes is more affected
because we have more IO request taking place fast one after the other. So the
LTTng load is greater. Both live tracing without sampling and normal tracing
affect the system in a similar way. The only change as far as LTTng is concerned
is that in case if live tracing the consumerd writes the tracing information to
a TCP socket instead of a local file descriptor. Finally, as we see in the
scenario of sampled tracing, the degradation is such that we can afford tracing
our system in production scale. In our scenario, we chose to sample 1/500 IO
requests. Depending on the system load, this sampling rate can be either reduced
or increased. It should be mentioned that in our instrumentation we used 113
annotations per single IO request in order to track its whole route. So, the
case of so-sampling, live tracing produced a significant amount of network
traffic and should be avoided not only because of this traffic, but also because
of it, the tracing information take more time to reach to Zipkin. Consequently,
we will need more time to detect a possible failure or problem.

\section{Using BlkKin to detect bottlenecks and failures}\label{sec:failures}

As it has become obvious, BlkKin can be used for various reasons and trying to
detect different problems either as part of the debugging process or as part of
a fault detection mechanism. In this subjection we evaluate both of these uses.

\subsection{Using Zipkin in debugging and system evaluation}
The most simple use of BlkKin is to analyze the system's performance, measure
the communication latencies, possible computation bottlenecks and generally to
get a general overview of the request's evolution.  For these reasons, the
Zipkin UI is really usefull. This UI enables us to do simple queries and to
understand the causal relations, to evaluate the time differences between the
different software layers and to access the key-value annotations or even make
queries based on them.


As seen in Figure \ref{fig:zipkin-overview.png} each span is represented as a
separate bar whose length is commensurate to the duration of this specific
processing phase. As far as time is concerned, each request is considered to
start at time 0 and all timestamp annotations have timestamps relative to the
first annotation of the trace. Also, if you click on a span that you are
interested about, you can access this span's annotations as can be seen in
Figure \ref{fig:zipkin-annotations.png}. In this screenshot we chose to
investigate an `OSD Handling op', namely the span that annotates the OSD's
actions to serve and IO request. At the top we can see each timestamp annotation
with its relative time, while at the bottom we can see the key-value
annotations. For example, this operation handling refers to the specific RADOS
object whose name can be seen as part of the binary annotation.

\diagram{Zipkin UI overview}{zipkin-overview.png}
\diagram{Zipkin UI Annotations view}{zipkin-annotations.png}

During this thesis evolution, Zipkin changed its UI. The old UI offered another
visualizing capability that is planned to be added to the new UI soon. This
capability was about endpoint dependencies. As it is seen in Figire
\ref{fig:zipkin-depend.png}, each service is depicted as a different circle
whose radius is commensurate to the processing duration of that specific phase.
This d3 visualization depicts exactly the information flow.

\diagram{Zipkin UI Dependencies View}{zipkin-depend.png}

In case that these visual representations are not enough and we want to extract
aggregate values, such as average values we have other choices. As mentioned
Twitter suggest to deploy Zipkin with Cassandra. However, we deployed Zipkin
using MySQL in order to have the ability of ad-hoc queries. So, depending on the
amount of tracing data we can extract the either from the database or by using
Hadoop. In our case, we used Hadoop to calculate the average journaling time per
OSD. So, simply by changing each local Scribe server's configuration file, we
chose to send data to a Scribe server connected to HDFS and not Zipkin. Then, we
created a Map-Reduce job to extract the information wanted as described in
Section \ref{sec:blkin-hadoop}. The information gathered can be seen in Table
\ref{tab:hadoop-journal}.

\begin{table}[H]
    \centering
    \begin{tabular}{ | l | l | }
        \hline
        OSD  & Journaling time ($\mu$sec)  \\ \hline \hline
        OSD1 & 401  \\ \hline
        OSD2 & 494  \\ \hline
        OSD3 & 475  \\ \hline
        OSD4 & 475  \\ \hline
    \end{tabular}
    \caption{Average journaling time for the 4 OSDs}
    \label{tab:hadoop-journal}
\end{table}

\subsection{Using Zipkin to detect abnormal behaviors}

The other BlkKin's use is as an online anomaly detection mechanism. According to
\cite{china-detector}, numerous techniques have been proposed for detecting
system anomalies. Among them, the simplest ones are the threshold-based
techniques which are a form of service level agreements (SLAs). They are very
useful on the condition that their users clearly know the key metric to monitor
and the best value of the thresholds in different scenarios. Unfortunately, it
is very difficult, even for an expert, to correctly choose the necessary metrics
to monitor and set the right values of the thresholds for different scenarios in
the context of today's complex and dynamic computer systems. In addition,
statistical learning or data mining techniques are widely employed to construct
probability models for detecting various anomalies in large-scale systems based
on some heuristics and assumptions, although these heuristics and assumptions
may only hold in some particular systems or scenarios. Other methods
(\cite{syslog-svm}), include artificial intelligence and neural networks, but
they require large training datasets.

In this part of the evaluation, we present how we used Zipkin to detect common
cases of anomaly that are possible to happen in a storage cluster. This
abnormalities refer to either network or disk problem. We tried a threshold
based alerting approach as a proof of concept, while the expressiveness and
correlation capabilities of BlkKin allow further investigation for more
correlative detection models. In our model, we conducted several tracing
sessions for different IO loads and filled a Hadoop cluster with these data.
Afterwards, through Map-Reduce jobs, we concluded on the thresholds we are going
to use for the specific hardware and depending on the expected load. The
thresholds included the communication times through the RADOS protocol and the
times for journaling and filestore access. Then, we inserted these threshold
values to the BlkKin monitoring tool. 

Apart from BlkKin, we needed to find a way to simulate a faulty situation. For a
network fault this was easy using the \texttt{tc} tool. This tool enables the
user to insert common network faults, such as packet loss, latency or packet
corruption.  However, the case of disk faults is more complex. To simulate a
faulty disk state there are multiple options. The most easy and the finally
chosen is to add a significant IO load using fio with multiple threads making
dummy read request to the specific disk. In our case of RADOS, since we
constantly write 4Mb objects, this will increase the disk latency. The second
choice is to use \texttt{cgroups} and the \texttt{blockio} controller. This
choice enables us to throttle a disk bandwidth, by throttling the bandwidth of
the processes writing to this disk, namely the OSD process. Another choice is to
use the device mapper to create a faulty sector. The final choice is to use the
Linux kernel fault injection
capabilities\footnote{https://www.kernel.org/doc/Documentation/fault-injection/fault-injection.txt}.
The complexity of the latter choices and the kernel dependencies made us avoid
them and just a dummy read IO load to the disk that we need to act as in case of
a fault.

So, to simulate a disk fault we added a read IO load in the OSD4 journal
partition. The results in Zipkin can be seen in Figure \ref{fig:disk-fault.png}.
As we can see, the Journaling span of the OSD4 takes significantly more time,
when compared to a normal behaviour as seen in Figure
\ref{fig:zipkin-overview.png} and as a result the whole request completion is
affected since journaling is synchronous and encapsulated within the request
process. 

\diagram{Injected disk fault - Journaling Latency}{disk-fault.png}

Of course, this journaling duration is above the accepted threshold. So, the
BlkKin monitoring UI would illustrate the problem as seen in Figure
\ref{fig:disk-fault-blkin.png}.

\diagram{Journaling Latency - BlkKin monitoring UI}{disk-fault-blkin.png}

A similar situation was simulated for a network problem. Using \texttt{tc} we
added 10 ms of latency to the NIC attached to the host where OSD1 and OSD3 run.
The result are extended communication times that can be seen as part of the
\texttt{Main} spans in Figure \ref{fig:network-error.png}. Again this faulty
situation is observed as seen in Figure \ref{fig:network-error-blkin.png}
through the BlkKin monitoring tool.

\diagram{Network Latency - Zipkin  UI}{network-error.png} 
\diagram{Network Latency - BlkKin monitoring UI}{network-error-blkin.png}

\chapter{Conclusion}\label{ch:conclusion}

\section{Concluding Remarks}

This thesis handles with the problem of low overhead distributed tracing aimed
to analyze software defined storage systems. In this kind of multi-layered
software architectures finding and locating bottlenecks and potential or even
real faults is pretty challenging because of their complexity. After evaluating
both various logging and tracing mechanism as well as different tracing
schemas, we decided to implement our own tracing infrastructure called BlkKin.
BlkKin is based on opensource technologies, specifically LTTng and Zipkin, and
implements the tracing semantics used by Google's Dapper tracing
infrastructure. So BlkKin is a low-overhead tracing infrastructure that enables
live tracing for applications written in C/C++ and provides two distinct user
interfaces, so that the end user can take real time information about the
system.

In order to accomplish this endeavour, we had to work with the LTTng community
and extend their software so that it can communicate with Zipkin. We also,
created an instrumentation library for easy application instrumentation.

As a proof of concept for our system, we instrumented Archipelago and RADOS
source code. Consequently, we were able to track an IO request's route from the
time Qemu accepted it until it was finally served by RADOS and investigate any
kind of latencies or bottlenecks each processing phase may have. Also, we
simulated different faulty situations that a distributed storage system may face
and investigated the use of BlkKin as an alerting mechanism for such kind of
faults.

However, this work was just the beginning of cross-layered distributed tracing,
since it provided the framework for further investigation. Our mechanism can be
used in any kind of low-overhead application that needs a tracing and
visualization infrastructure. So, far the Ceph community has showed interest
for BlkKin and we are in close contact so that BlkKin can become the main
tracing infrastructure for RADOS. 
 
\section{Future Work}
BlkKin future plans/work include:
\begin{itemize}
\item Better live support, after the Babeltrace plugin system is released.
\item Better sampling mechanism that takes into account the request's special
characteristics, so that meaningful information is not lost because of sampling
\item Offer Babeltrace plugins as part of the LTTng source tree in the form of a
Python module availabe at \texttt{pip}.
\item Use BlkKin in different kinds of distributed systems such as parallel
applications, MPI for example.
\end{itemize}

As far as the RADOS instrumentation is concerned, future work includes:
\begin{itemize}
\item Better RADOS instrumentation not only for read and write requests. This
instrumentation requires deep knowledge of the software's internals so that its
bottlenecks are found.
\item Implementation of a correlative alerting system that relates the
replica operations with the cluster's health in order to avoid our threshold
based alerting mechanism.
\item Use of BlkKin tracing data from RADOS instrumentation to create an
AI-based failure detector.
\end{itemize}

\backmatter
\cleardoublepage % start at the next odd page
\phantomsection % correct hyperlinking
\bibliography{references}
\bibliographystyle{plain}
% \include{glossary}
% \printindex

\end{document}
