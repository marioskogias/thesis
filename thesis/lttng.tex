\chapter{Linux Trace Toolkit - next generation (LTTng)}\label{ch:lttng}

In this chapter we analyze Linux Trace Toolkin - next generation (LTTng), which
was our choice for BlkKin's tracing backend, and we describe its internal
characteristics that led us to using it. Specifically, we give an overall
outline of its architecture and basic notions in Section
\ref{sec:lttng-overview}. Then, we describe the buffering scheme used both for
kernel and user space (Section \ref{sec:buffers}) and we continue by citing
kernel and use space implementation mechanism in Sections
\ref{sec:kernel-tracing}, \ref{sec:user-tracing}. Finally we cite the tracing
format used by LTTng (Section \ref{sec:ctf}) and the mechanism for live tracing
in Section \label{sec:relayd}.

\section{Overview}\label{sec:lttng-overview}

Linux Trace Toolkin - next generation is the successor of Linux Trace Toolkit.
It started as the Mathew Desnoyer's PhD dissertation \cite{desnoyer} in École
Polytechnique de Montréal. Since then, it is maintained by EfficiOS
Inc\footnote{http://www.efficios.com/} and the DORSAL lab in  École
Polytechnique de Montréal.

The LTTng project aims at providing highly efficient tracing tools for Linux.
Its tracers help tracking down performance issues and debugging problems
involving multiple concurrent processes and threads. Tracing across multiple
systems is also possible. This toolchain allows integrated kernel and user-space
tracing from a single user interface. It was initially designed and implemented
to reproduce, under tracing, problems occurring in normal conditions. It uses a
linearly scalable and wait-free RCU (Read-Copy Update) synchronization mechanism
and provides zero-copy data extraction. These mechanisms were implemented in
kernel and then ported to user-space as well.
 
Apart from LTTng's kernel tracer and userspace tracer, viewing and analysis
tools are part of the project. In this thesis, we worked with and extended 
\textit{Babeltrace} \footnote{http://lttng.org/babeltrace}.

Apart from the fact LTTng is a complete toolchan that can be easily installed in
almost any Linux distribution and the integrated kernel and user space tracing
offered, we chose LTTng because of its minimal performance overhead. Since it
was initially designed to `reproduce, under tracing, problems occurring in
normal conditions', LTTng was the ideal tool to use for real-time low-overhead,
block-storage tracing with BlkKin.

In order to understand how LTTng manages to have such a good performance, we
have to go through its internals. But first, we give an overview outline of its
architecture and basic components. According to D. Goulet's Master thesis
(\cite{goulet}), LTTng's architecture can be summarized as shown in Figure
\ref{fig:lttng-arch.png}.  

\diagram{LTTng Architecture}{lttng-arch.png}

The \texttt{lttng} command line interface is a small program used to interact
with the session daemon. Possible interaction are creating sessions, enabling
events, starting tracing and so on. The use of this command line tool is further
explained in Chapter \ref{} about how to use BlkKin.

Tracing sessions are used to isolate users from each other and create coherent
tracing data between all tracing sources (Ex: MariaDB vs Kernel). This
\textit{session daemon} routes user commands to the tracers and keeps an
internal state of the requested actions. The daemon makes sure that this
internal state is in complete synchronization with the tracers, and therefore no
direct communication with the tracers is allowed other than via the session
daemon.  This daemon is self-contained between users. Each user can run its own
session daemon but only one is allowed per user. No communication happens
between daemons. 

\textit{Consumer daemons} extract data from buffers containing recorded data and
write it to disk for later analysis. There are two separate consumer daemons,
one handling user space and the second one the kernel. A single consumer daemon
handles all the user space (and similarly for kernel space) tracing sessions for
a given session daemon. It is the session daemon that initiates the execution of
the user space and kernel consumer daemons and feeds them with tracing commands.

\section{Buffering scheme}\label{sec:buffers}

\section{Kernelspace tracing}\label{sec:kernel-tracing}

\section{Userspace tracing}\label{sec:user-tracing}

\section{Common Trace Format (CTF)}\label{sec:ctf}

\section{Live tracing}\label{sec:relayd}
