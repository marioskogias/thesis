\chapter{Evaluation}\label{ch:evaluation}

In this section we will describe our experience from using BlkKin in a real
usecase senario. The instrumented infrastructure is described in Section
\ref{sec:infra}. After that, in Section \ref{sec:metrics} we analyse performance
metrics concerning the network and system overhead that justify our design and
deployment choices. Finally, in Section \ref{sec:failures} we explain how we
used BlkKin to identify system faults which are virtually inserted by us, but
reflect possible real failures or bottlenecks. 


\section{Instrumented infrastructure}\label{sec:infra} As a use-case, we used
BlkKin to instrument Archipelago and RADOS. These systems were examined in
Sections \ref{sec:archip-bkg} and \ref{sec:rados} respectively. Archipelago is
written is C and RADOS in C++. So we used the BlkKin library C++ wrapper for
the RADOS instrumentation. Instead of using the \texttt{archip-bench} tool,
which is part of Archipelago, to initiate IO requests, we instrumented the Qemu
Archipelago driver. So, using Qemu we start a virtual machine which has an
Archipelago drive and create different IO loads to this driver using
\texttt{fio}\footnote{http://linux.die.net/man/1/fio}. Thus we can track the IO
request from the time Qemu receives it until it is finally served by RADOS. 

The Qemu Archipelago driver receives the IO requests from Qemu and creates XSEG
requests for the VLMC. Qemu initiates the tracing information as well and
Qemu spans are the root spans. After that, these tracing information is
carried as part of the XSEG request. To do that, we needed to extend
\texttt{libxseg}\footnote{https://github.com/grnet/libxseg} and add the
tracing information needed as shown in Listing \ref{lst:blkin-info.h} nested in
the XSEG request. So, as far as Archipelago in concerned, the tracing
information is transmitted as part of the XSEG request. Each Archipelago peer
is considered a different service, with a different endpoint that creates a
single span per IO request in the general case. So, in the Zipkin UI we expect
to see each peer represented as a single bar, whose length indicates the time
this peer needed to serve this specific request.

Unlike Archipelago where the instrumentation was obvious, instrumenting RADOS
was more challenging. RADOS exposes a C-API (librados) which is used in the
Archipelago rados-peer. So, the first thing we did was to instrument the read
and write calls of this API. Then, we needed to extend the RADOS classes to
transfer the tracing information. In a nutshell, after librados, Ceph protocol
which is TCP-based transfers the IO request to the cluster. So, the tracing
information is encoded as part of the \texttt{MOSDOp} Ceph object. Then the
request after being decoded, enters a dispatch queue and waits to be served.
Based on the objects affected, a different placement group handles it. After
the dispatch queue, the request is handled by this pg's primary OSD and then
based on the replication factor, equal number of replication operations are
issued that follow the same route. Request handling includes journal access and
filestore access. In an attempt not to expose much of the RADOS internals, so
that the Zipkin UI would be self-explanatory even for someone that is not
familiar with the RADOS code architecture, we tried to instrument the code so
that we can extract information such as the time spent in the dispatch queue,
the network communication time, or the journaling duration and at the same time
we follow the causal relations used by Zipkin. For example, the IO handling by
the primary OSD causes the replication operations. So the replication
operations are children spans.

As far as the test-bed is concerned, we used two physical nodes LAN
interconnected, and set up two OSDs on each node. On one of this nodes we
installed Archipelago and Qemu. So, on the one node we had the running VM,
Archipelago and 2 OSDs and on the other just two OSDs. Each node had a whole
BlkKin stack running and a local Scribe server. Each Scribe server communicated
with the central Zipkin collector or the Scribe server logging to the Hadoop
cluster. For Zipkin we used a 4-core. 8-gb RAM virtual machine, while for the
Hadoop cluster, as it was used only as a proof of concept, we used 2 2-core,
4-gb virtual machines. 

You can find some specs regarding the hardware and software infrastructure in
Tables
\ref{tab:hardware-specs} and \ref{tab:software-specs}. 

\begin{table}[H]
    \centering
    \begin{tabular}{ | l | l | }
        \hline
        Component & Description \\ \hline \hline
        CPU &  2 x Intel(R) Xeon(R) CPU E5645 @ 2.40GHz \cite{e5645} \\
         & Each CPU has six cores with Hyper-Threading enabled, which equals to 
         24 threads. \\ \hline
        RAΜ & 2 banks x 6 DIMMs PC3-10600 \\
        & Peak transfer rate: 10660 MB/s \\ \hline
        Hard disks & 12x 7.2k RPM 2TB SAS HDs, 12x 7.2k RPM 600GB SAS HDs, 6x \\
        & 100GB SSD SATA HDs \\ \hline
    \end{tabular}
    \caption{Test-bed hardware specs}
    \label{tab:hardware-specs}
\end{table}

The Ceph OSDs on the one node used two SSD disks in RAID 0, one each, and the
other two on the other node two SAS disks in RAID 0, one each.

\begin{table}[H]
    \centering
    \begin{tabular}{ | l | l | }
        \hline
        Software & Version \\ \hline \hline
        OS &  Debian Wheezy \\ \hline
        Linux kernel & 3.2.0-4 \\ \hline
        lttng-tools & 2.4 \\ \hline
    \end{tabular}
    \caption{Test-bed software specs}
    \label{tab:software-specs}
\end{table}

\section{Evaluation metrics}\label{sec:metrics}

In this section we analyze some metrics concerning the network and the system
overhead that BlkKin poses to the system. These metrics led us to the previous
deployment architecture.

\subsection{Thrift vs JSON format}
As we mentioned we created two similar Babeltrace plugins, one sending JSON
formatted messages and the other Thrift encoded messages to Scribe. It worths
measuring the message size in these two cases because the smaller the message
the lower the network overhead.
In order to evaluate this parameter we created a simple BlkKin message and send
it to a Scribe server both with the JSON and the Zipkin plugin. The message was
as simple as seen in Listing \ref{lst:message.json}

\ccode{A simple JSON formatted message}{message.json}

The packet size of the Zipkin-Thrift-encoded and the JSON encoded messages sent
to Scribe can be seen in Table \ref{tab:payloads}

\begin{table}[H]
    \centering
    \begin{tabular}{ | l | l | }
        \hline
        Protocol & Packet size in bytes \\ \hline \hline
        Thrift & 246 \\ \hline
        JSON  & 316 \\ \hline
    \end{tabular}
    \caption{Packet sizes per protocol used}
    \label{tab:payloads}
\end{table}

As it was expected, the Thrift message is much smaller that the JSON one even in
this case that the service and event names are small.

\subsection{Scribe vs relayd}
Another comparison we need to make to decide on the deployment architecture is
the network overhead created by Scribe and relayd. For example, instead of
running a local Scribe server, we could run a central realyd server per cluster
and then send the data to the central Scribe server.  Scribe, as mentioned
offers buffering and batch messaging. Also, the LTTng consumerd will be faster
when writing to localhost rather than to a remote server, thus reducing the
possibility to lose tracing information. However, we have to figure out the
amount of network traffic produced in the two cases. To evaluate this, we
created 10 simple messages as the previous ones and sent them to a local relayd
and then they were forward using Babeltrace live to the Scribe server. We
measured the network traffic to localhost and to the Scribe server.

Our first notice when using \texttt{tcpdump} on localhost is that relayd polls
consumerd on a specific interval to find out if there are any new data
available. So, we wouldn't like to have our cluster being flooded by polling
messages. Concerning the tracing data themselves, excluding polling, the sums of
all the TCP packets' payloads sent for the 10 messages mentioned, can be found
in Table \ref{tab:relayd-scribe} for each daemon.

\begin{table}[H]
    \centering
    \begin{tabular}{ | l | l | }
        \hline
        Daemon & Data size in bytes \\ \hline \hline
        Scribe & 1974 \\ \hline
        relayd  & 1079 \\ \hline
    \end{tabular}
    \caption{Data sent for 10 Scribe messages}
    \label{tab:relayd-scribe}
\end{table}

However, even if CTF-format is more compact that Thrift, we prefer to avoid the
polling messages in the cluster LAN and restrict them to localhost and to make
use of the Scribe batch messaging capability in favor of the less payload size
that CTF has to offer.

\subsection{System overhead}
\section{Using BlkKin to detect failures}\label{sec:failures}
