\chapter{Evaluation}\label{ch:evaluation}

In this section we will describe our experience from using BlkKin in a real
usecase senario. The instrumented infrastructure is described in Section
\ref{sec:infra}. After that, in Section \ref{sec:metrics} we analyse performance
metrics concerning the network and system overhead that justify our design and
deployment choices. Finally, in Section \ref{sec:failures} we explain how we
used BlkKin to identify system faults which are virtually inserted by us, but
reflect possible real failures or bottlenecks. 


\section{Instrumented infrastructure}\label{sec:infra} As a use-case, we used
BlkKin to instrument Archipelago and RADOS. These systems were examined in
Sections \ref{sec:archip-bkg} and \ref{sec:rados} respectively. Archipelago is
written is C and RADOS in C++. So we used the BlkKin library C++ wrapper for
the RADOS instrumentation. Instead of using the \texttt{archip-bench} tool,
which is part of Archipelago, to initiate IO requests, we instrumented the Qemu
Archipelago driver. So, using Qemu we start a virtual machine which has an
Archipelago drive and create different IO loads to this driver using
\texttt{fio}\footnote{http://linux.die.net/man/1/fio}. Thus we can track the IO
request from the time Qemu receives it until it is finally served by RADOS. 

The Qemu Archipelago driver receives the IO requests from Qemu and creates XSEG
requests for the VLMC. Qemu initiates the tracing information as well and
Qemu spans are the root spans. After that, these tracing information is
carried as part of the XSEG request. To do that, we needed to extend
\texttt{libxseg}\footnote{https://github.com/grnet/libxseg} and add the
tracing information needed as shown in Listing \ref{lst:blkin-info.h} nested in
the XSEG request. So, as far as Archipelago in concerned, the tracing
information is transmitted as part of the XSEG request. Each Archipelago peer
is considered a different service, with a different endpoint that creates a
single span per IO request in the general case. So, in the Zipkin UI we expect
to see each peer represented as a single bar, whose length indicates the time
this peer needed to serve this specific request.

Unlike Archipelago where the instrumentation was obvious, instrumenting RADOS
was more challenging. RADOS exposes a C-API (librados) which is used in the
Archipelago rados-peer. So, the first thing we did was to instrument the read
and write calls of this API. Then, we needed to extend the RADOS classes to
transfer the tracing information. In a nutshell, after librados, Ceph protocol
which is TCP-based transfers the IO request to the cluster. So, the tracing
information is encoded as part of the \texttt{MOSDOp} Ceph object. Then the
request after being decoded, enters a dispatch queue and waits to be served.
Based on the objects affected, a different placement group handles it. After
the dispatch queue, the request is handled by this pg's primary OSD and then
based on the replication factor, equal number of replication operations are
issued that follow the same route. Request handling includes journal access and
filestore access. In an attempt not to expose much of the RADOS internals, so
that the Zipkin UI would be self-explanatory even for someone that is not
familiar with the RADOS code architecture, we tried to instrument the code so
that we can extract information such as the time spent in the dispatch queue,
the network communication time, or the journaling duration and at the same time
we follow the causal relations used by Zipkin. For example, the IO handling by
the primary OSD causes the replication operations. So the replication
operations are children spans.

As far as the test-bed is concerned, we used two physical nodes LAN
interconnected, and set up two OSDs on each node. On one of this nodes we
installed Archipelago and Qemu. So, on the one node we had the running VM,
Archipelago and 2 OSDs and on the other just two OSDs. Each node had a whole
BlkKin stack running and a local Scribe server. Each Scribe server communicated
with the central Zipkin collector or the Scribe server logging to the Hadoop
cluster. For Zipkin we used a 4-core. 8-gb RAM virtual machine, while for the
Hadoop cluster, as it was used only as a proof of concept, we used 2 2-core,
4-gb virtual machines. 

You can find some specs regarding the hardware and software infrastructure in
Tables
\ref{tab:hardware-specs} and \ref{tab:software-specs}. 

\begin{table}[H]
    \centering
    \begin{tabular}{ | l | l | }
        \hline
        Component & Description \\ \hline \hline
        CPU &  2 x Intel(R) Xeon(R) CPU E5645 @ 2.40GHz \cite{e5645} \\
         & Each CPU has six cores with Hyper-Threading enabled, which equals to 
         24 threads. \\ \hline
        RAΜ & 2 banks x 6 DIMMs PC3-10600 \\
        & Peak transfer rate: 10660 MB/s \\ \hline
        Hard disks & 12x 7.2k RPM 2TB SAS HDs, 12x 7.2k RPM 600GB SAS HDs, 6x \\
        & 100GB SSD SATA HDs \\ \hline
    \end{tabular}
    \caption{Test-bed hardware specs}
    \label{tab:hardware-specs}
\end{table}

The Ceph OSDs on the one node used two SSD disks in RAID 0, one each, and the
other two on the other node two SAS disks in RAID 0, one each.

\begin{table}[H]
    \centering
    \begin{tabular}{ | l | l | }
        \hline
        Software & Version \\ \hline \hline
        OS &  Debian Wheezy \\ \hline
        Linux kernel & 3.2.0-4 \\ \hline
        lttng-tools & 2.4 \\ \hline
    \end{tabular}
    \caption{Test-bed software specs}
    \label{tab:software-specs}
\end{table}

\section{BlkKin tracing environment}
In order to set up the tracing environment, apart from the central Scribe
collector which could be either the Zipkin collector or a Scribe server
connected to HDFS, on each cluster node you have to follow the next steps to
live trace:

\begin{enumerate}
\item Start the local Scribe daemon \\ 
    The local Scribe daemon is configured to send the received messages to the
    next Scribe server. In case of connection loss or congestion, the data are
    stored locally and forward when the problem is solved.

\item Start LTTng live tracing \\
    Create a live tracing session and enable all the userspace events for it.
    Then start Babeltrace live and redirect its output to a named pipe.

\item Start the consumer \\
    Start the python consumer that reads from the named pipe and send data to
    the local Scribe server.  
\end{enumerate}

The overall procedure can be seen in Listing \ref{lst:start.sh}
\bcode{Setting up the tracing environment}{start.sh}
\section{Evaluation metrics}\label{sec:metrics}

In this section we analyze some metrics concerning the network and the system
overhead that BlkKin poses to the system. These metrics led us to the previous
deployment architecture.

\subsection{LTTng vs Syslog}
First of all, we should evaluate the use of LTTng versus another logging system
that is based on system calls every time some information needs to be logged.
SystemTap, DTrace or even syslog make a system call or a stop at a break point
whenever they need to trace something. This is claimed to have high overhead for
a system that need to run in full load. As mentioned the BlkKin library was
backend independent, so we we changed the LTTng backend for a syslog backend and
instrumented a simple client server application with the BlkKin library. Then we
measured the time eacg backend took to finish tracing. The application was a
simple two-process application, that communicated over a UNIX pipe. The server
created a root span, annotated and then passed a message to the client including
tracing information. The client created a child span annotated and answered
back. A single iteration triggers four annotations and we repeated this message
passing for 10000 times. The results can be seen in Table
\ref{tab:lttng-syslog}. 

\begin{table}[H]
    \centering
    \begin{tabular}{ | l | l | }
        \hline
        Backend & Time for 10000 iterations \\ \hline \hline
        LTTng &  1.8 sec \\ \hline
        Syslog & 3.38 sec \\ \hline
    \end{tabular}
    \caption{LTTng vs Syslog comparison}
    \label{tab:lttng-syslog}
\end{table}

As it was expected the syslog backend was about 90\% slower than LTTng even for
this single example. So LTTng was the only choice possible that combined the
low-overhead capability with the variety of tracing options such as user/kernel
space tracing and performance counter access.

\subsection{Thrift vs JSON format}
As we mentioned we created two similar Babeltrace plugins, one sending JSON
formatted messages and the other Thrift encoded messages to Scribe. It worths
measuring the message size in these two cases because the smaller the message
the lower the network overhead.
In order to evaluate this parameter we created a simple BlkKin message and send
it to a Scribe server both with the JSON and the Zipkin plugin. The message was
as simple as seen in Listing \ref{lst:message.json}

\ccode{A simple JSON formatted message}{message.json}

The packet size of the Zipkin-Thrift-encoded and the JSON encoded messages sent
to Scribe can be seen in Table \ref{tab:payloads}

\begin{table}[H]
    \centering
    \begin{tabular}{ | l | l | }
        \hline
        Protocol & Packet size in bytes \\ \hline \hline
        Thrift & 246 \\ \hline
        JSON  & 316 \\ \hline
    \end{tabular}
    \caption{Packet sizes per protocol used}
    \label{tab:payloads}
\end{table}

As it was expected, the Thrift message is much smaller that the JSON one even in
this case that the service and event names are small.

\subsection{Scribe vs relayd}
Another comparison we need to make to decide on the deployment architecture is
the network overhead created by Scribe and relayd. For example, instead of
running a local Scribe server, we could run a central realyd server per cluster
and then send the data to the central Scribe server.  Scribe, as mentioned
offers buffering and batch messaging. Also, the LTTng consumerd will be faster
when writing to localhost rather than to a remote server, thus reducing the
possibility to lose tracing information. However, we have to figure out the
amount of network traffic produced in the two cases. To evaluate this, we
created 10 simple messages as the previous ones and sent them to a local relayd
and then they were forward using Babeltrace live to the Scribe server. We
measured the network traffic to localhost and to the Scribe server.

Our first notice when using \texttt{tcpdump} on localhost is that relayd polls
consumerd on a specific interval to find out if there are any new data
available. So, we wouldn't like to have our cluster being flooded by polling
messages. Concerning the tracing data themselves, excluding polling, the sums of
all the TCP packets' payloads sent for the 10 messages mentioned, can be found
in Table \ref{tab:relayd-scribe} for each daemon.

\begin{table}[H]
    \centering
    \begin{tabular}{ | l | l | }
        \hline
        Daemon & Data size in bytes \\ \hline \hline
        Scribe & 1974 \\ \hline
        relayd  & 1079 \\ \hline
    \end{tabular}
    \caption{Data sent for 10 Scribe messages}
    \label{tab:relayd-scribe}
\end{table}

However, even if CTF-format is more compact that Thrift, we prefer to avoid the
polling messages in the cluster LAN and restrict them to localhost and to make
use of the Scribe batch messaging capability in favor of the less payload size
that CTF has to offer.

\subsection{System overhead}
In this subsection we evaluate the overhead that BlkKin poses to the
instrumented application. To do so, we created two different but typical IO
loads using fio. The first one was 2GB of random 4Kb writes and the second was
2GB of 64Kb sequential writes. These loads are diverse but really common and
help up evaluate our systems performance under different working conditions.
So, we ran fio in the virtual machine towards the Archipelago volume and
measured the bandwidth the system had under different conditions. The scenarios
we chose was without instrumentation, live tracing without sampling, live
tracing with 1/500 sampling and tracing without live support. The results can
be seen in Listings \ref{fig:4.png} and \ref{fig:64.png} and Tables
\ref{tab:4k-random} and \ref{tab:64k-sequential} respectively.

\diagramscale{Performance overhead for 4k random writes}{4.png}{0.82}
\diagramscale{Performance overhead for 64k sequential writes}{64.png}{0.82}

\begin{table}[H]
    \centering
    \begin{tabular}{ | l | l | }
        \hline
        Scenario  & Bandwidth in Kbytes/sec  \\ \hline \hline
        no tracing & 16887  \\ \hline
        stopped tracing & 16076 \\ \hline
        normal tracing & 14882  \\ \hline
        live tracing & 14941  \\ \hline
        live tracing sampling 500 & 15480  \\ \hline
    \end{tabular}
    \caption{Bandwidth for 64Kb sequential writes}
    \label{tab:64k-sequential}
\end{table}

\begin{table}[H]
    \centering
    \begin{tabular}{ | l | l | }
        \hline
        Scenario  & Bandwidth in Kbytes/sec  \\ \hline \hline
        no tracing & 1326.7  \\ \hline
        stopped tracing & 1247.1 \\ \hline
        normal tracing & 1100 \\ \hline
        live tracing & 1107.8  \\ \hline
        live tracing sampling 500 & 11927  \\ \hline
    \end{tabular}
    \caption{Bandwidth for 4Kb random writes}
    \label{tab:4k-random}
\end{table}

Although the above figures depict different loads and working conditions, BlkKin
performs in a similar way. We can see that when tracing is disabled, we have
about 5\% bandwidth degradation. This degradation is caused by the check whether
LTTng should trace or not. In case of full tracing the performance degradation
increases at about 12\% in case of sequential writes and at about 17\% in case
of small, random writes. This degradation is caused by the tracepoint function
calls that actually log the information. The case of 4Kb writes is more affected
because we have more IO request taking place fast one after the other. So the
LTTng load is greater. Both live tracing without sampling and normal tracing
affect the system in a similar way. The only change as far as LTTng is concerned
is that in case if live tracing the consumerd writes the tracing information to
a TCP socket instead of a local file descriptor. Finally, as we see in the
scenario of sampled tracing, the degradation is such that we can afford tracing
our system in production scale. In our scenario, we chose to sample 1/500 IO
requests. Depending on the system load, this sampling rate can be either reduced
or increased. It should be mentioned that in our instrumentation we used 113
annotations per single IO request in order to track its whole route. So, the
case of so-sampling, live tracing produced a significant amount of network
traffic and should be avoided not only because of this traffic, but also because
of it, the tracing information take more time to reach to Zipkin. Consequently,
we will need more time to detect a possible failure or problem.

\section{Using BlkKin to detect bottlenecks and failures}\label{sec:failures}

As it has become obvious, BlkKin can be used for various reasons and trying to
detect different problems either as part of the debugging process or as part of
a fault detection mechanism. In this subjection we evaluate both of these uses.

\subsection{Using Zipkin in debugging and system evaluation}
The most simple use of BlkKin is to analyze the system's performance, measure
the communication latencies, possible computation bottlenecks and generally to
get a general overview of the request's evolution.  For these reasons, the
Zipkin UI is really usefull. This UI enables us to do simple queries and to
understand the causal relations, to evaluate the time differences between the
different software layers and to access the key-value annotations or even make
queries based on them.


As seen in Figure \ref{fig:zipkin-overview.png} each span is represented as a
separate bar whose length is commensurate to the duration of this specific
processing phase. As far as time is concerned, each request is considered to
start at time 0 and all timestamp annotations have timestamps relative to the
first annotation of the trace. Also, if you click on a span that you are
interested about, you can access this span's annotations as can be seen in
Figure \ref{fig:zipkin-annotations.png}. In this screenshot we chose to
investigate an `OSD Handling op', namely the span that annotates the OSD's
actions to serve and IO request. At the top we can see each timestamp annotation
with its relative time, while at the bottom we can see the key-value
annotations. For example, this operation handling refers to the specific RADOS
object whose name can be seen as part of the binary annotation.

\diagram{Zipkin UI overview}{zipkin-overview.png}
\diagram{Zipkin UI Annotations view}{zipkin-annotations.png}

During this thesis evolution, Zipkin changed its UI. The old UI offered another
visualizing capability that is planned to be added to the new UI soon. This
capability was about endpoint dependencies. As it is seen in Figire
\ref{fig:zipkin-depend.png}, each service is depicted as a different circle
whose radius is commensurate to the processing duration of that specific phase.
This d3 visualization depicts exactly the information flow.

\diagram{Zipkin UI Dependencies View}{zipkin-depend.png}

In case that these visual representations are not enough and we want to extract
aggregate values, such as average values we have other choices. As mentioned
Twitter suggest to deploy Zipkin with Cassandra. However, we deployed Zipkin
using MySQL in order to have the ability of ad-hoc queries. So, depending on the
amount of tracing data we can extract the either from the database or by using
Hadoop. In our case, we used Hadoop to calculate the average journaling time per
OSD. So, simply by changing each local Scribe server's configuration file, we
chose to send data to a Scribe server connected to HDFS and not Zipkin. Then, we
created a Map-Reduce job to extract the information wanted as described in
Section \ref{sec:blkin-hadoop}. The information gathered can be seen in Table
\ref{tab:hadoop-journal}.

\begin{table}[H]
    \centering
    \begin{tabular}{ | l | l | }
        \hline
        OSD  & Journaling time ($\mu$sec)  \\ \hline \hline
        OSD1 & 401  \\ \hline
        OSD2 & 494  \\ \hline
        OSD3 & 475  \\ \hline
        OSD4 & 475  \\ \hline
    \end{tabular}
    \caption{Average journaling time for the 4 OSDs}
    \label{tab:hadoop-journal}
\end{table}

\subsection{Using Zipkin to detect abnormal behaviors}

The other BlkKin's use is as an online anomaly detection mechanism. According to
\cite{china-detector}, numerous techniques have been proposed for detecting
system anomalies. Among them, the simplest ones are the threshold-based
techniques which are a form of service level agreements (SLAs). They are very
useful on the condition that their users clearly know the key metric to monitor
and the best value of the thresholds in different scenarios. Unfortunately, it
is very difficult, even for an expert, to correctly choose the necessary metrics
to monitor and set the right values of the thresholds for different scenarios in
the context of today's complex and dynamic computer systems. In addition,
statistical learning or data mining techniques are widely employed to construct
probability models for detecting various anomalies in large-scale systems based
on some heuristics and assumptions, although these heuristics and assumptions
may only hold in some particular systems or scenarios. Other methods
(\cite{syslog-svm}), include artificial intelligence and neural networks, but
they require large training datasets.

In this part of the evaluation, we present how we used Zipkin to detect common
cases of anomaly that are possible to happen in a storage cluster. This
abnormalities refer to either network or disk problem. We tried a threshold
based alerting approach as a proof of concept, while the expressiveness and
correlation capabilities of BlkKin allow further investigation for more
correlative detection models. In our model, we conducted several tracing
sessions for different IO loads and filled a Hadoop cluster with these data.
Afterwards, through Map-Reduce jobs, we concluded on the thresholds we are going
to use for the specific hardware and depending on the expected load. The
thresholds included the communication times through the RADOS protocol and the
times for journaling and filestore access. Then, we inserted these threshold
values to the BlkKin monitoring tool. 

Apart from BlkKin, we needed to find a way to simulate a faulty situation. For a
network fault this was easy using the \texttt{tc} tool. This tool enables the
user to insert common network faults, such as packet loss, latency or packet
corruption.  However, the case of disk faults is more complex. To simulate a
faulty disk state there are multiple options. The most easy and the finally
chosen is to add a significant IO load using fio with multiple threads making
dummy read request to the specific disk. In our case of RADOS, since we
constantly write 4Mb objects, this will increase the disk latency. The second
choice is to use \texttt{cgroups} and the \texttt{blockio} controller. This
choice enables us to throttle a disk bandwidth, by throttling the bandwidth of
the processes writing to this disk, namely the OSD process. Another choice is to
use the device mapper to create a faulty sector. The final choice is to use the
Linux kernel fault injection
capabilities\footnote{https://www.kernel.org/doc/Documentation/fault-injection/fault-injection.txt}.
The complexity of the latter choices and the kernel dependencies made us avoid
them and just a dummy read IO load to the disk that we need to act as in case of
a fault.

So, to simulate a disk fault we added a read IO load in the OSD4 journal
partition. The results in Zipkin can be seen in Figure \ref{fig:disk-fault.png}.
As we can see, the Journaling span of the OSD4 takes significantly more time,
when compared to a normal behaviour as seen in Figure
\ref{fig:zipkin-overview.png} and as a result the whole request completion is
affected since journaling is synchronous and encapsulated within the request
process. 

\diagram{Injected disk fault - Journaling Latency}{disk-fault.png}

Of course, this journaling duration is above the accepted threshold. So, the
BlkKin monitoring UI would illustrate the problem as seen in Figure
\ref{fig:disk-fault-blkin.png}.

\diagram{Journaling Latency - BlkKin monitoring UI}{disk-fault-blkin.png}

A similar situation was simulated for a network problem. Using \texttt{tc} we
added 10 ms of latency to the NIC attached to the host where OSD1 and OSD3 run.
The result are extended communication times that can be seen as part of the
\texttt{Main} spans in Figure \ref{fig:network-error.png}. Again this faulty
situation is observed as seen in Figure \ref{fig:network-error-blkin.png}
through the BlkKin monitoring tool.

\diagram{Network Latency - Zipkin  UI}{network-error.png} 
\diagram{Network Latency - BlkKin monitoring UI}{network-error-blkin.png}
