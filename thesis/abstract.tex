\begin{abstractgr}
Η εξέλιξη της επιστήμης των υπολογιστών καθώς και οι αυξημένες απαιτήσεις σε
υπολογιστικούς πόρους αλλά και χώρους αποθήκευσης δεδομένων, έχει οδηγήσει στη
ραγδαία ανάπτυξη των κατανεμηνένων συστημάτων. Πλεόν μια υπηρεσία παρέχεται σαν
το αποτέλεσμα διαφορετικών διεργασιών οι οποίες αλληλεπιδρούν και εκτελούνται
σε διαφορετικούς υπολογιστικούς κόμβους. Ωστόσο, τόσο η αποσφαλμάτωση όσο και η
παρακολούθηση της σωστής λειτουργίας τέτοιων συστημάτων καθίσταται εξαιρετικά
δύσκολη λόγω της πολυπλοκότητας τους. H συμπεριφορά ενός κατανεμημένου
συστήματος εξαρτάται από τις εκάστοτες συνθήκες λειτουργίας, οι οποίες πρέπει
να ληφθούν σοβαρά υπόψιν στην λήψη σχεδιαστικών αλλά και βελτιωτικών αποφάσεων.

Η παρούσα διπλωματική μελετά το σχεδιασμό και την ανάπτυξη μιας υποδομής
παρακολούθησης κατανεμημένων εφαρμογών γραμμένων σε C/C++. Ο μηχανισμός
προσφέρει τη δυνατότητα παρακολούθησης της εφαρμογής σε πραγματικό χρόνο καθώς
αυτή εκτελείται σε πλήρεις συνθήκες λειτουργίας, επιβαρύνοντάς της ελάχιστα,
ώστε να διευρευνηθεί η συμπεριφορά της κάτω από διαφορετικά φορτία. Επιπλέον,
παρέχεται γραφικό περιββάλον μέσω του οποίου ο τελικός χρήστης μπορεί να
ερευνήσει περαιτέρω το σύνολο των αιτήσεων καθώς και τις σχέσεις εξάρτησης
μεταξύ των υποσυστημάτων που συνθέτουν το τελικό προς παρακολούθηση σύστημα. Ο
μηχανισμός αυτός ονομάζεται BlkKin και βασίζεται πάνω σε τεχνολογίες ανοιχτού
κώδικα, προσπαθώντας να εκμεταλλευτεί τα δυνατά στοιχεία κάθε υποσυστήματος και
συνδιάζοντάς τα να επιτύχει το ζητούμενο στόχο.

Η συνεισφορά αυτής της διπλωματικής έγγυται στην υλοποίση του μοντέλου
καταγραφής συγκεκριμένα για εφαρμογές χαμηλής επιβάρυνσης, καθώς και των
συνδετικών τμημάτων μεταξύ των διάφορων υποσυστημάτων που συνολικά υλοποιούν το
BlkKin.

Ο μηχανισμός αυτός χρησιμοποιήθηκε για την παρακολούθηση αιτήσεων Ε/Ε από
εικονικές μηχανές πάνω από Qemu προς το Archipelago, ένα κατανεμημένο συστήματο
αποθήκευσης σε περιβάλλον υπολογιστικού νέφους (cloud computing environment). Η
πορεία μιας τέτοια αίτησης καταγράφηκε μέχρι να ικανοποιηθεί από το
κατανεμηνένο σύστημα αποθήκευσης block, το RADOS. Επομένως δίνεται η δυνατότητα
γραφικής απεικόνησης των διαφορετικών στρωμάτων λογισμικού που απαιτούνται να
συνεργαστούν για την ικανοποίηση της συγκεκριμένης αίτησης.	
    \begin{keywordsgr}
    καταγραφή αιτήσεων, παρακολούθηση, κατανεμημένο σύστημα αποθήκευσης, RADOS,
    Archipelago, LTTng, Dapper
	\end{keywordsgr}
\end{abstractgr}

\begin{abstracten}
Distributed storage systems require special treatment concerning their
monitoring and tracing. In this thesis, we present the design and implementation
of BlkKin, a mechanism that provides the necessary infrastructure for tracing
software-defined storage systems. It enables live tracing and inserts minimal
overhead to the instrumented system, so that it can continue working effectively
in production scale. Its tracing semantics lead to a cross-layered, end-to-end
representation of the system and how the IO requests interact with it.
End-to-end request tracing enables elaborate information extraction concerning
specific system parts, specific workloads or specific system resources, allowing
the, otherwise impossible, localization of latencies and bottlenecks. BlkKin is
based on open-source technologies and provides a full stack implementation with
a data collector, data aggregator and a Web UI for visualizing the tracing
information, while it can be easily incorporated by any system, not only
distributed storage, in need of such a tracing framework.
	\begin{keywordsen}
    distributed storage, tracing, real-time, low-latency, low-overhead, LTTng, 
    Zipkin, Dapper, metrics, RADOS, Archipelago
	\end{keywordsen}
\end{abstracten}

\begin{acknowledgementsgr}
Με την παρούσα διπλωματική ολοκληρώνεται μια πορεία 5 χρόνων στη Σχολή
Ηλεκτρολόγων Μηχανικών και Μηχανικών Υπολογιστών. Διάστημα μέσα στο οποίο ήρθα
αντιμέτωπος με πολλές διαφορετικές προκλήσεις αλλά και δυσκολίες, η αντιμετώπιση
των οποίων όμως με έκανε να καταλάβω τι πραγματικά σημαίνει να είσαι μηχανικός.

Η διπλωματική αυτή, που αποτελεί το επιστέγασμα αυτής της προσπάθειας,
εκπονήθηκε υπό την καθοδήγηση του καθηγητή Νεκτάριου Κοζύρη, τον οποίο θα ήθελα
να ευχαριστήσω θερμά γιατί ήταν αυτός που από το μάθημα της Αρχιτεκτονικής
Υπολογιστών ακόμα με εισήγαγε στην έννοια των υπολογιστικών συστημάτων και γιατί
μου έδωσε τη δυνατότητα να ασχoληθώ με το συγκεκριμένο τομέα των κατανεμημένων
συστήματων και του cloud computing. 

Επιπλέον, οφείλω ένα μεγάλο ευχαριστώ στον Δρα. Βαγγέλη Κούκη του οποίου το
πάθος να λύσουμε πραγματικά προβλήματα οδήγησε στην υλοποίηση της παρούσας
διπλωματικής και σε ένα εργαλείο το οποίο ξεφεύγει από τα πλαίσια της θεωρίας
και μπορεί να αξιοποιηθεί στην πράξη. Οι συζητήσεις μαζί του πάντα έδιναν λύσεις
στα προβλήματα με τα οποία ήρθα αντιμέτωπος και κατάφεραν να μου μεταδώσουν την
αγάπη για τα υπολογιστικά συστήματα.

Δεδομένου ότι μεγάλο τμήμα της συγκεκριμένης δουλειάς πραγματοποιήθηκε στα
πλαία του προγράμματος Google Summer of Code 2014 στο οποίο συμμετείχα στο
LTTng project, θεωρώ υποχρέωσή μου να ευχαριστήσω τον μέντορά μου στη
συγκεκριμένη προσπάθεια Jeremie Galarnau ο οποίος πίστεψε στην ιδέα της
συγκεκριμένης εργασίας και με στήριξε σε ό,τι αφορά το σύστημα LTTng.

Ακόμα, θα ήταν παράλειψη να μην ευχαριστήσω τον άνθρωπο με τον οποίο
συνεργάστηκα περισσότερο απ'όλους τον περασμένο χρόνο, τον Υ.Δ Φίλιππο Γιαννάκο,
ο οποίος με βοήθησε έμπρακτα τόσο στην κατανόηση των ποικίλων συστημάτων που
χρησιμοποιήθηκαν όσο και στην σχεδίαση του τελικού μηχανισμού.

Tέλος, θέλω να ευχαριστήσω την οικογένειά μου και τους φίλους μου οι οποίοι
στάθηκαν πάντα δίπλα μου, ανεχόμενοι τις παραξενιές μου, σε όλη της διάρκεια
αυτής της πορείας, όντας πάντα εκεί για να με στηρίξουν, να μου δώσουν δύναμη
να συνεχίσω αλλά και για να μου χαρίσουν ωραίες αναμνήσεις οι οποίες θα με
συντροφεύουν πάντα στη μετέπειτα πορεία μου.
\end{acknowledgementsgr}
