\chapter{Introduction}\label{ch:intro}

When back in April 1965 Gordon E. Moore stated the following 
\begin{quotation}
    ``The complexity for minimum component costs has increased at a rate of 
    roughly a factor of two per year. Certainly over the short term this rate 
    can be expected to continue, if not to increase. Over the longer term, the 
    rate of increase is a bit more uncertain, although there is no reason to 
    believe it will not remain nearly constant for at least 10 years. That 
    means by 1975, the number of components per integrated circuit for minimum 
    cost will be 65,000. I believe that such a large circuit can be built on a 
    single wafer.''\cite{Moore}
\end{quotation}

had no idea that he had actually started a race among the academia and the
industry to overcome or at least abide the his law.

During the early years of this race, since the field was premature, the
development was based on the evolution in VLSI technology which went hand in
hand with the evolution in computer architecture. The more and faster
transistors resulted in the achievement of instruction level parallelism (ILP).
From 1975 to 2005 the endeavour put in computer architecture resulted in
technological advances varying from deeper pipelines and faster clock speeds to
superscalar architectures. But in around 2005 the ILP wall was hit. Transistors
could not be utilized to increase serial performance, logic became too complex
and performance attained was very low compared to power consumption. This lead
to the creation of multicore systems and entered the programmers to the jungle
of parallel software. So far the evolution was almost in accordance with the
famous law. However, in around 2009 to 2011, it was the power wall's time to be
hit. The famous power equation $P=cV^2f$ along with the CPU to memory gap
(Figure \ref{fig:mem_cpu_gap.jpg}) led to the technological burst of
distributed and cloud computing as a final attempt to abide by Moore's law.

\diagramscale{CPU to Memory gap}{mem_cpu_gap.jpg}{0.8}i

In 2009 Amazon.com introduced the Elastic Compute Cloud and since then the term
`cloud' is one of the hottest buzzwords not only among the industry and academia
but also among everyday people that take advantage of the `power of cloud'.
Although the term may be vague, the definition of cloud computing, according to
NIST (National Institute of Standards and Technology), is the following:

\begin{quotation}
    ``Cloud computing is a model for enabling ubiquitous, convenient, on-demand
    network access to a shared pool of configurable computing resources (e.g.,
    networks, servers, storage, applications, and services) that can be rapidly
    provisioned and released with minimal management effort or service provider
    interaction.  This cloud model is composed of five essential characteristics
    ,three service models, and four deployment models.''\cite{clouddef}
\end{quotation}

Different in motivation, though same in the outcome, the evolution of storage
systems followed a different route but again resulted in distributed storage
systems that serve the demands of cloud computing. Several technological aspects
,like the advent of RAID, and economic factors, like the decreasing hard drives'
price, led to the evolution of distributed storage, which nowadays in the only
way to cope with the profuse amount of data, no matter how fast the storage
media may become.

In the previous brief computer chronology, I kept describing bottlenecks and
walls to be overcome. However, it not clear how these bottlenecks become
obvious and how scientists can be sure that they have reached one's
technology's limits before moving on to the next one. The answer to the
previous questions has always been given through \textit{tracing}. Tracing is a
process recording information about a program's execution, while it is being
executed. These information may be low level metrics like performance counters
or time specific metrics in order to evaluate system's latencies and
throughput. Tracing data are mostly useful for developers and can be used for
debugging, performance tuning and performance evaluation. From the single-cpu,
integrated computer to the hundreds-node cloud infrastructure, trace and
performance engineers face challenging problems that vary from platform to
platform, but in every case play a vital role the system's design and
implementation.

Cloud and distributed computing provided trace engineers with even more
challenging problems. The system scale is now much greater and program
execution is far from deterministic and can take place in any cluster node. So
each program execution is not bounded to a specific context. Other problems
that needed solving was data and time correlation between the different
computing nodes. Also, unlike single chip platforms that can be individually
traced and evaluated, cloud infrastructures need to be traced with full-load
under production conditions. This sets more restrictions concerning the overhead
that tracing adds to the application. Finally, tracing is notorious about the
amount of data that produces. So distributed and cloud tracing demands the use
of distributed data storage systems and processing methods like distributed
NOSQL databases and Map-Reduce frameworks for keeping and manipulating the
tracing information. 

So to sum up, as described by every software design model, the system
verification consists a major part of a system's implementation and working
process. Verification is achieved through monitoring and tracing. Depending on
the system's nature, the process of tracing and monitoring as well as the tools
used may vary. Picking the right tracing tools that will reveal the system's
vulnerabilities and faults can be very demanding and it is the performance
engineer's taks to bring them to light, respecting all the prerequisites set by
the system.

\section{Thesis motivation}
The motivation behind this thesis emerged from concerns about the storage 
performance of the Synnefo \footnote{www.synnefo.org/} cloud software, which 
powers the \okeanos \footnote{https://okeanos.grnet.gr/} public cloud service 
\cite{okeanos}. I will briefly explain what \okeanos and Synnefo are in the 
following paragraphs.

\okeanos is an IaaS (Infrastructure as a Service) that provides Virtual 
Machines, Virtual Networks and Storage services to the Greek Academic and 
Research community. It is an open-source service that has been running in 
production servers since 2011 by GRNET S.A.
\footnote{Greek Research and Technology Network, https://www.grnet.gr/}

Synnefo \cite{synnefo} is a cloud software stack, also created by GRNET S.A., 
that implements the following services which are used by \okeanos:

\begin{itemize}
    \item \textit{Compute Service}, which is the service that enables the 
        creation and management of Virtual Machines.
    \item \textit{Network Service}, which is the service that provides network 
        management, creation and transparent support of various network 
        configurations.
    \item \textit{Storage Service}, which is the service responsible for 
        provisioning the VM volumes and storing user data.
    \item \textit{Image Service}, which is the service that handles the 
        customization and the deployment of OS images.
    \item \textit{Identity Service}, which is the service that is responsible 
        for user authentication and management, as well as for managing the 
        various quota and projects of the users.
\end{itemize}

Synnefo provides each virtual machine with at least one virtual volume
provisioned by the Volume Service called Archipelago\cite{archip-paper} and will
be furthered detailed in Section \ref{sec:archip-bkg}. This thesis' purpose is
to provide the developer or the system administrator with a cross-layer
representation accompanied with the equivalent metrics and time information of
an I/O request's route within the infrastructure from the time it is created
inside the virtual machine till it is finally served by the storage backend. The
design and implementation has to be done respecting the following two
prerequisites:

\begin{itemize}
    \item The tracing information should be gathered and processed in real-time
        from every node participating in the request serving.
    \item The tracing infrastructure should add the least possible overhead to
        the instrumented system, which should continued working properly 
        production-wise
\end{itemize}

After the end of the tracing infrastructure implementation, the developer should
be able to identify the distinct phases and the duration of each that an IO
request passes through, measure communication latencies between the different
layers and collect all the necessary information (chosen by him) that would help
him understand the full context under which this specific request was served.
All these information can be used for software faults detection and performance
tuning as well as hardware malfunctions and faults like disk or network failures
that would be difficult to detect otherwise.

The novelty of this thesis consists in combining live cross-layer, multi-node
data aggregation, which is typical for monitoring but not for tracing, with the
precision and accuracy of tracing, respecting a hard prerequisite of low
overhead. Previous tracing infrastructures offered only partial solutions. Some
of them would separate the tracing from the working phase because of the great
added overhead, others provided no mechanism for data correlation, while the
traditional monitoring systems did not meet our low-level tracing needs.

The proposed system is called \textit{BlkKin}. It is designed respecting the
aforementioned prerequisites and makes use of the latest tracing semantics and
infrastructures employed by great tech companies like Google and Twitter.  

\section{Thesis structure} 
This thesis is structured as follows:

\begin{description}
\item[Chapter \ref{ch:bkg}] \hfill \\
We provide all the necessary background information, both technical and
theoretical, so that the reader becomes familiar with the concepts and ideas
described. We cite a brief overview of the existing tracing and monitoring
infrastructures along with their advantages and disadvantages and how they
affected BlkKin's design and implementation. Finally, we describe Archipelago
and RADOS, the two systems we instrumented as a BlkKin's proof of concept.

\item[Chapter \ref{ch:lttng}] \hfill \\
We analyze Linux Trace Toolkit - next generation (LTTng), which is one of the
basic building blocks of BlkKin. LTTng is used as BlkKin tracing backend.

\item[Chapter \ref{ch:dapper-zipkin}] \hfill \\
We describe the tracing concepts we employed in BlkKin so that we can achieve
the needed expressiveness. We also present Zipkin, which is an open-source
implementation of these tracing concepts and another BlkKin's building block.

\item[Chapter \ref{ch:design}] \hfill \\
We describe the BlkKin's design and architecture, the communication protocols
used and the tracing information flow.

\item[Chapter \ref{ch:implementation}] \hfill \\
We analyze the process of creating BlkKin, our contributions to the tracing
infrastructure and the means used or created to extract the information needed
in order to serve the different roles that BlkKin can play.

\item[Chapter \ref{ch:evaluation}] \hfill \\
We cite our experience of using BlkKin in Archipelago and RADOS instrumentation
and its use as a debugging and an alerting mechanism.

\item[Chapter \ref{ch:conclusion}] \hfill \\
We provide some concluding remarks and give some future work that could be done
to improve and evolve BlkKin.
\end{description}
