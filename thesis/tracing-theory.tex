\chapter{Tracing semantics}\label{ch:dapper-zipkin}

Apart from the mechanism that enable tracing, it is very important to choose
what kind of information concerning the application is finally logged. A wise
choice will ease the process of data correlation and assumption extraction. In
this chapter, we cite the different schools behind tracing semantics, we analyze
our tracing choice and finally we discuss how this choice is implemented.

\section{Data correlation}\label{sec:data-cor}

Data resulting after tracing can be very bulky. Consequently, the process of
extracting the information needed that triggered tracing is challenging. Data
should be correlated and only the needed parts of the logs should be isolated
and processed in order to extract meaningful information. This requires that
tracing data are capable of being correlated. By data correlation we refer to
data grouping concerning a specific subsystem or a specific system request.

Although there have been proposed many different tracing schemes, according to
specific applications needs, all these schemes can be summarized into two main
categories. According to Google's Dapper paper\cite{dapper} these categories
are:

\begin{description}

\item[black-box] schemes \cite{blackbox1, blackbox2} assume there is no
additional information other than the message record described above and use
statistical regression techniques to infer that association.

\item[annotation-based] schemes \cite{magpie, xtrace} rely on applications or
middleware to explicitly tag every record with a global identifier that links
these message records back to the originating request.

\end{description}

While black-box schemes are more portable than annotation-based methods, they
need more data in order to gain sufficient accuracy due to their reliance on
statistical inference. The key disadvantage of annotation-based methods is,
obviously, the need to manually instrument programs by adding instrumentation
points in their source code.

As mentioned, BlkKin wants to achieve an end-to-end tracing so that latencies
and faults between the different subsystem layers to become obvious.
Consequently, we decided to use an annotation-based tracing schema.

\section{Dapper tracing concepts}\label{sec:dapper}

Dapper\cite{dapper} is a large scale distributed systems tracing infrastructure
created by Google. It uses an annotation-based tracing scheme, which enables
Google developers to monitor Google' infrastructure only by instrumenting a
small set of common libraries (RPC system, control flow). Although it is closed
source, the tracing semantics used in Dapper are publicly available and have
been used in BlkKin's development. Indeed, Google proposed a complete
annotation-based scheme, which describes the following concepts for tracing:

\begin{description}
\item[annotation]
The actual information being logged. There are two kinds of annotations. Either
\emph{timestamp}, where the specific timestamp of an event is being logged or
\emph{key-value}, where a specific key-value pair is being logged.
\item[span]
The basic unit of the process tree. Each specific processing phase can be
depicted as a different span. Each span should have a specific name and a
distinct span id. It is important to note that each span can contain information
from multiple hosts.
\item[trace]
Every span is associated with a specific trace. A different trace id is used to
group data so that all spans associated with the same initial request share the
common trace id. For our case, information concerning a specific IO request
share the same trace id and each distinct IO request initiates a new trace id.
\item[parent span]
In order to depict the causal relationships between different spans in a single
trace, parent span id is used. Spans without a parent span ids are  known as
\emph{root spans}.
\end{description}

The previous concepts fit out demands for end-to-end tracing. So, we implemented
them in a tracing library for C/C++ applications, which is described thoroughly
in Section \ref{sec:contribution}.
 
\section{Zipkin: a Dapper open-source implementation}\label{sec:zipkin}

Dapper does not only describe the tracing semantics mentioned before, but is a
full stack tracing infrastructure which includes subsystems to aggregate data
per host, a central collector, a storage service and a user interface to query
across the collected information. BlkKin, also has the same needs. So, to cover
some of them, instead of rewriting the needed subsystems, we decided to use
\textit{Zipkin}.

\diagram{Zipkin Architecture}{zipkin-architecture.png}

Zipkin\footnote{http://twitter.github.io/zipkin/} is an open-source
implementation of the Dapper paper by Twitter. It is used to gather timing data
for all the disparate services at Twitter. It manages with both the collection
and lookup of this data through a Collector and a Query service as well as the
data presentation through a Web UI. Zipkin is written in Scala, while the Web
interface is written in Ruby and Javascript using the
D3.js\footnote{http://d3js.org/} framework. So, Zipkin is a full-stack system
that encapsulates the Dapper tracing semantics out of the box. This is why we
chose to use Zipkin.

Concerning transportation, Zipkin uses Scribe, which enables Zipkin to Scale.
So, in order to feed Zipkin with data a Scribe client is needed. As mentioned
in Chapter \ref{sec:logging-bkg} about Scribe, a category and a message is
needed to log to Scribe. Zipkin messages themselves are also Thrift encoded so
that the collector can handle them and add them in the database. Zipkin thrift
messages are encoded according to the following thrift file (Listing
\ref{lst:zipkin.thrift}) 

This thift file defines:

\begin{description}
\item[Endpoint] is the location where an annotation took place. An endpoint is
identified by its name, IP and port.
\item[Annotation] is the tracing information itself, exactly like the Dapper
annotation
\item[Span] is also the Dapper span identified by its name, span, trace and
parent ids and can contain multiple annotations.
\end{description} 

\ccode{Zipkin message thrift definition file}{zipkin.thrift}

Concerning the final data storage Zipkin provides various choices including SQL
databases like SQLite, MySQL, and PostgreSQL as well as NoSQL databases like
Cassandra and even Redis. Twitter suggests using Cassandra in Zipkin
installations. They support that it performs well. However, Cassandra imposes
limitations concerning ad-hoc queries. So we used Zipkin both with Cassandra and
MySQL based on experiment and the information we wanted to extract.

The resulting Zipkin architecture can be seen in Figure
\ref{fig:zipkin-architecture.png}. Although we did not use it, Twitter
encourages using Zookeeper for trace synchronization purposes. However, in our
case and scale there was no such need.
