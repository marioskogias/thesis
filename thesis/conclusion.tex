\chapter{Conclusion}\label{ch:conclusion}

\section{Concluding Remarks}

This thesis handles with the problem of low overhead distributed tracing aimed
to analyze software defined storage systems. In this kind of multi-layered
software architectures finding and locating bottlenecks and potential or even
real faults is pretty challenging because of their complexity. After evaluating
both various logging and tracing mechanism as well as different tracing schemas,
we decided to implement our own tracing infrastructure called BlkKin. BlkKin is
based on opensource technologies, namely LTTng and Zipkin, and implements the
tracing semantics used by Google's Dapper tracing infrastructure. So BlkKin is a
low-overhead tracing infrastructure that enables live tracing for applications
written in C/C++ and provides two distinct user interfaces, so that the end user
can take real time information about the system.

In order to accomplish this endeavour, we had to work with the LTTng community
and extend their software so that it can communicate with Zipkin and create an
instrumentation library for easy application instrumentation.

As a proof of concept for our system, we instrumented Archipelago and RADOS
source code. Consequently, we were able to track an IO request's route from the
time Qemu accepted it until it was finally served by RADOS and investigate any
kind of latencies or bottlenecks each processing phase may have. Also, we
simulated different faulty situations that a distributed storage system may face
and investigated the use of BlkKin as an alerting mechanism for such kind of
faults.

However, this work was just the beginning of cross-layered distributed tracing,
since it provided the framework for further investigation. Our mechanism can be
used in any kind of low-overhead application that needs a tracing and
visualization infrastructure.  So, far the Ceph community has showed interest
for BlkKin and we are in close contact so that BlkKin can become the main
tracing infrastructure for RADOS. 
 
\section{Future Work}
BlkKin future plans/work include:
\begin{itemize}
\item Better live support, after the Babeltrace plugin system is released.
\item Better sampling mechanism that takes into account the request's special
characteristics, so that meaningful information is not lost because of sampling
\item Offer Babeltrace plugins as part of the LTTng source tree in the form of a
Python module availabe at \texttt{pip}.
\item Use BlkKin in different kinds of distributed systems such as parallel
applications, MPI for example.
\end{itemize}

As far as the RADOS instrumentation is concerned, future work includes:
\begin{itemize}
\item Better RADOS instrumentation not only for read and write requests. This
instrumentation requires deep knowledge of the software's internals so that its
bottlenecks are found.
\item Implementation of a correlative alerting system that relates the
replica operations with the cluster's health in order to avoid our threshold
based alerting mechanism.
\item Use of BlkKin tracing data from RADOS instrumentation to create an
AI-based failure detector.
\end{itemize}
