\chapter{Theoretical Background}\label{ch:bkg}

In this chapter we provide the necessary background to familiarize the reader
with the main concepts and mechanism used later in the document. For every
subsystem employed in BlkKin we briefly describe some counterparts justifying
our choice. The approach made is rudimentary, intended to introduce a reader
with elementary knowledge on distributed systems.

Specifically, Section \ref{sec:storage} covers the concepts around distributed
storage systems and they difficulties concerning their monitoring.  In Section
\ref{sec:archip-bkg} we describe Archipelago, Synnefo's Volume Service, and how
IO requests initiated within the virtual machine end up being served by a
distributed storage system. In Section \ref{sec:tracing-bkg} we explain the need
for tracing and cite various open-source tracing systems  with their advantages
and disadvantages. Finally, in Section \ref{sec:logging-bkg} we describe the
different needs covered by logging and cite some popular logging systems.


\section{Distributed storage systems}\label{sec:storage}

\section{Archipelago}\label{sec:archip-bkg}

\section{Tracing Systems}\label{sec:tracing-bkg}

\section{Logging Systems}\label{sec:logging-bkg}
