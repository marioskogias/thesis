\chapter{BlkKin Design}\label{ch:design}

In the previous chapters we described the various challenges faced when dealing
with distributed tracing and tools that can be used in order to achieve our
tracing goals. In this chapter we cite the design of tracing infrastructure
called BlkKin. The name comes from the amalgamation of \textit{Block storage}
and \textit{Zipkin}, which is one of the used building blocks and was described
thoroughly in Section \ref{sec:zipkin}.

By building BlkKin we want to create a tracing infrastructure intended to cover
the tracing needs created in software defined and distributed storage systems
(but of course not restricted to them). We defined the following prerequisites
that should be present in BlkKin:

\begin{description}

\item[low-overhead tracing] \hfill \\
The traced system should be able to continue working in production scale serving
real workloads in order to locate deficiencies and faults that are not obvious
in debugging or tracing mode.

\item[live-tracing]
BlkKin should be able to send traces at the time the are being produced so that
the developer or the administrator can have an overview of the system at that
specific time.

\item[Dapper tracing semantics]
Tracing logic should be implemented in accordance with the concepts used by
Dapper so that causal relationships and cross-layer architecture are depicted.

\item[User interface]
BlkKin should provide various endpoints for the end user to collect and analyze
data. One of those should be a graphical user interface that gives a graphical
overview of the system's performance per specific layer.

\end{description}
